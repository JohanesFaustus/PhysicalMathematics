\documentclass[../../main.tex]{subfiles}
\begin{document}
\subsection*{Appendix: Frobenius' Method}
I will demonstrate this technique. Consider the following differential equation.
\begin{equation*}
  x^2 y''+ 4xy' + (x^2 + 2)y = 0
\end{equation*}
The solution will take the form 
\begin{equation*}
  y=\sum_{n=0}^{\infty} a_nx^{n+s}
\end{equation*}
Substituting this into each terms, we have 
\begin{align*}
  x^2y''&=\sum_{n=0}^{\infty} (n+s) (n+s-1)a_nx^{n+s} \\
  4xy'&=\sum_{n=0}^{\infty} 4(n+s) a_nx^{n+s}\\
  xy&=\sum_{n=0}^{\infty} a_n x^{n+s+2}\\
  2y&=\sum_{n=0}^{\infty} 2a_nx^{n+s}
\end{align*}
Then we put them into table.
\begin{center}
$\begin{array}{c || c c c}
  &x^{n+s}&x^s&x^{s+1}\\
  \hline\hline
  x^2y''&(n + s)(n + s - 1)a_n &s(s-1)a_0  &s(s+1)a_1 \\
  4xy'&4(n+s)a_n &4sa_0 &4(s+1)a_1 \\
  x^2y&a_{n-2} &-&- \\
  2y& 2a_n& 2a_0& 2a_1\\
\end{array}$
\end{center}
Using the terms on $x^{s}$ column, we have the following indicial equation.
\begin{align*}
  s(s-1)a_0+4sa_0+2a_0&=0\\
  a_0 \left[s(s+3)+2 \right] &=0
\end{align*}
Since $a_0$ cannot be zero, we write 
\begin{equation*}
  s^2+3s+2 =0
\end{equation*}
By solving the indicial equation we obtain $s=(-1,-2)$. From the $x^{n+s}$, we obtain the general formula for $a_n$ in terms of $a_{n-2}$
\begin{align*}
  a_n\left[(n+s)(n+s+3)+2\right]&=-a_{n-2}
\end{align*}
We also obtain the fact the value of $a_1$ is zero, proved by the terms in $x^{s+1}$ column 
\begin{equation*}
  \begin{rcases*}
    a_1\left[(s+1)(s+4)+2\right]=0\\
    s=(-1,-2)
  \end{rcases*}\implies a_0=0
\end{equation*}

Since we have two value of $s$, we first consider the case for $s=-1$. The general $a_n$ formula evaluated into 
\begin{equation*}
  a_n=-\frac{a_{n-2}}{(n-1)(n+2)+2}=-\frac{a_{n-2}}{n^2+n}=-\frac{a_{n-2}}{n(n+1)}
\end{equation*}
The values of $a_n$ for few $n$ are as follows 
\begin{align*}
  a_2&=-\frac{a_0}{3!}\\
  a_4&=-\frac{a_2}{4\cdot 5}=\frac{a_0}{5!}\\
  a_6&=-\frac{a_4}{6\cdot7}=-\frac{a_0}{7!}
\end{align*}
Thus the solution for this case is 
\begin{multline*}
  y_{-1}=\sum_{n=0}^{\infty} a_nx^{n-1}=\frac{a_0}{x}-\frac{a_0}{3!}x+\frac{a_0}{5!}x^3-\frac{a_0}{7!}x^5 +\dots\\
  =\frac{a_0}{x^2}\left(x-\frac{x^3}{3!}+\frac{x^5}{5!}-\frac{x^7}{7!}+\dots\right)=\frac{a_0}{x^2}\sin x
\end{multline*}

For the case of $s=-2$, the general $a_n$ formula evaluated into
\begin{equation*}
  a_n=-\frac{a_{n-2}}{(n-2)(n+1)+2}=-\frac{a_{n-2}}{n^2-n}=-\frac{a_{n-2}}{n(n-1)}
\end{equation*}
The values of $a_n$ for few $n$ are as follows 
\begin{align*}
  a_2&=-\frac{a_0}{2!}\\
  a_4&=-\frac{a_2}{4\cdot 3}=\frac{a_0}{4!}\\
  a_6&=-\frac{a_4}{6\cdot 5}=-\frac{a_0}{6!}
\end{align*}
Thus the solution for this case is 
\begin{multline*}
  y_{-2}=\sum_{n=0}^{\infty} a_nx^{n-2}= \frac{a_0}{x^2}-\frac{a_0}{2!}+\frac{a_0}{4!}x^2-\frac{a_0}{6!}x^4 +\dots\\
  =\frac{a_0}{x^2}\left(1-\frac{x^2}{2!}+\frac{x^4}{4!}-\frac{x^6}{6!}+\dots\right)=\frac{a_0}{x^2}\cos x
\end{multline*}

Hence the complete form of the solution is 
\begin{equation*}
  y=\frac{a_0}{x^2}\left(\cos x +\sin x\right)
\end{equation*}

\subsection*{Appendix: Differential Equation Study Guide}
\subsubsection*{First Order Equations.} General Form of ODE
\begin{align*}
\dfrac{dy}{dx}=f(x,y)
\end{align*}

Initial Value Problem
\begin{equation*}
y'=f(x,y),\ y(x_0) = y_0
\end{equation*}

\subsubsection*{Linear Equations.} General Form:
\begin{align*}
y'+p(x)y=f(x)
\end{align*}
Integrating Factor
\begin{align*}
 \mu(x) &= e^{\int p(x)dx}\\
  \implies & \dfrac{d}{dx}\left( \mu(x) y \right) = \mu(x) f(x)\\
\end{align*}
General Solution
\begin{equation*}
y=\frac{1}{\mu(x)}\left( \int \mu(x) f(x) dx + C\right)
\end{equation*}

\subsubsection*{Homogeneous Equations.} General form
\begin{align*}
 y'=f(y/x)
\end{align*}
Substitution
\begin{equation*}
y=zx  \implies y'=z + xz'
\end{equation*}
The result is always separable in $z$: 
\begin{equation*}
\dfrac{dz}{f(z)-z} = \dfrac{dx}{x}
\end{equation*}

\subsubsection*{Bernoulli Equations.} General Form
\begin{align*}
y'+p(x)y=q(x)y^n
\end{align*}
Substitution
\begin{align*}
z = y^{1-n}
\end{align*}
The result is always linear in $z$:
\begin{equation*}
 z' +(1-n)p(x) z = (1-n)q(x)
\end{equation*}

\subsubsection*{Exact Equations.} General Form
\begin{align*}
M(x,y)dx + N(x,y)dy = 0 
\end{align*}
Text for Exactness
\begin{equation*}
\dfrac{\partial M}{\partial y}=\dfrac{\partial N}{\partial x}
\end{equation*}
Solution
\begin{equation*}
\phi=C
\end{equation*}
where
\begin{equation*}
M=\dfrac{\partial \phi}{\partial x}\quad\text{ and }\quad N=\dfrac{\partial \phi}{\partial y}
\end{equation*}

\subsubsection*{Method for Solving Exact Equations. }

\begin{enumerate}
\item Let $\phi=\int M(x,y)dx + h(y)$
\item Set $\dfrac{\partial \phi}{\partial y} = N(x,y)$
\item Simplify and solve for $h(y)$
\item  Substitute the result for $h(y)$ in the expression for $\phi$ from step 1 and then set $\phi=0$. This is the solution. 
\end{enumerate}

Alternatively: 
\begin{enumerate}
\item Let $\phi=\int N(x,y)dy + g(x)$
\item Set $\dfrac{\partial \phi}{\partial x} = M(x,y)$
\item Simplify and solve for $g(x)$. 
\item Substitute the result for $g(x)$ in the expression for $\phi$ from step 1 and then set $\phi=0$. This is the solution. 
\end{enumerate}

\subsubsection*{Integrating Factors.} Case 1. If $P(x,y)$ depends only on $x$, where
\begin{equation*}
P(x,y)=\dfrac{M_y-N_x}{N} \implies \mu(y) = e^{\int P(x)dx}
\end{equation*}
then
\begin{equation*}
\mu(x) M(x,y) dx + \mu(x) N(x,y) dy = 0
\end{equation*}
is exact.

Case 2. If $Q(x,y)$ depends only on $y$, where
\begin{equation*}
Q(x,y)=\dfrac{N_x-M_y}{M} \implies \mu(y) = e^{\int Q(y)dy}
\end{equation*}
Then 
\begin{equation*}
\mu(y) M(x,y) dx + \mu(y)N(x,y) dy =0
\end{equation*}
is exact.

\subsubsection*{Second Order Linear Equations} General Form of the Equation
\begin{equation} 
a(t)y''+b(t)y'+c(t)y=g(t) \label{eq:general}
\end{equation}
Homogeneous
\begin{equation}
a(t)y''+b(t)y'+c(t)y=0\label{eq:homog}
\end{equation}
Standard Form
\begin{equation}
 y''+p(t)y'+q(t)y=f(t) \label{eq:ODE}
\end{equation}

\subsubsection*{General Solution.} The general solution of \eqref{eq:general} or \eqref{eq:ODE} is 
\begin{equation}
y = C_1 y_1(t) + C_2 y_2 (t) + y_p(t)
\end{equation}
where $y_1(t)$ and $y_2(t)$ are linearly independent solutions of \eqref{eq:homog}.

\subsubsection*{Linear Independence and The Wronskian.} Two functions $f(x)$ and $g(x)$ are linearly dependent if there exist numbers $a$ and $b$, not both zero, such that $af(x)+bg(x)=0$ for all $x$. If $y_1$ and $y_2$ are two solutions of \eqref{eq:homog}, then Wronskian
\begin{equation*}
W(t) = y_1(t) y_2'(t) - y_1'(t) y_2(t)
\end{equation*}
and Abel's Formula
\begin{equation*}
W(t) = Ce^{-\int{p(t)dt}}
\end{equation*}
and the following are all equivalent: 
\begin{enumerate}
\item $\{y_1,y_2\}$ are linearly independent.
\item $\{y_1,y_2\}$ are a fundamental set of solutions.
\item $W(y_1, y_2)(t_0)\neq 0$ at some point $t_0$.
\item $W(y_1,y_2)(t) \neq 0$ for all $t$.
\end{enumerate}


\subsubsection*{Initial Value Problem.} The initial value problem includes two initial conditions at the same point in time, one condition on $y(t)$ and one condition on $y'(t)$. 
\begin{equation*}
\left\{
\begin{array}{l}
y''+p(t)y'+q(t)y=0\\ y(t_0)=y_0 \\ y'(t_0)=y_1
\end{array} 
\right.
\end{equation*}
The initial conditions are applied to the entire solution $y=y_h+y_p$. 

\subsubsection*{Linear Equation With Constant Coefficients.} The general form of the homogeneous equation is  
\begin{equation}
ay'' + by' + cy=0 \label{eq:linearhomog}
\end{equation}
Non-homogeneous
\begin{equation}
ay''+by'+cy = g(t)\label{eq:linearnonhomog}
\end{equation}
Characteristic Equation
\begin{equation*}
ar^2 + br + c=0
\end{equation*}
Quadratic Roots
\begin{equation}
r=\frac{-b\pm\sqrt{b^2-4ac}}{2a}
\end{equation}
The solution of \eqref{eq:linearhomog} of Real Roots $(r_1 \neq r_2)$
\begin{equation}
y_h = C_1 e^{r_1 t} + C_2 e^{r_2t} \label{eq:LH1}
\end{equation}
Repeated $(r_1 = r_2)$
\begin{equation}
y_h = (C_1 + C_2 t)e^{r_1t}\label{eq:LH2}
\end{equation}
Complex $(r=\alpha\pm i\beta)$
\begin{equation}
y_H=e^{\alpha t}(C_1 \cos \beta t + C_2 \sin \beta t) \label{eq:LH3}
\end{equation}
The solution of \eqref{eq:linearnonhomog} is $y=y_p+y_h$ where $y_h$ is given by \eqref{eq:LH1} through \eqref{eq:LH3} and $y_p$ is found by undetermined coefficients or reduction of order.

\subsubsection*{Heuristics for Undetermined Coefficients.} Also called Trial and Error
\begin{center}
\begin{tabular}{|l|l|}
\hline 
If $f(t)=$ & then guess that a particular solution $y_p=$. \\
\hline\hline
$P_n(t)$ & $t^s (A_0 + A_1 t + \cdots + A_n t^n)$ \\
\hline 
$P_n(t)e^{at}$ & $t^s (A_0 + A_1 t + \cdots + A_n t^n)e^{at}$ \\
\hline
$P_n(t)e^{at}\sin bt$ & $t^s e^{at} [(A_0 + A_1 t + \cdots + A_n t^n)\cos bt$ \\
or $P_n(t)e^{at}\cos bt$ & \ \ \ \ \ $ + (A_0 + A_1 t + \cdots + A_n t^n)\sin bt]$ \\
\hline 
\end{tabular}
\end{center}


\subsubsection*{Method of Reduction of Order.} When solving \eqref{eq:homog}, given $y_1$, then $y_2$ can be found by solving
\begin{equation*}
y_1 y_2' - y_1'y_2 = Ce^{-\int p(t) dt}
\end{equation*}
The solution is given by 
\begin{equation}
y_2 = y_1\int \dfrac{e^{-\int p(x) dx} dx}{y_1(x)^2}\label{eq:ROE}
\end{equation}

\subsubsection*{Method of Variation of Parameters.} If $y_1(t)$ and $y_2(t)$ are a fundamental set of solutions to \eqref{eq:homog} then a particular solution to \eqref{eq:ODE} is 
\begin{equation}
y_P (t) = -y_1(t) \int \dfrac{y_2(t) f(t)}{W(t)}dt + y_2(t) \int \dfrac{y_1(t) f(t)}{W(t)}dt 
\end{equation}

\subsubsection*{Cauchy-Euler Equation.} For ODE
\begin{equation}
ax^2y''+bxy'+cy=0 \label{eq:CEODE}
\end{equation}
with auxiliary Equation
\begin{equation}
ar(r-1)+br+c=0\label{eq:CEODEAux}
\end{equation}
The solutions of \eqref{eq:CEODE} depend on the roots $r_{1,2}$ of \eqref{eq:CEODEAux}. For Real Roots
\begin{equation*}
y = C_1x^{r_1} + C_2x^{r_2}
\end{equation*}
Repeated Root
\begin{equation*}
y = C_1 x^r + C_2 x^r \ln x
\end{equation*}
Complex
\begin{equation}
y=x^{\alpha}[C_1\cos(\beta \ln x) + C_2 \sin (\beta \ln x)] \label{eq:comprootCEODE}
\end{equation}
In \eqref{eq:comprootCEODE} $r_{1,2}=\alpha\pm i\beta$, where $\alpha$,$\beta\in\mathds{R}$

\subsubsection*{Series Solutions.} 
\begin{equation}
(x-x_0)^2y''+(x-x_0)p(x)y'+q(x)y=0 \label{eq:SS}
\end{equation}
If $x_0$ is a regular point of \eqref{eq:SS} then 
\begin{equation*}
y_1(t) = (x-x_0)^n\sum_{k=0}^{\infty}a_k(x-x_k)^k
\end{equation*}
At a Regular Singular Point $x_0$, the indicial Equation
\begin{equation}
r^2+(p(0)-1)r + q(0)=0 \label{eq:indicial}
\end{equation}
First Solution
\begin{equation*}
y_1=(x-x_0)^{r_1}\sum_{k=0}^{\infty}a_k(x-x_k)^k
\end{equation*}
Where $r_1$ is the larger real root if both roots of \eqref{eq:indicial} are real or either root if the solutions are complex. 

\end{document}