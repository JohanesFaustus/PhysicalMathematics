\documentclass[../../main.tex]{subfiles}
\begin{document}
\subsection*{Appendix: Frobenius' Method}
I will demonstrate this technique. Consider the following differential equation.
\begin{equation*}
  x^2 y''+ 4xy' + (x^2 + 2)y = 0
\end{equation*}
The solution will take the form 
\begin{equation*}
  y=\sum_{n=0}^{\infty} a_nx^{n+s}
\end{equation*}
Substituting this into each term, we have 
\begin{align*}
  x^2y''&=\sum_{n=0}^{\infty} (n+s) (n+s-1)a_nx^{n+s} \\
  4xy'&=\sum_{n=0}^{\infty} 4(n+s) a_nx^{n+s}\\
  xy&=\sum_{n=0}^{\infty} a_n x^{n+s+2}\\
  2y&=\sum_{n=0}^{\infty} 2a_nx^{n+s}
\end{align*}
Then we put them into table.
\begin{center}
$\begin{array}{c || c c c}
  &x^{n+s}&x^s&x^{s+1}\\
  \hline\hline
  x^2y''&(n + s)(n + s - 1)a_n &s(s-1)a_0  &s(s+1)a_1 \\
  4xy'&4(n+s)a_n &4sa_0 &4(s+1)a_1 \\
  x^2y&a_{n-2} &-&- \\
  2y& 2a_n& 2a_0& 2a_1\\
\end{array}$
\end{center}
Using the terms on $x^{s}$ column, we have the following indicial equation.
\begin{align*}
  s(s-1)a_0+4sa_0+2a_0&=0\\
  a_0 \left[s(s+3)+2 \right] &=0
\end{align*}
Since $a_0$ cannot be zero, we write 
\begin{equation*}
  s^2+3s+2 =0
\end{equation*}
By solving the indicial equation we obtain $s=(-1,-2)$. From the $x^{n+s}$, we obtain the general formula for $a_n$ in terms of $a_{n-2}$
\begin{align*}
  a_n\left[(n+s)(n+s+3)+2\right]&=-a_{n-2}
\end{align*}
We also obtain the fact the value of $a_1$ is zero, proved by the terms in $x^{s+1}$ column 
\begin{equation*}
  \begin{rcases*}
    a_1\left[(s+1)(s+4)+2\right]=0\\
    s=(-1,-2)
  \end{rcases*}\implies a_0=0
\end{equation*}

Since we have two value of $s$, we first consider the case for $s=-1$. The general $a_n$ formula evaluated into 
\begin{equation*}
  a_n=-\frac{a_{n-2}}{(n-1)(n+2)+2}=-\frac{a_{n-2}}{n^2+n}=-\frac{a_{n-2}}{n(n+1)}
\end{equation*}
The values of $a_n$ for few $n$ are as follows 
\begin{align*}
  a_2&=-\frac{a_0}{3!}\\
  a_4&=-\frac{a_2}{4\cdot 5}=\frac{a_0}{5!}\\
  a_6&=-\frac{a_4}{6\cdot7}=-\frac{a_0}{7!}
\end{align*}
Thus the solution for this case is 
\begin{multline*}
  y_{-1}=\sum_{n=0}^{\infty} a_nx^{n-1}=\frac{a_0}{x}-\frac{a_0}{3!}x+\frac{a_0}{5!}x^3-\frac{a_0}{7!}x^5 +\dots\\
  =\frac{a_0}{x^2}\left(x-\frac{x^3}{3!}+\frac{x^5}{5!}-\frac{x^7}{7!}+\dots\right)=\frac{a_0}{x^2}\sin x
\end{multline*}

For the case of $s=-2$, the general $a_n$ formula evaluated into
\begin{equation*}
  a_n=-\frac{a_{n-2}}{(n-2)(n+1)+2}=-\frac{a_{n-2}}{n^2-n}=-\frac{a_{n-2}}{n(n-1)}
\end{equation*}
The values of $a_n$ for few $n$ are as follows 
\begin{align*}
  a_2&=-\frac{a_0}{2!}\\
  a_4&=-\frac{a_2}{4\cdot 3}=\frac{a_0}{4!}\\
  a_6&=-\frac{a_4}{6\cdot 5}=-\frac{a_0}{6!}
\end{align*}
Thus the solution for this case is 
\begin{multline*}
  y_{-2}=\sum_{n=0}^{\infty} a_nx^{n-2}= \frac{a_0}{x^2}-\frac{a_0}{2!}+\frac{a_0}{4!}x^2-\frac{a_0}{6!}x^4 +\dots\\
  =\frac{a_0}{x^2}\left(1-\frac{x^2}{2!}+\frac{x^4}{4!}-\frac{x^6}{6!}+\dots\right)=\frac{a_0}{x^2}\cos x
\end{multline*}

Hence, the complete form of the solution is 
\begin{equation*}
  y=\frac{a_0}{x^2}\left(\cos x +\sin x\right)
\end{equation*}

\subsection*{Bessel Equation}
\subsubsection*{Ex. 1.} Suppose we are going to solve 
\begin{equation*}
  y'' +9xy=0
\end{equation*}
We know that the equation has no $y'$ factor, then 
\begin{equation*}
  \frac{1-2a}{x}=0\implies a=\frac{1}{2}
\end{equation*}
By assuming
\begin{equation*}
  2c-2=1\implies c=\frac{3}{2}
\end{equation*}
We can equate the first $x$ coefficient
\begin{equation*}
  (bc)^2=9\implies b=2
\end{equation*}
And
\begin{equation*}
  \frac{a^2-p^2c^2}{x^2}=0\implies p=\sqrt{\frac{a^2}{c^2}}=\frac{1}{3}
\end{equation*}
The solution takes the form of 
\begin{equation*}
  y=x^{1/2}Z_{1/3}(2x^{3/2})=x^{1/2}\left[AJ_{1/3}(2x^{3/2})+BN_{1/3}(2x^{3/2})\right]
\end{equation*}
\end{document}