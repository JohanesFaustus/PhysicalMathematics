\documentclass[../../main.tex]{subfiles}
\begin{document}
\subsection*{Appendix: Using Gamma Function to Evaluate Integral}
\subsubsection*{Ex. 1.} Consider
\begin{equation*}
    \int_{0}^{3}x^3e^{-4x^2}=\frac{\gamma(4,32)}{2\cdot4^2}
\end{equation*}

\subsubsection*{Ex. 2.} Consider
\begin{equation*}
    \int_{1}^{2}x^3e^{-x}\;dx=\gamma(4,2)-\gamma(4,1)
\end{equation*}

\subsubsection*{Ex. 3.} The distribution function for classical partial with respect to its energy is given by 
\begin{equation*}
    n(\epsilon)=\frac{2\pi N}{(\pi kT)^{3/2}}\epsilon^{1/2}\exp \left(-\frac{\epsilon}{kT}\right)
\end{equation*}
Suppose we want to determine the fraction of particle with energy more than $\epsilon \geq 3kT$. Said fraction is determined first find the number of particle within the range of energy
\begin{equation*}
    n(\epsilon \geq 3kT)=\int_{0}^{3kT}\frac{2\pi N}{(\pi kT)^{3/2}}\epsilon^{1/2}\exp \left(-\frac{\epsilon}{kT}\right)\;d\epsilon
\end{equation*}
By substituting $\epsilon/kT=x$, we have $\epsilon=kTx$ and $d\epsilon=kT\;dx$
\begin{equation*}
    n(\epsilon \geq 3kT)=\frac{2N}{\sqrt{\pi}}\int_{0}^{3}x^{1/2}\exp \left(-x\right)\;dx=\frac{2N}{\sqrt{\pi}}=\frac{2N}{\sqrt{\pi}}\gamma\left(\frac{3}{2},3\right)
\end{equation*}

Then the fraction
\begin{equation*}
    F=\frac{n(\epsilon \geq 3kT)}{N}=\frac{2}{\sqrt{\pi}}\gamma\left(\frac{3}{2},3\right)
\end{equation*}

\subsubsection*{Ex. 4.} Now suppose we want to find the fraction of particle whose speed lies within $(\sqrt{2kT/m}, \sqrt{8kT/m})$ where its distribution function is given by 
\begin{equation*}
    n(v)=4\pi N\left(\frac{m}{2\pi kT}\right)^{3/2}v^2\exp\left(-\frac{mv^2}{2kT}\right)
\end{equation*}
The number of particle within said range is 
\begin{multline*}
    n\left(\sqrt{\frac{2kT}{m}}\leq v\leq \sqrt{\frac{8kT}{m}}\right)=4\pi N\left(\frac{m}{2\pi kT}\right)^{3/2}\\
    \int_{\sqrt{2kT/m}}^{\sqrt{8kT/m}}v^2\exp\left(-\frac{mv^2}{2kT}\right)\;dv
\end{multline*}
By substituting $mv^2/2kT=x^2$, we have 
\begin{equation*}
    v^2=\frac{2kT}{m}x^2\quad\text{and}\quad dv=\sqrt{\frac{2kT}{m}}\;dx
\end{equation*}
and limit of 
\begin{align*}
    v=\sqrt{\frac{2kT}{m}}&\implies x=1\\
    v=\sqrt{\frac{8kT}{m}}&\implies x=2
\end{align*}
Then 
\begin{align*}
    n=\frac{4N}{\sqrt{\pi}}\frac{m}{2kT}\int_{1}^{2}\frac{2kT}{m}x^2 e^{-x^2} \sqrt{\frac{ 2kT}{m }}\;dx=\frac{4N}{\sqrt{\pi}}\left[\gamma(1.5,1)-\gamma(1.5,2)\right]
\end{align*}

\subsection*{Appendix: Using Beta Function to Evaluate Integral}
\subsubsection*{Ex. 1.} Consider
\begin{equation*}
    I=\int_{0}^{\pi/2}\cos^n\theta\;d\theta
\end{equation*}
In terms of beta function
\begin{equation*}
    I=\frac{1}{2}B\left(\frac{n+1}{2},\frac{1}{2}\right)=\frac{1}{2}\frac{\Gamma\left(\frac{n+1}{2}\right)\Gamma\left(\frac{1}{2}\right)}{\Gamma\left(\frac{n}{2}+1\right)}
\end{equation*}

For the case of odd $n$, we write 
\begin{equation*}
    \Gamma\left(\frac{n+1}{2}\right)=\frac{n-1}{2}\frac{n-3}{2}\dots1=\frac{(n-1)!!}{2^{(n-1)/2}}
\end{equation*}
where its argument is an even number. We also write 
\begin{equation*}
    \Gamma\left(\frac{n}{2}+1\right)= \frac{n}{2}\left(\frac{n-2}{2}\right)\dots\sqrt{\pi}=\frac{n!!\sqrt{\pi}}{2^{(n+1)/2}}
\end{equation*}
where its argument is an odd number. Hence, 
\begin{equation*}
    I=\frac{(n-1)!!}{n!!}
\end{equation*}

For the case of even $n$, we then write 
\begin{equation*}
    \Gamma\left(\frac{n+1}{2}\right)=\frac{n-1}{2}\frac{n-3}{2}\dots\sqrt{\pi}=\frac{\sqrt{\pi}(n-1)!!}{2^{(n+1)/2}}
\end{equation*}
where its argument is an odd number. We also write 
\begin{equation*}
    \Gamma\left(\frac{n}{2}+1\right)= \frac{n}{2}\left(\frac{n-2}{2}\right)\dots1=\frac{n!!}{2^{(n-1)/2}}
\end{equation*}
where its argument is an even number. Hence, 
\begin{equation*}
    I=\frac{\pi}{2}\frac{(n-1)!!}{n!!}
\end{equation*}

Putting it all together
\begin{equation*}
    \int_{0}^{\pi/2}\cos^n\theta\;d\theta=\begin{cases}
        \dfrac{(n-1)!!}{n!!},&\text{for odd $n$}\\\\
        \dfrac{\pi}{2}\dfrac{(n-1)!!}{n!!},&\text{for even $n$}
    \end{cases}
\end{equation*}

\subsubsection*{Ex. 2} Consider 
\begin{equation*}
    I=\int_{-1}^{1}x^{2n}(1-x^2)^{1/2}\;dx
\end{equation*}
Since the integrand is an even function, we write 
\begin{equation*}
    I=2\int_{0}^{1}x^{2n}(1-x^2)^{1/2}\;dx
\end{equation*}
Substituting $t=x^2$ and $dt=2\sqrt{t}\;dx$ and writing the resulting integral in terms of beta function
\begin{equation*}
    I=\int_{0}^{1}t^{(2n-1)/2}(1-t)^{1/2}\;dt=B\left(n+\frac{1}{2}, \frac{3}{2}\right)
\end{equation*}
In terms of gamma function
\begin{equation*}
    I=\frac{\Gamma\left(\frac{3}{2}\right)\Gamma\left(n+\frac{1}{2}\right)}{\Gamma(n+2)}
\end{equation*}
For any even and odd $n$, we write 
\begin{equation*}
    \Gamma\left(n+\frac{1}{2}\right)=\left(\frac{2n-1}{2}\right)\left(\frac{2n-3}{2}\right) \dots \sqrt{\pi}=\frac{(2n-1)!!}{2^n}\sqrt{\pi}
\end{equation*}
and 
\begin{equation*}
    \Gamma(n+2)=\frac{(2n+2)!!}{2^{n+1}}
\end{equation*}
Therefore
\begin{equation*}
    I= \frac{(2n-1)!!}{(2n+1)!!}\pi
\end{equation*}
Since the result differ for different $n$, we then evaluate for both case and write 
\begin{equation*}
    \int_{-1}^{1}x^{2n}(1-x^2)^{1/2}\;dx=\begin{cases}
        \dfrac{\pi}{2},&\text{for $n=0$}\\\\
        \dfrac{(2n-1)!!}{(2n+1)!!}\pi,&\text{for interger $n>0$}
    \end{cases}
\end{equation*}
\end{document}