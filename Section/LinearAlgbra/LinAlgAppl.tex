\documentclass[../main.tex]{subfiles}
\begin{document}
\subsection*{Application: Gram-Schmidt Theorem}
Convert these linearly independent basis into orthonormal basis
\begin{equation*}
	\ket{I}=\begin{bmatrix}
		3 \\0\\0
	\end{bmatrix}\quad
	\ket{II}=\begin{bmatrix}
		0 \\1\\2
	\end{bmatrix}
	\ket{III}=\begin{bmatrix}
		0 \\2\\5
	\end{bmatrix}
\end{equation*}

First normalize
\begin{equation*}
	\ket{1}=\frac{\ket{I}}{\sqrt{\braket{I|I}}}=\frac{1}{3}\begin{bmatrix}
		3 \\0\\0
	\end{bmatrix}=\begin{bmatrix}
		1 \\0\\0
	\end{bmatrix}
\end{equation*}
Then we construct
\begin{equation*}
	\ket{2'}=\ket{II}-\ket{1}\braket{1|II}=\begin{bmatrix}
		0 \\1\\2
	\end{bmatrix}
	-
	\begin{bmatrix}
		1 \\0\\0
	\end{bmatrix}
	\begin{bmatrix}
		1 & 0 & 0
	\end{bmatrix}
	\begin{bmatrix}
		0 \\2\\5
	\end{bmatrix}
	=
	\begin{bmatrix}
		0 \\1\\2
	\end{bmatrix}
\end{equation*}
and normalize
\begin{equation*}
    \ket{2}=\frac{\ket{2'}}{|2'|}=\frac{1}{\sqrt{5}}\begin{bmatrix}
        0\\1\\2
    \end{bmatrix}=
    \begin{bmatrix}
        0\\1/\sqrt{5}\\2/\sqrt{5}
    \end{bmatrix}
\end{equation*}
Doing the something for the third base
\begin{align*}
    \ket{3'}&=\ket{III}-\ket{1}\braket{1|III}-\ket{2}\braket{2|III}\\
    &=\begin{bmatrix}
        0\\2\\5
    \end{bmatrix}-\begin{bmatrix}
        1\\0\\0
    \end{bmatrix}
    \begin{bmatrix}
        1&0&0
    \end{bmatrix}
    \begin{bmatrix}
        0\\2\\5
    \end{bmatrix}
    -
    \begin{bmatrix}
        0\\1/\sqrt{5}\\2/\sqrt{5}
    \end{bmatrix}
    \begin{bmatrix}
        0&1/\sqrt{5}&2/\sqrt{5}
    \end{bmatrix}
    \begin{bmatrix}
        0\\2\\5
    \end{bmatrix}\\
    &=\begin{bmatrix}
        0\\2\\5
    \end{bmatrix}
    -
    \begin{bmatrix}
        0\\12/5\\24/5
    \end{bmatrix}
    =
    \begin{bmatrix}
        0\\-2/5\\1/5
    \end{bmatrix}
\end{align*}
And normalize it 
\begin{equation*}
    \ket{3}=\frac{\ket{3'}}{|3'|}=\sqrt{5}\begin{bmatrix}
        0\\-2/5\\1/5
    \end{bmatrix}
    =
    \begin{bmatrix}
        0\\-2/\sqrt{5}\\1/\sqrt{5}
    \end{bmatrix}
\end{equation*}

\subsection*{Application: Determining determinant}
Consider the matrix
\begin{equation*}
    A=\begin{bmatrix}
        2&-5&2\\
        7&3&4\\
        2&1&5\\
    \end{bmatrix}
\end{equation*}
We use the elements of third column first
\begin{align*}
    \det A&=\begin{vmatrix}
        2&-5&2\\
        7&3&4\\
        2&1&5\\
    \end{vmatrix}
    =
    2\begin{vmatrix}
        7&3\\
        2&1\\
    \end{vmatrix}
    -
    4\begin{vmatrix}
        2&-5\\
        2&1\\
    \end{vmatrix}
    +
    5\begin{vmatrix}
        2&-5\\
        7&3\\
    \end{vmatrix}
    \\
    &=2\cdot1-4\cdot11+5\cdot38=148
\end{align*}
Then, as a check we use the first row's 
\begin{align*}
    \det A    =
    1\begin{vmatrix}
        3&4\\
        1&5\\
    \end{vmatrix}
    +
    5\begin{vmatrix}
        7&4\\
        2&5\\
    \end{vmatrix}
    +
    2\begin{vmatrix}
        7&3\\
        2&1\\
    \end{vmatrix}
    \\
    &=11+135+2=
\end{align*}

\subsection*{Application: Inverse matrix}
We use inverse matrix to solve the following equation
\begin{equation*}
    \begin{bmatrix}
        1&0&-1\\
        -2&3&0\\
        1&-3&2\\
    \end{bmatrix}
    \begin{bmatrix}
        x\\
        y\\
        z\\
    \end{bmatrix}
    =
    \begin{bmatrix}
        5\\
        1\\
        -10\\
    \end{bmatrix}
\end{equation*}
Notice the equation has the following form
\begin{align*}
    \Omega \ket{V}&=\omega\\
    \ket{V}=\Omega^{-1}\omega
\end{align*}
To find vector $\ket{V}$ that satisfy the equation, we need to determine the inverse of $\Omega$.
The minor of each element are
\begin{align*}
	M_{11} & =6 , \quad
	M_{12} = -4, \quad
	M_{13} = 3,         \\
	M_{21} & -3= , \quad
	M_{22} = 3, \quad
	M_{23} = -3,         \\
	M_{31} & = 3, \quad
	M_{32} = -2, \quad
	M_{33} =3\\ 
\end{align*}
Thus, the cofactor is 
\begin{equation*}
    \text{C}=\begin{bmatrix}
        6&4&3\\
        3&3&3\\
        3&2&3\\
    \end{bmatrix}
\end{equation*}
Next, using the first row to find the determinant
\begin{equation*}
    \det\Omega=1\cdot6+1\cdot(6-3)=3
\end{equation*}
Finally the inverse is 
\begin{equation*}
    \Omega^{-1}=\frac{1}{\det\Omega}\text{C}^T=\frac{1}{3}
    \begin{bmatrix}
        6&3&3\\
        4&3&2\\
        3&3&3\\
    \end{bmatrix}
\end{equation*}
Now we can use the inverse to find the value of the vector
\begin{equation*}
    \ket{V}=\frac{1}{3}
    \begin{bmatrix}
        6&3&3\\
        4&3&2\\
        3&3&3\\
    \end{bmatrix}
    \begin{bmatrix}
        5\\
        1\\
        -10
    \end{bmatrix}
    =\begin{bmatrix}
        10+1-10\\
        \frac{20+3-20}{3}\\
        5+1-10\\
    \end{bmatrix}
    =
    \begin{bmatrix}
        1\\
        1\\
        -4
    \end{bmatrix}
\end{equation*}
\end{document}