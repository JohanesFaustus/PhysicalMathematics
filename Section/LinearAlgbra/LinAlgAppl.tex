\documentclass[../main.tex]{subfiles}
\begin{document}
\subsection*{Application: Gram-Schmidt Theorem}
Convert these linearly independent basis into orthonormal basis
\begin{equation*}
	\ket{I}=\begin{bmatrix}
		3 \\0\\0
	\end{bmatrix}\quad
	\ket{II}=\begin{bmatrix}
		0 \\1\\2
	\end{bmatrix}
	\ket{III}=\begin{bmatrix}
		0 \\2\\5
	\end{bmatrix}
\end{equation*}

First normalize
\begin{equation*}
	\ket{1}=\frac{\ket{I}}{\sqrt{\braket{I|I}}}=\frac{1}{3}\begin{bmatrix}
		3 \\0\\0
	\end{bmatrix}=\begin{bmatrix}
		1 \\0\\0
	\end{bmatrix}
\end{equation*}
Then we construct
\begin{equation*}
	\ket{2'}=\ket{II}-\ket{1}\braket{1|II}=\begin{bmatrix}
		0 \\1\\2
	\end{bmatrix}
	-
	\begin{bmatrix}
		1 \\0\\0
	\end{bmatrix}
	\begin{bmatrix}
		1 & 0 & 0
	\end{bmatrix}
	\begin{bmatrix}
		0 \\2\\5
	\end{bmatrix}
	=
	\begin{bmatrix}
		0 \\1\\2
	\end{bmatrix}
\end{equation*}
and normalize
\begin{equation*}
    \ket{2}=\frac{\ket{2'}}{|2'|}=\frac{1}{\sqrt{5}}\begin{bmatrix}
        0\\1\\2
    \end{bmatrix}=
    \begin{bmatrix}
        0\\1/\sqrt{5}\\2/\sqrt{5}
    \end{bmatrix}
\end{equation*}
Doing the something for the third base
\begin{align*}
    \ket{3'}&=\ket{III}-\ket{1}\braket{1|III}-\ket{2}\braket{2|III}\\
    &=\begin{bmatrix}
        0\\2\\5
    \end{bmatrix}-\begin{bmatrix}
        1\\0\\0
    \end{bmatrix}
    \begin{bmatrix}
        1&0&0
    \end{bmatrix}
    \begin{bmatrix}
        0\\2\\5
    \end{bmatrix}
    -
    \begin{bmatrix}
        0\\1/\sqrt{5}\\2/\sqrt{5}
    \end{bmatrix}
    \begin{bmatrix}
        0&1/\sqrt{5}&2/\sqrt{5}
    \end{bmatrix}
    \begin{bmatrix}
        0\\2\\5
    \end{bmatrix}\\
    &=\begin{bmatrix}
        0\\2\\5
    \end{bmatrix}
    -
    \begin{bmatrix}
        0\\12/5\\24/5
    \end{bmatrix}
    =
    \begin{bmatrix}
        0\\-2/5\\1/5
    \end{bmatrix}
\end{align*}
And normalize it 
\begin{equation*}
    \ket{3}=\frac{\ket{3'}}{|3'|}=\sqrt{5}\begin{bmatrix}
        0\\-2/5\\1/5
    \end{bmatrix}
    =
    \begin{bmatrix}
        0\\-2/\sqrt{5}\\1/\sqrt{5}
    \end{bmatrix}
\end{equation*}

\subsection*{Application: Determining determinant}
Consider the matrix
\begin{equation*}
    A=\begin{bmatrix}
        2&-5&2\\
        7&3&4\\
        2&1&5\\
    \end{bmatrix}
\end{equation*}
We use the elements of third column first
\begin{align*}
    \det A&=\begin{vmatrix}
        2&-5&2\\
        7&3&4\\
        2&1&5\\
    \end{vmatrix}
    =
    2\begin{vmatrix}
        7&3\\
        2&1\\
    \end{vmatrix}
    -
    4\begin{vmatrix}
        2&-5\\
        2&1\\
    \end{vmatrix}
    +
    5\begin{vmatrix}
        2&-5\\
        7&3\\
    \end{vmatrix}
    \\
    &=2\cdot1-4\cdot11+5\cdot38=148
\end{align*}
Then, as a check we use the first row's 
\begin{align*}
    \det A    =
    1\begin{vmatrix}
        3&4\\
        1&5\\
    \end{vmatrix}
    +
    5\begin{vmatrix}
        7&4\\
        2&5\\
    \end{vmatrix}
    +
    2\begin{vmatrix}
        7&3\\
        2&1\\
    \end{vmatrix}
    \\
    &=11+135+2=
\end{align*}

\subsection*{Application: Inverse matrix}
We use inverse matrix to solve the following equation
\begin{equation*}
    \begin{bmatrix}
        1&0&-1\\
        -2&3&0\\
        1&-3&2\\
    \end{bmatrix}
    \begin{bmatrix}
        x\\
        y\\
        z\\
    \end{bmatrix}
    =
    \begin{bmatrix}
        5\\
        1\\
        -10\\
    \end{bmatrix}
\end{equation*}
Notice the equation has the following form
\begin{align*}
    \Omega \ket{V}&=\omega\\
    \ket{V}=\Omega^{-1}\omega
\end{align*}
To find vector $\ket{V}$ that satisfy the equation, we need to determine the inverse of $\Omega$.
The minor of each element are
\begin{align*}
	M_{11} & =6 , \quad
	M_{12} = -4, \quad
	M_{13} = 3,         \\
	M_{21} & -3= , \quad
	M_{22} = 3, \quad
	M_{23} = -3,         \\
	M_{31} & = 3, \quad
	M_{32} = -2, \quad
	M_{33} =3\\ 
\end{align*}
Thus, the cofactor is 
\begin{equation*}
    \text{C}=\begin{bmatrix}
        6&4&3\\
        3&3&3\\
        3&2&3\\
    \end{bmatrix}
\end{equation*}
Next, using the first row to find the determinant
\begin{equation*}
    \det\Omega=1\cdot6+1\cdot(6-3)=3
\end{equation*}
Finally the inverse is 
\begin{equation*}
    \Omega^{-1}=\frac{1}{\det\Omega}\text{C}^T=\frac{1}{3}
    \begin{bmatrix}
        6&3&3\\
        4&3&2\\
        3&3&3\\
    \end{bmatrix}
\end{equation*}
Now we can use the inverse to find the value of the vector
\begin{equation*}
    \ket{V}=\frac{1}{3}
    \begin{bmatrix}
        6&3&3\\
        4&3&2\\
        3&3&3\\
    \end{bmatrix}
    \begin{bmatrix}
        5\\
        1\\
        -10
    \end{bmatrix}
    =\begin{bmatrix}
        10+1-10\\
        \frac{20+3-20}{3}\\
        5+1-10\\
    \end{bmatrix}
    =
    \begin{bmatrix}
        1\\
        1\\
        -4
    \end{bmatrix}
\end{equation*}

\subsection*{Application: Eigenvalue Problem}
We shall find the eigenvalue of the hermitian matrix
\begin{equation*}
    \Omega=\begin{bmatrix}
        0&0&1\\
        0&0&0\\
        1&0&0\\
    \end{bmatrix}
\end{equation*}
First write the eigenvalue equation
\begin{equation*}
    (\Omega-\omega I)\ket{\omega}=
    \begin{bmatrix}
        -\omega&0&1\\
        0&-\omega&0\\
        1&0&-\omega\\
    \end{bmatrix}
    \ket{\omega}=0
\end{equation*}
The characteristic equation is
\begin{equation*}
    -\omega^3+\omega=\omega(\omega^2+1)=0
\end{equation*}
This implies the eigenvalues are
\begin{equation*}
    \omega=0,\pm 1
\end{equation*}
Next, we substitute the eigenvalue into the eigenvalue equation.
First consider the eigenvalue $\omega=0$
\begin{equation*}
    (\Omega-\omega I)\ket{\omega_0}=0\implies
    \begin{bmatrix}
        0&0&1\\
        0&0&0\\
        1&0&0\\
    \end{bmatrix}
    \begin{bmatrix}
        v_1\\
        v_2\\
        v_3\\
    \end{bmatrix}
    =
    \begin{bmatrix}
        v_3=0\\
        0=0\\
        v_1=0
    \end{bmatrix}
\end{equation*}
And we get arbitrary $v_2$, to normalize the eigenvector we choose
\begin{equation*}
    \bra{\omega_0}=
    \begin{bmatrix}
        0&1&0\\
    \end{bmatrix}
\end{equation*}
Next is the case of $\omega=1$
\begin{equation*}
    (\Omega-\omega I)\ket{\omega_1}=0\implies
    \begin{bmatrix}
        -1&0&1\\
        0&-1&0\\
        1&0&-1\\
    \end{bmatrix}
    \begin{bmatrix}
        v_1\\
        v_2\\
        v_3\\
    \end{bmatrix}
    =
    \begin{bmatrix}
        -v_1+v_3=0\\
        -v_2=0\\
        v_1-v_3=0
    \end{bmatrix}
\end{equation*}
the eigenvector corresponds to the eigenvalue is 
\begin{equation*}
    \ket{\omega_1}=\frac{1}{\sqrt{2}}
    \begin{bmatrix}
        1&0&1
    \end{bmatrix}
\end{equation*}
Finally the last eigenvalue $\omega=-1$
\begin{equation*}
    (\Omega-\omega I)\ket{\omega_{-1}}=0\implies
    \begin{bmatrix}
        1&0&1\\
        0&1&0\\
        1&0&1\\
    \end{bmatrix}
    \begin{bmatrix}
        v_1\\
        v_2\\
        v_3\\
    \end{bmatrix}
    =
    \begin{bmatrix}
        v_1+v_3=0\\
        v_2=0\\
        v_1+v_3=0
    \end{bmatrix}
\end{equation*}
with the eigenvector of 
\begin{equation*}
    \bra{\omega_{-1}}=\frac{1}{\sqrt{2}}
    \begin{bmatrix}
        1&0&-1
    \end{bmatrix}
\end{equation*}

We can also use these eigenvectors to diagonal the operator $\Omega$.
From the eigenvector, we construct the unitary matrix 
\begin{equation*}
    U=
    \begin{bmatrix}
        0&\frac{1}{\sqrt{2}}&\frac{1}{\sqrt{2}}\\
        1&0&0\\
        0&\frac{1}{\sqrt{2}}&-\frac{1}{\sqrt{2}}\\
    \end{bmatrix}
\end{equation*}
As a check, we can also confirm the unitary identity of unitary matrix
\begin{equation*}
    U ^\dagger U=
    \begin{bmatrix}
        0&1&0\\
        \frac{1}{\sqrt{2}}&0&\frac{1}{\sqrt{2}}\\\
        \frac{1}{\sqrt{2}}&0&-\frac{1}{\sqrt{2}}\\
    \end{bmatrix}    \begin{bmatrix}
        0&\frac{1}{\sqrt{2}}&\frac{1}{\sqrt{2}}\\
        1&0&0\\
        0&\frac{1}{\sqrt{2}}&-\frac{1}{\sqrt{2}}\\
    \end{bmatrix}
    =
    \begin{bmatrix}
        1&0&0\\
        0&1&0\\
        0&0&1\\
    \end{bmatrix}
\end{equation*} 
As expected.
We can now jump to the diagonalization
\begin{align*}
    U ^\dagger \Omega U&=
    \begin{bmatrix}
        0&1&0\\
        \frac{1}{\sqrt{2}}&0&\frac{1}{\sqrt{2}}\\\
        \frac{1}{\sqrt{2}}&0&-\frac{1}{\sqrt{2}}\\
    \end{bmatrix}    
    \begin{bmatrix}
        0&0&1\\
        0&0&0\\
        1&0&0\\
    \end{bmatrix} 
    \begin{bmatrix}
        0&\frac{1}{\sqrt{2}}&\frac{1}{\sqrt{2}}\\
        1&0&0\\
        0&\frac{1}{\sqrt{2}}&-\frac{1}{\sqrt{2}}\\
    \end{bmatrix}\\
    &=
    \begin{bmatrix}
        0&1&0\\
        \frac{1}{\sqrt{2}}&0&\frac{1}{\sqrt{2}}\\\
        \frac{1}{\sqrt{2}}&0&-\frac{1}{\sqrt{2}}\\
    \end{bmatrix}    
    \begin{bmatrix}
        0&\frac{1}{\sqrt{2}}&-\frac{1}{\sqrt{2}}\\
        1&0&0\\
        0&\frac{1}{\sqrt{2}}&\frac{1}{\sqrt{2}}\\
    \end{bmatrix}\\
    U ^\dagger \Omega U&=
    \begin{bmatrix}
        0&0&0\\
        0&1&0\\
        0&0&-1\\
    \end{bmatrix}
\end{align*}
Which is just the eigenvalue as the diagonal element.

\subsection*{Application: Eigenvalue Problem With Degeneracy}
Now consider operator with matrix element
\begin{equation*}
    \Omega=
    \begin{bmatrix}
        1&0&1\\
        0&2&0\\
        1&0&1\\
    \end{bmatrix}
\end{equation*}
The eigenvalue equation is 
\begin{equation*}
    (\Omega-\omega I)\ket{\omega}=0\implies
    \begin{bmatrix}
        1-\omega&0&1\\
        0&2-\omega&0\\
        1&0&1-\omega\\
    \end{bmatrix}
    \ket{\omega}=0
\end{equation*}
while the characteristic equation
\begin{equation*}
    (2-\omega)[(1-\omega)^2-1]=(2-\omega)[\omega^2-2\omega]=(2-\omega)\omega(\omega-2)=0
\end{equation*}
which result in eigenvalues of 
\begin{equation*}
    \omega=0,2,2
\end{equation*}
with $\omega=2$ as degenerate.
First we consider the eigenvalue $\omega=0$
\begin{equation*}
    (\Omega-\omega I)\ket{\omega_0}=0\implies
    \begin{bmatrix}
        1&0&1\\
        0&2&0\\
        1&0&1\\
    \end{bmatrix}
    \begin{bmatrix}
        v_1\\
        v_2\\
        v_3\\
    \end{bmatrix}
    =
    \begin{bmatrix}
        v_1+v_2=0\\
        2v_2=0\\
        v_1+v_3=0\\
    \end{bmatrix}
\end{equation*}
with eigenvector of
\begin{equation*}
    \bra{\omega_0}=\frac{1}{\sqrt{2}}
    \begin{bmatrix}
        1&0&-1
    \end{bmatrix}
\end{equation*}
Next is the degenerate eigenvalue of $\omega=2$
\begin{equation*}
    (\Omega-\omega I)\ket{\omega_2}=0\implies
    \begin{bmatrix}
        -1&0&1\\
        0&0&0\\
        1&0&-1\\
    \end{bmatrix}
    \begin{bmatrix}
        v_1\\
        v_2\\
        v_3\\
    \end{bmatrix}
    =
    \begin{bmatrix}
        -v_1+v_2=0\\
        0=0\\
        -v_1+v_3=0\\
    \end{bmatrix}
\end{equation*}
The eigenvector, which are also degenerate, has the arbitrary component $v_2$.
This mean that the two degenerate eigenvector lies on the same plane.
For the first degenerate eigenvector, let us just choose the simplest normalized vector that also satisfies the equation above
\begin{equation*}
    \ket{\omega_2,\alpha}=\frac{1}{\sqrt{3}}
    \begin{bmatrix}
        1\\
        1\\
        1\\
    \end{bmatrix}
\end{equation*}
For the second degenerate eigenvector, we choose in a way such that it is orthonormal with the first.
The orthogonal condition state that 
\begin{equation*}
    \braket{\omega_2,\alpha|^\circ_2, \beta}=0
\end{equation*}
Since the degenerate vectors lies in the same plane on arbitrary $v_2$, we shall determine the value of $v_2$ such that $\ket{\omega_2,\alpha}$ is orthogonal with $\ket{\omega_2,\beta}$
\begin{equation*}
    \braket{\omega_2,\alpha|\omega_2', \beta}= 
    \begin{bmatrix}
        1\\1\\1
    \end{bmatrix}
    \begin{bmatrix}
        1\\
        \beta\\
        1\\
    \end{bmatrix}
    =\beta+2
\end{equation*}
or $\beta=-2$.
All that left is normalizing it 
\begin{equation*}
    \ket{\omega_2,\beta}=\frac{1}{\sqrt{6} }
    \begin{bmatrix}
        1\\
        -2\\
        1
    \end{bmatrix}
\end{equation*}
\end{document}