\documentclass[../../main.tex]{subfiles}
\begin{document}
\subsection{Permutation and Combination}
\subsubsection{Permutation.} Consider finite set $A$ with $n$ elements. An $r$-permu-tation is an ordered selection of $r$ elements from $A$, with $1\leq r\leq n$. In permutation, order does matter, unlike combination, and that all arrangements are distinct. $r$-permutation of an $n$ elements set is defined as 
\begin{equation*}
    P(n,r)=n(n-1)\dots(n-r+1)
\end{equation*}
or simply
\begin{equation*}
    P(n,r)=\frac{n!}{(n-r)!}
\end{equation*}

\subsubsection{Combination.} Combination counts the number of ways to chose $r$ object form finite set $A$ with $n$ elements where order of selection does not matter. For all integer $n$ and $1\leq r\leq n$, the number of combination when $r$ elements are chosen out of finite set with $n$ elements $C(n,r)$ is 
\begin{equation*}
    C(n,r)=\frac{P(n,r)}{r!}={n \choose r }
\end{equation*}
or 
\begin{equation*}
    C(n,r)=\frac{n!}{r!(n-r)!}
\end{equation*}

\subsubsection{Difference.} Suppose we are choosing 2 people out of 4 to be president and vice-president. Here order matter, thus we say that there are 
\begin{equation*}
    P(4,2)=\frac{4!}{(4-2)!}=12
\end{equation*}
ways to choose 2 people out of 4 to be president and vice-president. Now, we change the situation into choosing 2 out of 4 people to be given a gift. Here, order does not matter, hence we say that there are 
\begin{equation*}
    C(4,2)=\frac{4!}{2!(4-2)!}=6
\end{equation*}
ways to choose 2 people out of 4 to be given gift.

\subsection{Restricted Partition Generating Functions}
\subsubsection{Definition.} To find the number of ways distributing, called configuration, $L$ identical object in $N$ distinct boxes subject to condition that not more that $P$ object are in one box, we use 
\begin{equation*}
    D(N,P,L)=\frac{1}{L!}\frac{d^L}{dx^L}f(x)\bigg|_{x=0}
\end{equation*}
where
\begin{equation*}
    f(x)=\left(1+x+x^2+\cdots+ x^P\right)^N=\left(\sum^{P }_{i=0}x^i\right)^N
\end{equation*}

In other hands, the number of configuration of certain set $n_{k|P}$ is given by 
\begin{equation*}
    D(N,P,n_{k|P})=\frac{N!}{\displaystyle\prod_{i=0}^{P}n_i!}
\end{equation*}
while the total number of configuration form all possible set is 
\begin{equation*}
    D_T(N,P)=(P+1)^N
\end{equation*}

For a special case when $L\leq N$, the expression for $D(N,P,L)$ simplify into 
\begin{equation*}
    D(N,P,L)=\frac{1}{L!}\frac{(N+L-1)!}{(N-1)!}
\end{equation*}
This equation gives the number of ways to distribute $L$ indistinguishable objects in $P$ distinguishable box.

\subsubsection{Derivation.} Let the boxes be numbered $1,\dots, N$ and $p_i$ as number of objects in $i$-th box, then 
\begin{equation*}
    \sum_{i=1}^{N}p_i=L
\end{equation*}
with $0\leq p_i\leq P$. Set obtained form interchanging $p_i$ and $p_k$, with $p_i\neq p_k$ is counted as different set, however the same exchange with $p_i=p_k$, does not count as different set. Let also $n_k$ as the number of boxed having $k$ number object, hence we have these two restricted for our combination
\begin{equation*}
    \sum_{k=0}^{P}n_k=N,\quad\sum_{k=0 }^{P }kn_k=L
\end{equation*}
which we will denote as restriction $R_\Romannum{1}$ and $R_\Romannum{2}$ respectively.

Now consider the number of configuration $D(N,P,n_{k|P})$ obtained by counting different ways to choose the set of $n_{k|P}\equiv(n_1,\dots,n_P)$, which is evaluated by choosing $n_0$ from $N$ boxes, followed by choosing $n_1$ from $N-n_0$ boxes, and so on. Hence,
\begin{align*}
    D(N,P,n_{k|P})&={N\choose n_0}\cdots{N-\cdots-n_{p-1} \choose n_p}\\
    &=\frac{N!}{n_0!(N-n_0)!}\cdots \frac{(N-\cdots-n_{p-1}!)}{n_p!(N-\cdots-n_p)!}\\
    D(N,P,n_{k|P})&=\frac{N!}{\displaystyle\prod_{i=0}^{P}n_i!}
\end{align*} 
The number of configuration satisfies the first condition, however it does not satisfy the second condition since the number of objects in the set $n_{k|P}$ is 
\begin{equation*}
    \sum_{k=0}^{P }kn_k\equiv M(n_{k|P})
\end{equation*}
is not necessarily $L$. Our task is then to find the configuration $D$ which satisfies our restriction, formally 
\begin{equation*}
    D(N,P,L)=\sum_{R_\Romannum{1}\text{ and }R_\Romannum{2}}D(N,P,n_{k|P})
\end{equation*}

To find the number of configuration that satisfy those two restriction, we first consider the value of summing $D(N,P,n_{k|P})$ over all possible value of $n_k$; this makes it so that $D$ satisfies the first condition, but not the second. Formally 
\begin{equation*}
    D_T(N,P)\equiv \sum_{R_\Romannum{1}}D(N,P,n_{k|P})
\end{equation*}
Using the result that we derived previously 
\begin{equation*}
    D_T(N,P)= \frac{N!}{\displaystyle\prod_{i=0}^{P}n_i!}
\end{equation*}
Recall the multinomial theorem
\begin{equation*}
    \left(\sum_{i=0 }^{P}x_i\right)^N=\sum_{R_\Romannum{1}}^{}\frac{N}{\displaystyle\prod_{i=0}^{P}n_i!} \; \prod_{i=0}^{P}x_{i}^{n_{i}}
\end{equation*}
Let $x_i=1$ for all $i$, and we get 
\begin{equation*}
    (P+1)^N=\sum_{R_\Romannum{1}}^{}\frac{N}{\displaystyle\prod_{i=0}^{P}n_i!} 
\end{equation*}
Therefore 
\begin{equation*}
    D_T(N,P)=(P+1)^N
\end{equation*}
This is the number of ways to distribute $M=0,\dots, NP$ objects in $N$ boxes, with each box only having maximum $P$ objects. What we want however is $M=L$. To do that, we put $x_i=x^i$ in to multinomial theorem
\begin{align*}
    \left(\sum_{i=0 }^{P}x^i\right)^N&=\sum_{R_\Romannum{1}}^{}\frac{N}{\displaystyle\prod_{i=0}^{P}n_i!} \; \prod_{i=0}^{P}x^{i\cdot n_{i}}
    =\sum_{R_\Romannum{1}}^{}\frac{N}{\displaystyle\prod_{i=0}^{P}n_i!} \; x^{\displaystyle\sum_{i=0}^{P}i\cdot n_{i}}\\
    \left(\sum_{i=0 }^{P}x^i\right)^N&=\sum_{R_\Romannum{1}}^{}\frac{N}{\displaystyle\prod_{i=0}^{P}n_i!} \; x^{M(n_{i|P})}
\end{align*}
Clearly,
\begin{equation*}
    D_T(N,P,L)=\text{Coefficient of $x^L$ in } \left(\sum_{i=0 }^{P}x^i\right)^N
\end{equation*}
which is obtained by 
\begin{equation*}
    D(N,P,L)=\frac{1}{L!}\frac{d^L}{dx^L}\left(\sum_{i=0 }^{P}x^i\right)^N \bigg|_{x=0}\quad\blacksquare
\end{equation*}

Now we consider special case when $L\leq P$. Note that the $L$-th derivative of $f(x) $, especially $x^{L+k}$ with $k\equiv 1,2.\dots$ at $x=0$ is also zero. We can expand the definition of $f(x)$ as polynomial degree $P$ into degree infinity and write
\begin{equation*}
    f(x)=\left(\sum^{\infty}_{i=0}x^i\right)^N
\end{equation*}
Evaluating it
\begin{equation*}
    f(x)=(1-x)^{-N}
\end{equation*}
Substituting it into the expression for $D(N,P,L)$, we see that 
\begin{multline*}
    D(N,P,L)=\frac{1}{L!}\frac{d^L}{dx^L}(1-x)^N =\frac{1}{L!}N\frac{d^{L-1}}{dx^{L-1}}(1-x)^{N-1}\\ =\frac{1}{L!} N(N+1) \frac{d^{L-2}}{dx^{L-2}}(1-x)^{N-2}
\end{multline*}
Hence, in general
\begin{equation*}
    D(N,P,L)=\frac{1}{L!}\frac{(N+L-1)!}{(N-1)!}\quad \blacksquare
\end{equation*}
\end{document}