\documentclass[../main.tex]{subfiles}
\begin{document}
Mainly consist of precalculus, and basic Calculus.
\subsection*{Algebra}
\subsubsection*{Laws of Exponents.}
\begin{align*}
    x^{\tfrac{m}{n}}&=\sqrt[n]{m}\\
    (x^m)^n&=x^{mn}\\
    x^mx^n&=x^{m+n}\\
    x^ay^a&=(xy)^a\\
\end{align*}

\subsubsection*{Special Factorization.}
\begin{align*}
    x^2-y^2&=(x+y)(x-y)\\
    x^3-y^3&=(x-y)(x^2+xy+y^2)\\
    x^3+y^3&=(x+y)(x^2-xy+y^2)\\
\end{align*}

\subsubsection*{Quadratic formula.} The following formula can be used to find the roots of quadratic equation $ax^2+bx+c$
\begin{align*}
    x_1,x_2=\dfrac{-b\pm \sqrt{b^2-4ac}}{2a}\;\;
    \begin{cases}
        D>0 \;\;\text{re(2)}\\
        D=0\;\;\text{re(1)}\\
        D<0\;\;\text{im(2)}
    \end{cases}
\end{align*}
The said quadratic equation can also be writen as 
\begin{equation*}
    ax^2+bc+c=a(x-x_1)(x-x_2)
\end{equation*}

\subsubsection*{Binomial theorem.}
\begin{align*}
    (a+b)^n=\sum_{k=0}^{\infty}\binom{n}{k}a^{n-k}b^{k}
\end{align*}
with
\begin{equation*}
    \binom{n}{k} = \frac{n!}{k!(n-k)!}=\frac{\Gamma (n+1)}{\Gamma(k+1)\Gamma(n-k+1)}
\end{equation*}

\subsection*{Trigonometry}
\subsubsection*{Trigonometry Definitions.}
\begin{align*}
    \sin \theta&=\dfrac{1}{\csc \theta}\\
    \cos \theta&=\dfrac{1}{\sec \theta}\\
    \tan \theta&=\dfrac{1}{\cot \theta}\\
\end{align*}

\subsubsection*{Pythagorean Identities.}
\begin{align*}
    \sin^2 \theta+ \cos^2 \theta &=1\\
    \sec^2 \theta- \tan^2 \theta &=1\\
    \csc^2 \theta- \cot^2 \theta &=1\\
\end{align*}

\subsubsection*{Law of Sines.}
\begin{align*}
    \frac{\sin A}{a}=\frac{\sin B}{b}=\frac{\sin C}{c}
\end{align*}

\subsubsection*{Law of Cosines.}
\begin{align*}
    a^2=b^2+c^2-2bc\cos A
\end{align*}

\subsubsection*{Trigonometry Double Angle Identities.}
\begin{align*}
    \sin 2\theta&=2\sin \theta \cos \theta\\
    \cos 2 \theta&=1-2\sin^2\theta\\
    &=2\cos^2\theta -1\\
    &=\cos^2\theta-\sin^2\theta\\
    \tan 2\theta&=\frac{2\tan\theta}{1-\tan^2\theta}
\end{align*}

\subsubsection*{Trigonometry Addition and Difference Identities.}
\begin{align*}
    \sin(x+y)&=\sin x \cos y+ \cos x \sin y\\
    \sin(x-y)&=\sin x \cos y- \cos x \sin y\\
    \cos(x+y)&=\cos x \cos y- \sin x \sin y\\
    \cos(x-y)&=\cos x \cos y- \sin x \sin y\\
    \tan (x+y)&=\frac{\tan x+\tan y}{1-\tan x\tan y}\\
    \tan (x-y)&=\frac{\tan x-\tan y}{1+\tan x\tan y}\\
\end{align*}

\subsubsection*{Trigonometry Product Rule.}
\begin{align*}
    \cos x \cos y&= \frac{1}{2}[\cos(x-y)+\cos(x+y)]\\
    \sin x \sin y&= \frac{1}{2}[\cos(x-y)-\cos(x+y)]\\
    \sin x \cos y&= \frac{1}{2}[\sin(x+y)+\sin(x-y)]\\
    \cos x \sin y&= \frac{1}{2}[\sin(x+y)-\sin(x-y)]
\end{align*}

\subsubsection*{Neat Mnemonics.}
\begin{align*}
    \begin{vmatrix}
        \mathrm{S^+}\\
        \mathrm{S^-}\\
        \mathrm{C^+}\\
        \mathrm{C^-}
    \end{vmatrix}=
    \begin{vmatrix}
        \mathrm{SC+CS}\\
        \mathrm{SC-CS}\\
        \mathrm{CC-SS}\\
        \mathrm{CC+SS}
    \end{vmatrix}
    &&\mathrm{and}&&
    \begin{vmatrix}
        \mathrm{CC}\\
        \mathrm{SS}\\
        \mathrm{SC}\\
        \mathrm{CS}
    \end{vmatrix}=
    \begin{vmatrix}
        \mathrm{C^-+C^+}\\
        \mathrm{C^--C^+}\\
        \mathrm{S^++S^-}\\
        \mathrm{S^+-S^-}
    \end{vmatrix}
\end{align*}

\subsection*{Logarithm}
\subsubsection*{Definition (informal).} $\log_a b$ means $a$ to the power of what equal $b$.

\subsubsection*{Few important log rule.}
\begin{align*}
    \log_c (ab)&=\log_c (a)+\log_c (b)\\
    \log_c (\frac{a}{b})&=\log_c (a)-\log_c (b)\\
    \log_a b&=\frac{\log_c (b)}{\log_c (a)}\\
    a^{\log_a b}&=b
\end{align*}

\subsection*{Limit}
\subsubsection*{Few Important Limits.}
\begin{align*}
    \lim_{x\to a} c&=c\\
    \lim_{x\to 0^+} \frac{1}{x}&=\infty\rightarrow\frac{1}{0^+}=\infty\\
    \lim_{x\to 0^-} \frac{1}{x}&=-\infty\rightarrow\frac{1}{0^-}=-\infty
\end{align*}
\subsubsection*{Limit as Definition of Derivative.}
\begin{align*}
    \df y=\lim_{h\to 0}\frac{f(x+h)-f(x)}{h}
\end{align*}

\subsection*{Derivative}

\subsubsection*{Order of calculation.} How to determine the order of derivation: last computation is the first thing to do.

\subsubsection*{General Formula.}
\begin{align*}
     D\; x^n&=nx^{n-1}\\
     D\; (uv)&= D\;u\cdot v+u\cdot D\;v\\
     D\;\bigg(\frac{u}{v}\bigg)&=\frac{ D\;u\cdot v-u\cdot D\;v}{v^2}
\end{align*}

\subsubsection*{Trigonometry Formula.}
\begin{align*}
     D\; \sin x&= \cos x\\
     D\; \cos x&= -\sin x\\
     D\; \tan x&= \sec^2 x\\
     D\; \cot x&= -\csc^2 x\\
     D\; \sec x&= \sec x\tan x\\
     D\; \csc x&= -\cot x\csc x\\
\end{align*}

\subsubsection*{Neat Mnemonics.}
\begin{align*}
\begin{matrix}
    \sec&\sec&\tan&\downarrow \textrm{cofunction}\\
    \csc&-\csc&\cot&\\
    &\leftrightarrow\textrm{multiply}
\end{matrix}
\end{align*}

\subsubsection*{Exponential and Logarithmic Functions.}
\begin{align*}
     D\; \ln x &= \frac{1}{x}\\
     D\; a^n&=a^n\ln x\\
     D\; \log_a b&=\frac{1}{b\ln a}
\end{align*}

\subsubsection*{Minima and Maxima test.} First derivative test:
\begin{itemize}
    \item Determine critical points ($ Dy=0$), then divide into region;
    \item Pick value from each region and plug into \emph{derivative}; and
    \item Do the sign-graph thing.
\end{itemize}
Second derivative test:
\begin{itemize}
    \item Determine critical points;
    \item Plug critical into second derivative; and
    \item Positive $ D^2y$ means concave up ($\smile$) or minima, negative means concave down ($\frown$) or maxima, and 0 means inconclusive. Simply put, positive means minima, while negative means maxima.
\end{itemize}

\subsubsection*{Optimization with constrain.} Elimination method:
\begin{enumerate}
    \item Write the function $f$. The function itself must be in terms of one independent variable, say $x$, which can often be achieved by substituting our constraint, say $y(x)$, into the function $f(x)$.
    \item Find the critical points.
    \item Use whatever test you need.
\end{enumerate}
Implicit differentiation method:
\begin{enumerate}
    \item Write the function $f(x,y)$. This method assumes that it is not possible to solve substituting the constant $y$ into our equation.
    \item Write the differentiation with respect to independent $x$ variable. Note that the derivative of the dependent variable often still remains; we need to solve for them too.
    \item Use the following result to find the critical points.
    \item Use the second derivative test. Note that you'll also need the second derivative of the constraint $y$ evaluated at critical points to determine the second derivative of the function $f(x,y)$
\end{enumerate}

\subsubsection*{Differentiation under integral sign.} Differentiation under integral sign stated by Leibniz' rule
\begin{equation*}
    \frac{d}{dx}\int_{u(x)}^{v(x)}f(x,t)\;dt=\int_{u}^{v}\frac{\partial f}{\partial x}\;dt + f(x,v)\frac{dv}{dx}-f(x,u)\frac{du}{dx}
\end{equation*}

\emph{Proof.} Suppose we want $dI/dx$ where
\begin{equation*}
    I=\int_{u }^{v }f(t)\;dt
\end{equation*}
By the fundamental theorem of calculus
\begin{equation*}
    I=F(v)-F(u)=\mathcal{F}(v,u)
\end{equation*}
or $I$ is a function of $v$ and $u$. Finding $dI/dx$ is then a partial differentiation problem. We can write
\begin{equation*}
    \frac{dI}{dx}=\frac{\partial I}{\partial v}\frac{dv}{dx}+\frac{\partial I}{\partial u}\frac{du}{dx}
\end{equation*}
By the fundamental theorem of calculus, we have 
\begin{align*}
    \frac{d}{dv}\int_{a}^{v}f(x)\;dt&=\frac{d}{dv}\bigl[F(v)-F(a)\bigr]=f(v)\\
    \frac{d}{dv}\int_{u}^{b}f(x)\;dt&=\frac{d}{dv}\bigl[F(b)-F(u)\bigr]=-f(u)
\end{align*}
where $u$ and $v$ are a function of $x$, while $a$ and $b$ are a constant. This is the case when we consider $\partial I/\partial v$ or $\partial I/\partial v$; the other variable is constant. Then 
\begin{equation*}
    \frac{d}{dx}\int_{u }^{v }f(t)\;dt=f(v)\frac{dv}{dx}-f(u)\frac{du}{dx}
\end{equation*}

Under not too restrictive conditions, 
\begin{equation*}
    \frac{d}{dx}\int_{a }^{b }f(x,t)\;dt =\int_{a }^{b }\frac{\partial f(x,t)}{\partial x}\;dt
\end{equation*}
where, as before, $a$ and $b$ are constant. In other words, we can differentiate under the integral sign. It is convenient to collect these formulas into one formula known as Leibniz' rule:
\begin{equation*}
    \frac{d}{dx}\int_{u(x)}^{v(x)}f(x,t)\;dt=\int_{u}^{v}\frac{\partial f}{\partial x}\;dt + f(x,v)\frac{dv}{dx}-f(x,u)\frac{du}{dx}\quad\blacksquare
\end{equation*}

\subsubsection*{Leibniz’ rule for differentiating a product.}
\begin{equation*}
    \left(\frac{d}{dx}\right)^n(fg)=\sum_{k=0}^{n}{n\choose k}\left(\frac{d}{dx}\right)^{n-k}(f)\left(\frac{d}{dx}\right)^k(g)
\end{equation*}
where
\begin{equation*}
    {n \choose k}={\frac{n!}{k! (n-k)!}}
\end{equation*}

\subsection*{Integral}
\subsubsection*{Basic Formula (integration constant omitted).}
\begin{align*}
    \int x^n \;dx &= \frac{1}{n+1}x^{n+1}\\
    \int \frac{1}{x}\;dx& = \ln |x|\\
    \int u \;dv& = uv - \int v\; du\\
    \int a^x \;dx&=\frac{a^x}{\ln a}
\end{align*}

\subsubsection*{Trigonometry.}
\begin{align*}
    \int \sin x \; dx&=-\cos x\\
    \int \cos x \; dx&=\sin x\\
    \int \sec^2 x \; dx&=\tan x\\
    \int \csc^2 x \; dx&=-\cot x\\
    \int \sec x\tan x \; dx&=\sec x\\
    \int \csc x\tan x \; dx&=-\csc x
\end{align*}

\subsubsection*{Root.}
\begin{align*}
    \int \frac{1}{\sqrt{a^2-x^2}}\;dx&=\arcsin \frac{x}{a}\\
    \int \frac{1}{\sqrt{x^2\pm a^2}}\;dx&=\ln x +\sqrt{x^2\pm a^2}\\
    \int \frac{1}{\sqrt{a^2+x^2}}\;dx&=\frac{1}{a}\arctan \frac{x}{a}
\end{align*}

\subsubsection*{Integration by part.}
\begin{enumerate}
    \item Splits the integrand. Choose $u$ using LIATEN and let the rest be $dv$. (LIATEN: Log, Inverse trigonometry, Algebra, Trigonometry, ExponeN)
    \item Do the box thing
    \begin{table*}[h]
        \begin{center}
    \caption*{Table: The box thing}
    \begin{tabular}{c || c || c}
        $u$&$v$&$\downarrow$ Differentiate\\ 
        \hline\hline
        $du$&$dv$&$\uparrow$ Integrate
    \end{tabular}
        \end{center}
    \end{table*}
    \item $ \int u \;dv = uv - \int v\; du$
\end{enumerate}

\subsubsection*{Tabular Method.} Refer to the table.
\begin{table*}[h]
\begin{center}
\caption*{Table: The table}
\begin{tabular}{c || c || c}
    &Differentiate&Integrate \\ 
    \hline\hline
    +&$a\searrow$&b \\ 
    \hline
    -&$a'\searrow$&b \\ 
    \hline
    +&$a''\searrow$&b \\ 
    \hline
    $\vdots$&$\vdots$&$\vdots$
\end{tabular}
\end{center}
\end{table*}
Steps:
\begin{enumerate}
    \item 0 in $D$ column or use LIATEN,
    \item Integrate a row, and
    \item A row repeats.
\end{enumerate}

\subsubsection*{Trigonometry Integral.} Pythagorean Identity.
\begin{align*}
    \sin^2x+\cos^2x&=1\\
    \sin^2x&=\frac{1-\cos 2x}{2}\\
    \cos^2x&=\frac{1+\cos2x}{2}
\end{align*}
note that argument inside quadratic trigonometry is half of trigonometry, which means $\cos^22x=(1+\cos 4x)/{2}$. There are few cases of tricky trigonometry integral. First, if power of sin is odd and positive. The steps to evaluate it are as follows.
\begin{enumerate}
    \item Remove one power off
    \item convert remaining (even power) using Pythagorean Identity in terms of cosine 
    \item integrate using subs method
\end{enumerate}
If the power of sine is odd and positive.
\begin{enumerate}
    \item Same as before
\end{enumerate}
If the power of sine and cosine is even and nonegative, then:
\begin{enumerate}
    \item convert using Pythagorean Identity and solve
\end{enumerate}

\subsubsection*{Trigonometry substitution.} Trigonometry function and its radical pair
\begin{align*}
    \tan \theta&=\sqrt{u^2+a^2}\\
    \sin \theta&=\sqrt{a^2-u^2}\\
    \sec \theta&=\sqrt{u^2-a^2}\\
\end{align*}
where $u$ is the variable we are differentiating with respect to. Mnemonics: $+$ looks like tangent; $-$ for sin and sec; and it is \emph{a} \emph{s}in. Trigonometry substitution step is then as follows.
\begin{enumerate}
    \item Draw a right triangle where trigonometry pair equal $u/a$
    \item using the trigonometry pair equation*, solve for $x$ and $dx$
    \item find trigonometry where $\sqrt{}/a$
    \item subs again if equation* still contain $\theta$ and solve
\end{enumerate}

\subsubsection*{Partial Fraction. }
\begin{enumerate}
    \item Factor out denominator
    \item Breakup the function and put unknown (Capital Letter) into numerator. Put numerator normally if factor is linear, put $Px+Q$ Irreducible quadratic factor IQF. In general, \begin{equation*}
        \frac{Ax^{n-1}+Bx^{n-2}+\cdots}{x^{n}+x^{n-1}+\cdots}
    \end{equation*}
    \item Multiply both side by left side's denominator
    \item Take the roots of the linear factors and plug them into x, and solve for the unknowns
    \item Put unknowns into step 2
    \item Splits Integral, then solve
    \item For equating coefficients like terms, after step 3, expand equation*. Then, collect like terms and equate coefficient of like terms from both side
\end{enumerate}
\clearpage

\subsection*{Appendix: Example of Optimization Problem}
\subsubsection*{One variable problem.} Consider this simple problem.
\begin{quotation}
    What is the maximum volume of a box without top that can be made from a square plate with a side of 30 unit, given that you can cut of its corner?
\end{quotation}
\begin{figure*}[b]
    \centering
    \normfig{../../Rss/Fundamentals/OptimizationBox.png}
    \caption*{Figure: Plate and its configuration}
\end{figure*}

We know the volume of the box is calculated by $V=lwh$, then the volume as function of height is
\begin{equation*}
    V(h)=(30-2h)(30-2h)h=4h^3-120h^2+900
\end{equation*}
This is the function that we want to maximize. Setting the derivative to zero, we have 
\begin{equation*}
    \begin{array}{l l}
        \dfrac{dV}{dh}&=12h^2-240h+900\\
        0&=h^2-20h+75
    \end{array}\implies    
    h=15\;\lor\; h=5
\end{equation*}
Putting those critical number, into $V(h)$, we obtain $V(5)=2000$ and $V(15)=0$. Hence, box with height 5 will maximize the volume.

\subsubsection*{Two variable problem.} More complicated, but still very easy.
\begin{quotation}
    Consider a floorless pup tent made from the least possible material with width $2w$, length $l$, creating angle $\theta$. Find $\theta$.
\end{quotation}

First we consider the volume and area of the shape in question.
\begin{align*}
    V&=\frac{2w}{2}w\tan \theta\cdot l=w^2l\tan\theta\\
    A&=\frac{2w}{2}w\tan \theta\cdot 2+\frac{wl}{\cos \theta}\cdot 2= 2w^2\tan\theta+2wl\sec \theta
\end{align*}
Solving for $l$ from the equation of $V$, we get $l=V/w^2\tan \theta$. Then we substitute the result into $A$ to obtain
\begin{equation*}
    A=2w^2\tan\theta+\frac{2wV\sec \theta}{w^2\tan\theta}= 2w^2\tan\theta+\frac{2V}{w}\csc\theta
\end{equation*}
This is the function that we want to minimize. Since $A$ is a function of two variable $A(w,\theta)$, we need to differentiate it with respect to those two variable
\begin{align*}
    \frac{\partial A}{\partial w}&=4w\tan \theta-\frac{2V}{w^2}\csc \theta=0\\
    \frac{\partial A}{\partial \theta}&=2w^2\sec^2\theta-\frac{2V}{w}\csc\theta\cot\theta=0
\end{align*}
Solving those equations for $w^3$, we get
\begin{align*}
    w^3=\frac{V\csc\theta}{2\tan\theta}\;\land\; w^3=\frac{V\csc\theta\cot\theta}{\sec^2\theta}
\end{align*}
Equating them to obtain
\begin{equation*}
    \frac{1}{\sec^2\theta}=\frac{1}{2}\implies \theta=\frac{\pi}{4}=45^\circ 
\end{equation*}

\subsection*{Appendix: Example of Optimization Problem with Constrain.} 
We try to solve this example using few methods: elimination, implicit derivative, and Lagrange multiplier. Now, consider this problem.
\begin{quotation}
    What is the shortest distance from origin to curve $y=1-x^2$?
\end{quotation} 

\subsubsection*{Elimination.} What we want to minimize is the distance $d=({x^2+y^2})^{1/2}$, however it is more convenient to minimize $f=x^2+y^2$ instead. The function $y=1-x^2$ acts as constraint. Substituting the constraint into our function, we have 
\begin{equation*}
    f(x)=x^2+(1-x^2)^2=x^4-x^2+1
\end{equation*}
The critical points of this function determined by 
\begin{equation*}
    \frac{df}{dx}=4x^3-2x=x(4x^2-2)=0\implies x=0\lor x=\pm\sqrt{1/2}
\end{equation*}
Then to determine the maxima or minima of the function, we use second test derivative 
\begin{equation*}
    \frac{d^2f}{dx^2}=12x^2-2=\begin{cases}
        -2, &x=0, \quad\text{Maxima}\\
        4, &x=\pm \sqrt{1/2},\quad \text{Minima}\\
    \end{cases}
\end{equation*}
Therefore, the minimum distance is 
\begin{equation*}
    d=\left[\left(\sqrt{\frac{1}{2}}\right)^2+\left(1-\frac{1}{2}\right)^2\right]^{1/2}=\frac{\sqrt{3}}{2}
\end{equation*}

\subsubsection*{Implicit differentiation.} We use this method if it is not possible to substitute the constraint $y(x)$ into our function $f$. Differentiating $f(x,y)x^2+y^2$ with respect to $x$
\begin{equation*}
    \frac{df}{dx}=2x+2y\frac{dy}{dx}
\end{equation*}
From our constraint equation, we have the relation of $dy$ in terms of $dx$ as $dy=-2x\;dx$. Substituting this equation into $df/dx$ while also setting it equal to zero 
\begin{equation*}
    2x-4xy=x(1-2y)=0\implies x=0\lor y=1/2
\end{equation*}
And we obtain one of the critical points. To obtain the rest, we substitute the result $y=1$ into our constraint equation $y=1-x^2$. We have then 
\begin{equation*}
    x=\left(-\sqrt{\frac{1}{2}},0,\sqrt{\frac{1}{2}}\right)
\end{equation*}
The second derivative test for this method is rather different from the usual. What differs is simply how to evaluate the second derivative. First we determine the second derivative of $f(x,y)$ with respect to $x$
\begin{equation*}
    \frac{d^2f}{dx^2}=\frac{d}{dx}\left(2x+2y\frac{dy}{dx}\right)=2+\left(\frac{dy}{dx}\right)^2 +2y\frac{d^2y}{dx^2}
\end{equation*}
At $x=0$, we have $y=1,\;dy/dx=0,\text{ and } d^2y/dy^2=-2$; while at $x=\pm\sqrt{1/2}$, we have $y=1/2,\;dy/dx=\mp\sqrt{2},\text{ and } d^2y/dy^2=-2$. Hence,
\begin{equation*}
    \frac{d^2f}{dx^2}\bigg|_{x=0}=-2\text{ (Maxima)}\quad\land\quad \frac{d^2f}{dx^2}\bigg|_{x=\pm\sqrt{1/2}}=-4\text{ (Minima)}
\end{equation*}
as before. Substituting the result into the equation for distance $d=(x^2+y^2)^{1/2}$ and we have the same result

\subsubsection*{Lagrange multiplier.} We have the function that we want to maximize $f(x,y)=x^2+y^2$ and constrain $\phi(x,y)=x^2+y=1$. Then we construct the function
\begin{equation*}
    F(x,y)=x^2+y^2+\lambda(x^2+y)
\end{equation*}
The partial derivative of $F(x,y)$ with respect to each variable is 
\begin{equation*}
    \frac{\partial F}{\partial x}=2x+2\lambda x\quad\land\quad \frac{\partial F}{\partial y}=2y+\lambda 
\end{equation*}
In addition with the constraint, we then need to solve those three equations
\begin{align*}
    2x+2\lambda x&=0\\
    2y+\lambda& =0\\
    x^2+y&=1
\end{align*}
Solving the first equation
\begin{equation*}
    x(1+\lambda)=0\implies x=0\lor\lambda=-1
\end{equation*}
Using $\lambda $ on the second equation
\begin{equation*}
    2y+\frac{1}{2} =0\implies y=\frac{1}{2}
\end{equation*}
Hence the constraint equation reads
\begin{equation*}
    x^2+\frac{1}{2}=1\implies x=\pm\sqrt{1/2}
\end{equation*}
As before, we obtain critical points $x=\left(-\sqrt{1/2},0,\sqrt{1/2}\right)$. We then can move to the next step:  minima test and calculating the distance, both will need not to be repeated.

\subsection*{Appendix: Integration Technique Example.} 
\subsubsection*{Trigonometry substitution.} Find $\int \dfrac{dx}{\sqrt{9x^2+4}}$. Refer to the Mnemonics, the trigonometry pair is tangent.
\begin{align*}
    I&=\int \dfrac{dx}{\sqrt{(3x)^2+2^2}}\\
    \tan\theta&=\frac{3x}{2}
\end{align*}
solving for x and dx
\begin{align*}
    x&=\frac{2}{3}\tan\theta\\
    dx&=\frac{2}{3}\sec^2\theta \;d\theta
\end{align*}
trigonometry where $\frac{\sqrt{}}{a}$ holds is secant, solving for radical
\begin{align*}
    \sec \theta&=\frac{\sqrt{9x^2+4}}{2}\\
    \sqrt{9x^2+4}&=2\sec\theta
\end{align*}
the integral is then
\begin{align*}
    I&=\frac{1}{3}\int \sec\theta \;d\theta\\
    &=\frac{1}{3}\ln |\sec\theta+\tan\theta|+C
\end{align*}
substituting the $\theta$ function
\begin{align*}
    I&=\frac{1}{3}\ln \bigg|\frac{\sqrt{9x^2+4}}{2}+\frac{3x}{2}\bigg|+C\\
    &=\frac{1}{3} \ln \bigg | \sqrt{9x^2+4} +3 \bigg|+C
\end{align*}

\subsection*{Appendix: Moar Integral}
\subsubsection*{Basic.} Most common integrals.
\begin{flalign*}
&\begin{aligned}
        &\int \frac{1}{x}\;dx = \ln |x| \\
        &\int u \;dv = uv - \int v \;du    \\
        &\int u\;dv = uv - \int v \;du\\
        &\int \frac{1}{ax+b}\;dx = \frac{1}{a} \ln |ax + b| \\
\end{aligned}&
\end{flalign*}

\subsubsection*{Rational Functions.} Integrals of rational function
\begin{flalign*}
    &\begin{aligned}
        &\int \frac{1}{(x+a)^2}\;dx = -\frac{1}{x+a}\\
        &\int (x+a)^n \;dx = \frac{(x+a)^{n+1}}{n+1}, n\ne -1\\
        &\int x(x+a)^n \;dx = \frac{(x+a)^{n+1} ( (n+1)x-a)}{(n+1)(n+2)}\\
        & \int \frac{1}{1+x^2}\;dx = \tan^{-1}x\\
        &\int \frac{1}{a^2+x^2}\;dx = \frac{1}{a}\tan^{-1}\frac{x}{a}\\
        &\int \frac{x}{a^2+x^2}\;dx = \frac{1}{2}\ln|a^2+x^2|\\
        &\int \frac{x^2}{a^2+x^2}\;dx = x-a\tan^{-1}\frac{x}{a}\\
        &\int \frac{x^3}{a^2+x^2}\;dx = \frac{1}{2}x^2-\frac{1}{2}a^2\ln|a^2+x^2|\\
        &\int \frac{1}{ax^2+bx+c}\;dx = \frac{2}{\sqrt{4ac-b^2}}\tan^{-1}\frac{2ax+b}{\sqrt{4ac-b^2}}\\
        &\int \frac{1}{(x+a)(x+b)}\;dx = \frac{1}{b-a}\ln\frac{a+x}{b+x}, \quad a\neq b\\
        &\int \frac{x}{(x+a)^2}\;dx = \frac{a}{a+x}+\ln |a+x|\\
        &\int \frac{x}{ax^2+bx+c}\;dx = \frac{1}{2a}\ln|ax^2+bx+c|-\\
        &\frac{b}{a\sqrt{4ac-b^2}}\tan^{-1}\frac{2ax+b}{\sqrt{4ac-b^2}}\\
    \end{aligned}&
\end{flalign*}
  
\subsubsection*{Roots.} Integrals of roots.
\begin{flalign*}
    &\begin{aligned}
        &\int \sqrt{x-a}\; dx = \frac{2}{3}(x-a)^{3/2}\\
        &\int \frac{1}{\sqrt{x\pm a}}\; dx = 2\sqrt{x\pm a} \\
        &\int \frac{1}{\sqrt{a-x}}\; dx = -2\sqrt{a-x} \\
        &\int x\sqrt{x-a}\; dx =  \begin{cases}\frac{2 a}{3} \left({x-a}\right)^{3/2} +\frac{2 }{5}\left( {x-a}\right)^{5/2},\text{ or} \\ \frac{2}{3} x(x-a)^{3/2} - \frac{4}{15} (x-a)^{5/2}, \text{ or}\\ \frac{2}{15}(2a+3x)(x-a)^{3/2}
        \end{cases}\\
        &\int \sqrt{ax+b}\; dx = \left(\frac{2b}{3a}+\frac{2x}{3}\right)\sqrt{ax+b} \\
        &\int (ax+b)^{3/2}\; dx =\frac{2}{5a}(ax+b)^{5/2}\\
        &\int \frac{x}{\sqrt{x\pm a} } \; dx = \frac{2}{3}(x\mp 2a)\sqrt{x\pm a}\\
        &\int \sqrt{\frac{x}{a-x}}\; dx =  -\sqrt{x(a-x)}-a\tan^{-1}\frac{\sqrt{x(a-x)}}{x-a}\\
    \end{aligned}&
\end{flalign*}
\begin{flalign*}
    &\begin{aligned}
        &\int \sqrt{\frac{x}{a+x}}\; dx =  \sqrt{x(a+x)} -a\ln  [ \sqrt{x} + \sqrt{x+a}] \\
        &\int x \sqrt{ax + b}\; dx =\frac{2}{15 a^2}(-2b^2+abx + 3 a^2 x^2)\sqrt{ax+b}\\
        &\int \sqrt{x^3(ax+b)} \; dx =\left[ \frac{b}{12a}-\frac{b^2}{8a^2x}+\frac{x}{3}\right] \sqrt{x^3(ax+b)} \\
        & + \frac{b^3}{8a^{5/2}}\ln \left| a\sqrt{x} + \sqrt{a(ax+b)} \right|\\
        &\int  \sqrt{a^2 - x^2}\; dx = \frac{1}{2} x \sqrt{a^2-x^2} +\frac{1}{2}a^2\tan^{-1}\frac{x}{\sqrt{a^2-x^2}}\\
        &\int  x \sqrt{x^2 \pm a^2}\; dx= \frac{1}{3}\left( x^2 \pm a^2 \right)^{3/2} \\
        &\int \frac{1}{\sqrt{x^2 \pm a^2}}\; dx = \ln \left| x + \sqrt{x^2 \pm a^2} \right| \\
        &\int \frac{1}{\sqrt{a^2 - x^2}}\; dx = \sin^{-1}\frac{x}{a} \\
        &\int \frac{x}{\sqrt{x^2\pm a^2}}\; dx = \sqrt{x^2 \pm a^2} \\
        &\int \frac{x}{\sqrt{a^2-x^2}}\; dx = -\sqrt{a^2-x^2} \\
        &\int \frac{x^2}{\sqrt{x^2 \pm a^2}}\; dx = \frac{1}{2}x\sqrt{x^2 \pm a^2}\mp \frac{1}{2}a^2 \ln \left| x + \sqrt{x^2\pm a^2} \right| \\
        &\int\frac{1}{\sqrt{ax^2+bx+c}}\;dx=\frac{1}{\sqrt{a}}\ln \left| 2ax+b + 2 \sqrt{a(ax^2+bx+c)} \right| \\
        &\int\frac{dx}{(a^2+x^2)^{3/2}}=\frac{x}{a^2\sqrt{a^2+x^2}}\\
    \end{aligned}&
\end{flalign*}

\subsubsection*{Integrals with Logarithms.}

\begin{flalign*}
    &\begin{aligned}
        &\int \ln ax \;dx = x \ln (ax) - x \\
        &\int \frac{\ln ax}{x} \;dx = \frac{1}{2} ( \ln ax )^2 \\
        &\int \ln (ax + b) dx =  ( x + \frac{b}{a} ) \ln (ax+b) - x , \quad a \neq 0\\
        &\int \ln  ( x^2 + a^2 )\;{dx} = x \ln (x^2 + a^2  ) +2a\tan^{-1} \frac{x}{a} - 2x\\
        &\int \ln  ( x^2 - a^2 )\;dx= x \ln (x^2 - a^2  ) +a\ln \frac{x+a}{x-a} - 2x\\
        &\int \ln  ( x^2 - a^2 )\;dx = x \ln (x^2 - a^2  ) +a\ln \frac{x+a}{x-a} - 2x \\
        &\int \ln  ( ax^2 + bx + c) dx  = \frac{1}{a}\sqrt{4ac-b^2}\tan^{-1}\frac{2ax+b}{\sqrt{4ac-b^2}} -2x\\
        &+ ( \frac{b}{2a}+x )\ln \ (ax^2+bx+c ) \\
    \end{aligned}&
\end{flalign*}
\begin{flalign*}
    &\begin{aligned}
        &\int x \ln (ax + b) dx = \frac{bx}{2a}-\frac{1}{4}x^2+\frac{1}{2}(x^2-\frac{b^2}{a^2})\ln (ax+b) \\
        &\frac{1}{2}( x^2 - \frac{a^2}{b^2}  ) \ln  (a^2 -b^2 x^2 )\\
    \end{aligned}&
\end{flalign*}
    
\subsubsection*{Integrals with Exponential.}
\begin{flalign*}
    &\begin{aligned}
        &\int e^{ax} dx = \frac{1}{a}e^{ax} \\
        &\int \sqrt{x} e^{ax} dx = \frac{1}{a}\sqrt{x}e^{ax} +\frac{i\sqrt{\pi}}{2a^{3/2}}\erf\left(i\sqrt{ax}\right)\\
        &\int x e^x dx = (x-1) e^x \\
        &\int x e^{ax} dx = \left(\frac{x}{a}-\frac{1}{a^2}\right) e^{ax} \\
        &\int x^2 e^{x} dx = \left(x^2 - 2x + 2\right) e^{x}\\
        &\int x^2 e^{ax} dx = \left(\frac{x^2}{a}-\frac{2x}{a^2}+\frac{2}{a^3}\right) e^{ax} \\
        &\int x^3 e^{x} dx = \left(x^3-3x^2 + 6x - 6\right) e^{x} \\
        &\int x^n e^{ax}\; dx= \dfrac{x^n e^{ax}}{a} -\dfrac{n}{a}\int x^{n-1}e^{ax}\; dx\\
        &\int x^n e^{ax}\;dx = \frac{(-1)^n}{a^{n+1}}\Gamma[1+n,-ax]\\
        &\int e^{ax^2}\; dx= -\frac{i\sqrt{\pi}}{2\sqrt{a}}\erf\left(ix\sqrt{a}\right) \\
        &\int e^{-ax^2}\; dx= \frac{\sqrt{\pi}}{2\sqrt{a}}\erf\left(x\sqrt{a}\right) \\
    \end{aligned}&
\end{flalign*}
\begin{flalign*}
    &\begin{aligned}
        &\int x e^{-ax^2}\ dx= -\dfrac{1}{2a}e^{-ax^2} \\
        &\int x^2 e^{-ax^2}\ dx= \dfrac{1}{4}\sqrt{\dfrac{\pi}{a^3}}\erf(x\sqrt{a}) -\dfrac{x}{2a}e^{-ax^2}\\
    \end{aligned}&
\end{flalign*}

\subsubsection*{Integrals with Trigonometry Functions.}
\begin{flalign*}
    &\begin{aligned}
        &\int \sin ax \;dx = -\frac{1}{a} \cos ax \\
        &\int \sin^2 ax \;dx = \frac{x}{2} - \frac{\sin 2ax} {4a} \\
        &\int \sin^n ax \;dx =-\frac{1}{a}\cos ax \times {_2F_1}\left[\frac{1}{2}, \frac{1-n}{2}, \frac{3}{2}, \cos^2 ax \right] \\
        &\int \sin^3 ax \;dx = -\frac{3 \cos ax}{4a} + \frac{\cos 3ax} {12a} \\
        &\int \cos ax \;dx= \frac{1}{a} \sin ax \\
        &\int \cos^2 ax \;dx = \frac{x}{2}+\frac{ \sin 2ax}{4a} \\
        &\int \cos^p ax \;dx  = -\frac{1}{a(1+p)}{\cos^{1+p} ax} \times  {_2F_1}\left[ \frac{1+p}{2}, \frac{1}{2}, \frac{3+p}{2}, \cos^2 ax \right] \\
        &\int \cos^3 ax \;dx = \frac{3 \sin ax}{4a}+\frac{ \sin 3ax}{12a} \\
    \end{aligned}&
\end{flalign*}
\begin{flalign*}
    &\begin{aligned}
        &\int \cos ax \sin bx \;dx = \frac{\cos[(a-b) x]}{2(a-b)} -  \frac{\cos[(a+b)x]}{2(a+b)} ,\quad a\neq b\\
        &\int \sin^2 ax \cos bx \;dx = -\frac{\sin[(2a-b)x]}{4(2a-b)}  + \frac{\sin bx}{2b} - \frac{\sin[(2a+b)x]}{4(2a+b)}\\
        &\int \sin^2 x \cos x \;dx = \frac{1}{3} \sin^3 x\\
        &\int \cos^2 ax \sin bx \;dx = \frac{\cos[(2a-b)x]}{4(2a-b)}  - \frac{\cos bx}{2b} - \frac{\cos[(2a+b)x]}{4(2a+b)}\\
        &\int \cos^2 ax \sin ax \;dx = -\frac{1}{3a}\cos^3{ax} \\
        &\int \sin^2 ax \cos^2 bx \;dx = \frac{x}{4} -\frac{\sin 2ax}{8a}- \frac{\sin[2(a-b)x]}{16(a-b)} +\frac{\sin 2bx}{8b}\\
        &- \frac{\sin[2(a+b)x]}{16(a+b)}\\
        &\int \sin^2 ax \cos^2 ax \;dx = \frac{x}{8}-\frac{\sin 4ax}{32a}\\
        &\int \tan ax \;dx = -\frac{1}{a} \ln \cos ax \\
        &\int \tan^2 ax \;dx = -x + \frac{1}{a} \tan ax \\
        &\int \tan^n ax \;dx =     \frac{\tan^{n+1} ax }{a(1+n)} \times {_2}F_1\left( \frac{n+1}{2}, 1, \frac{n+3}{2}, -\tan^2 ax \right) \\
        &\int \tan^3 ax \;dx = \frac{1}{a} \ln \cos ax + \frac{1}{2a}\sec^2 ax \\
        &\int \sec x \;dx= \ln | \sec x + \tan x | = 2 \tanh^{-1} \left(\tan \frac{x}{2} \right) \\
        &int \sec^2 ax \;dx = \frac{1}{a} \tan ax \\
        &\int \sec^3 x \;dx= \frac{1}{2} \sec x \tan x + \frac{1}{2}\ln | \sec x + \tan x |\\
        &\int \sec x \tan x \;dx = \sec x \\
        &\int \sec^2 x \tan x \;dx = \frac{1}{2} \sec^2 x \\
        &\int \sec^n x \tan x \;dx = \frac{1}{n} \sec^n x , \quad n\neq 0\\
        &\int \csc x \;dx = \ln \left| \tan \frac{x}{2} \right|  = \ln | \csc x - \cot x| + C\\
        &\int \csc^2 ax \;dx = -\frac{1}{a} \cot ax \\
        &\int \csc^3 x \;dx = -\frac{1}{2}\cot x \csc x + \frac{1}{2} \ln | \csc x - \cot x | \\
        &\int \csc^nx \cot x \;dx = -\frac{1}{n}\csc^n x, n\ne 0\\
        &\int \sec x \csc x \;dx = \ln | \tan x | 
    \end{aligned}&
\end{flalign*}
 \subsubsection*{Products of Trigonometry Functions and Monomials.}
\begin{flalign*}
    &\begin{aligned}
        &\int x \cos x \;dx = \cos x + x \sin x \\
        &\int x \cos ax \;dx = \frac{1}{a^2} \cos ax + \frac{x}{a} \sin ax \\
        &\int x^2 \cos x \;dx = 2 x \cos x + \left ( x^2 - 2 \right ) \sin x \\
        &\int x^2 \cos ax \;dx = \frac{2 x \cos ax }{a^2} + \frac{ a^2 x^2 - 2  }{a^3} \sin ax \\
        &\int  x^n \cos x \;dx = -\frac{1}{2}(i)^{n+1}\left[ \Gamma(n+1, -ix)  + (-1)^n \Gamma(n+1, ix)\right] \\
        &\int x^n cos ax \;dx = \frac{1}{2}(ia)^{1-n}\left[ (-1)^n  \Gamma(n+1, -iax)  -\Gamma(n+1, ixa)\right] \\
        &\int x \sin x \;dx = -x \cos x + \sin x \\
        &\int x \sin ax \;dx = -\frac{x \cos ax}{a} + \frac{\sin ax}{a^2} \\
        &\int x^2 \sin x \;dx = \left(2-x^2\right) \cos x + 2 x \sin x\\
        &\int x^2 \sin ax \;dx =\frac{2-a^2x^2}{a^3}\cos ax +\frac{ 2 x \sin ax}{a^2} \\
        &\int x^n \sin x \;dx = -\frac{1}{2}(i)^n\left[ \Gamma(n+1, -ix)  - (-1)^n\Gamma(n+1, -ix)\right] 
    \end{aligned}&
\end{flalign*}

\subsubsection*{Products of Trigonometry Functions and Exponential.}
\begin{flalign*}
    &\begin{aligned}
       &\int e^x \sin x \;dx = \frac{1}{2}e^x (\sin x - \cos x) \\
        &\int e^{bx} \sin ax \;dx = \frac{1}{a^2+b^2}e^{bx} (b\sin ax - a\cos ax) \\
        &\int e^x \cos x \;dx = \frac{1}{2}e^x (\sin x + \cos x)  \\
        &\int e^{bx} \cos ax \;dx = \frac{1}{a^2 + b^2} e^{bx} ( a \sin ax + b \cos ax ) \\
        &\int x e^x \sin x \;dx = \frac{1}{2}e^x (\cos x - x \cos x + x \sin x) \\ 
        &\int x e^x \cos x \;dx = \frac{1}{2}e^x (x \cos x - \sin x + x \sin x) \\
    \end{aligned}&
\end{flalign*}

\subsubsection*{Integrals of Hyperbolic Functions.}
\begin{flalign*}
    &\begin{aligned}
        &\int \cosh ax dx =\frac{1}{a} \sinh ax \\
        &\int e^{ax}  \cosh bx dx = \begin{cases}
            \dfrac{e^{ax}}{a^2-b^2} \left[ a \cosh bx - b \sinh bx \right],\quad    a\neq b \\\\
            \dfrac{e^{2ax}}{4a} + \dfrac{x}{2},\quad a = b
            \end{cases}\\
        &\int \sinh ax dx = \frac{1}{a} \cosh ax \\
        &\int e^{ax} \sinh bx dx =  \begin{cases}
            {\dfrac{e^{ax}}{a^2-b^2} }\left[ -b \cosh bx + a \sinh bx \right]  ,\quad a\neq b \\\\
            {\dfrac{e^{2ax}}{4a} - \dfrac{x}{2}}   a = b
            \end{cases}\\
        &\int  e^{ax} \tanh bx dx =   \begin{cases}
            \displaystyle{ \frac{ e^{(a+2b)x}}{(a+2b)} {_2F_1}\left[ 1+\frac{a}{2b},1,2+\frac{a}{2b}, -e^{2bx}\right] }& \\\\
            \displaystyle{ -\frac{1}{a}e^{ax}{_2F_1}\left[ \frac{a}{2b},1,1E, -e^{2bx}\right]},\quad a\neq b \\\\
            \displaystyle{\frac{e^{ax}-2\tan^{-1}[e^{ax}]}{a} },\quad a = b
            \end{cases}\\
        &\int  \tanh ax\;dx =\frac{1}{a} \ln \cosh ax \\
        &\int \cos ax \cosh bx dx = \frac{1}{a^2 + b^2} \left[  a \sin ax \cosh bx + b \cos ax \sinh bx \right] \\
        &\int \cos ax \sinh bx dx = \frac{1}{a^2 + b^2} \left[    b \cos ax \cosh bx + a \sin ax \sinh bx  \right] \\
        &\int \sin ax \cosh bx dx = \frac{1}{a^2 + b^2} \left[ -a \cos ax \cosh bx +b \sin ax \sinh bx \right] \\
        &\int \sin ax \sinh bx dx =  \frac{1}{a^2 + b^2} \left[ b \cosh bx \sin ax - a \cos ax \sinh bx \right] \\
        &\int \sinh ax \cosh ax dx= \frac{1}{4a}\left[  -2ax + \sinh 2ax \right]\\
        &\int \sinh ax \cosh bx dx= \frac{1}{b^2-a^2}\left[  b \cosh bx \sinh ax  - a \cosh ax \sinh bx \right]\\
    \end{aligned}&
\end{flalign*}
\end{document}