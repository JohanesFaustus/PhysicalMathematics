\documentclass[../main.tex]{subfiles}
\begin{document}
Mainly consist of precalculus, and basic Calculus.
\subsection*{Algebra}
\subsubsection*{Laws of Exponents.}
\begin{align*}
    x^{\tfrac{m}{n}}&=\sqrt[n]{m}\\
    (x^m)^n&=x^{mn}\\
    x^mx^n&=x^{m+n}\\
    x^ay^a&=(xy)^a\\
\end{align*}

\subsubsection*{Special Factorization.}
\begin{align*}
    x^2-y^2&=(x+y)(x-y)\\
    x^3-y^3&=(x-y)(x^2+xy+y^2)\\
    x^3+y^3&=(x+y)(x^2-xy+y^2)\\
\end{align*}

\subsubsection*{Quadratic formula.}
\begin{align*}
    x=\dfrac{-b\pm \sqrt{b^2-4ac}}{2a}\;\;
    \begin{cases}
        D>0 \;\;\text{re(2)}\\
        D=0\;\;\text{re(1)}\\
        D<0\;\;\text{im(2)}
    \end{cases}
\end{align*}

\subsubsection*{Binomial theorem.}
\begin{align*}
    (a+b)^n=\sum_{k=0}^{\infty}\binom{n}{k}a^{n-k}b^{k}
\end{align*}
with
\begin{equation*}
    \binom{n}{k} = \frac{n!}{k!(n-k)!}=\frac{\Gamma (n+1)}{\Gamma(k+1)\Gamma(n-k+1)}
\end{equation*}

\subsection*{Trigonometry}
\subsubsection*{Trigonometry Definition.}
\begin{align*}
    \sin \theta&=\dfrac{1}{\csc \theta}\\
    \cos \theta&=\dfrac{1}{\sec \theta}\\
    \tan \theta&=\dfrac{1}{\cot \theta}\\
\end{align*}

\subsubsection*{Pythagorean Identity.}
\begin{align*}
    \sin^2 \theta+ \cos^2 \theta &=1\\
    \sec^2 \theta- \tan^2 \theta &=1\\
    \csc^2 \theta- \cot^2 \theta &=1\\
\end{align*}

\subsubsection*{Law of Sines.}
\begin{align*}
    \frac{\sin A}{a}=\frac{\sin B}{b}=\frac{\sin C}{c}
\end{align*}

\subsubsection*{Law of Cosines.}
\begin{align*}
    a^2=b^2+c^2-2bc\cos A
\end{align*}

\subsubsection*{Trigonometry Double Angle Identity.}
\begin{align*}
    \sin 2\theta&=2\sin \theta \cos \theta\\
    \cos 2 \theta&=1-2\sin^2\theta\\
    &=2\cos^2\theta -1\\
    &=\cos^2\theta-\sin^2\theta\\
    \tan 2\theta&=\frac{2\tan\theta}{1-\tan^2\theta}
\end{align*}

\subsubsection*{Trigonometry Addition and Difference Identity.}
\begin{align*}
    \sin(x+y)&=\sin x \cos y+ \cos x \sin y\\
    \sin(x-y)&=\sin x \cos y- \cos x \sin y\\
    \cos(x+y)&=\cos x \cos y- \sin x \sin y\\
    \cos(x-y)&=\cos x \cos y- \sin x \sin y\\
    \tan (x+y)&=\frac{\tan x+\tan y}{1-\tan x\tan y}\\
    \tan (x-y)&=\frac{\tan x-\tan y}{1+\tan x\tan y}\\
\end{align*}

\subsubsection*{Trigonometry Product Rule.}
\begin{align*}
    \cos x \cos y&= \frac{1}{2}[\cos(x-y)+\cos(x+y)]\\
    \sin x \sin y&= \frac{1}{2}[\cos(x-y)-\cos(x+y)]\\
    \sin x \cos y&= \frac{1}{2}[\sin(x+y)+\sin(x-y)]\\
    \cos x \sin y&= \frac{1}{2}[\sin(x+y)-\sin(x-y)]
\end{align*}

\subsubsection*{Neat Mnemonics.}
\begin{align*}
    \begin{vmatrix}
        \mathrm{S^+}\\
        \mathrm{S^-}\\
        \mathrm{C^+}\\
        \mathrm{C^-}
    \end{vmatrix}=
    \begin{vmatrix}
        \mathrm{SC+CS}\\
        \mathrm{SC-CS}\\
        \mathrm{CC-SS}\\
        \mathrm{CC+SS}
    \end{vmatrix}
    &&\mathrm{and}&&
    \begin{vmatrix}
        \mathrm{CC}\\
        \mathrm{SS}\\
        \mathrm{SC}\\
        \mathrm{CS}
    \end{vmatrix}=
    \begin{vmatrix}
        \mathrm{C^-+C^+}\\
        \mathrm{C^--C^+}\\
        \mathrm{S^++S^-}\\
        \mathrm{S^+-S^-}
    \end{vmatrix}
\end{align*}

\subsection*{Logarithm}
\subsubsection*{Definition (informal).} $\log_a b$ means $a$ to the power of what equal $b$.

\subsubsection*{Few important log rule.}
\begin{align*}
    \log_c (ab)&=\log_c (a)+\log_c (b)\\
    \log_c (\frac{a}{b})&=\log_c (a)-\log_c (b)\\
    \log_a b&=\frac{\log_c (b)}{\log_c (a)}\\
    a^{\log_a b}&=b
\end{align*}

\subsection*{Limit}
\subsubsection*{Few Important Limits.}
\begin{align*}
    \lim_{x\to a} c&=c\\
    \lim_{x\to 0^+} \frac{1}{x}&=\infty\rightarrow\frac{1}{0^+}=\infty\\
    \lim_{x\to 0^-} \frac{1}{x}&=-\infty\rightarrow\frac{1}{0^-}=-\infty
\end{align*}
\subsubsection*{Limit as Definition of Derivative.}
\begin{align*}
    \df y=\lim_{h\to 0}\frac{f(x+h)-f(x)}{h}
\end{align*}

\subsection*{Derivative}
How to determine the order of derivation: last computation is the first thing to do.

\subsubsection*{General Formula.}
\begin{align*}
    \mathrm{D}\; x^n&=nx^{n-1}\\
    \mathrm{D}\; (uv)&=\mathrm{D}\;u\cdot v+u\cdot\mathrm{D}\;v\\
    \mathrm{D}\;\bigg(\frac{u}{v}\bigg)&=\frac{\mathrm{D}\;u\cdot v-u\cdot\mathrm{D}\;v}{v^2}
\end{align*}

\subsubsection*{Trigonometry Formula.}
\begin{align*}
    \mathrm{D}\; \sin x&= \cos x\\
    \mathrm{D}\; \cos x&= -\sin x\\
    \mathrm{D}\; \tan x&= \sec^2 x\\
    \mathrm{D}\; \cot x&= -\csc^2 x\\
    \mathrm{D}\; \sec x&= \sec x\tan x\\
    \mathrm{D}\; \csc x&= -\cot x\csc x\\
\end{align*}

\subsubsection*{Neat Mnemonics.}
\begin{align*}
\begin{matrix}
    \sec&\sec&\tan&\downarrow \textrm{cofunction}\\
    \csc&-\csc&\cot&\\
    &\leftrightarrow\textrm{multiply}
\end{matrix}
\end{align*}

\subsubsection*{Exponential and Logarithmic Functions.}
\begin{align*}
    \mathrm{D}\; \ln x &= \frac{1}{x}\\
    \mathrm{D}\; a^n&=a^n\ln x\\
    \mathrm{D}\; \log_a b&=\frac{1}{b\ln a}
\end{align*}

\subsubsection*{Minima and Maxima test.} First derivative test:
\begin{itemize}
    \item Determine critical points ($\mathrm{D}y=0$), then divine into region;
    \item Pick value from each region and plug into \emph{derivative};
    \item Do the sign-graph.
    \item Determine local minima and maxima, the plug into \emph{original function}
\end{itemize}
Second derivative test:
\begin{itemize}
    \item determine critical points;
    \item plug critical into second derivative; and
    \item positive $\mathrm{D}^2y$ means concave up ($\smile$), negative means concave down ($\frown$), and 0 means inconclusive.
\end{itemize}

\subsubsection*{Differentiation under integral sign.} Differentiation under integral sign stated by Leibniz' rule
\begin{equation*}
    \frac{d}{dx}\int_{u(x)}^{v(x)}f(x,t)\;dt=\int_{u}^{v}\frac{\partial f}{\partial x}\;dt + f(x,v)\frac{dv}{dx}-f(x,u)\frac{du}{dx}
\end{equation*}

\emph{Proof.} Suppose we want $dI/dx$ where
\begin{equation*}
    I=\int_{u }^{v }f(t)\;dt
\end{equation*}
By the fundamental theorem of calculus
\begin{equation*}
    I=F(v)-F(u)=\mathcal{F}(v,u)
\end{equation*}
or $I$ is a function of $v$ and $u$. Finding $dI/dx$ is then a partial differentiation problem. We can write
\begin{equation*}
    \frac{dI}{dx}=\frac{\partial I}{\partial v}\frac{dv}{dx}+\frac{\partial I}{\partial u}\frac{du}{dx}
\end{equation*}
By the fundamental theorem of calculus, we have 
\begin{align*}
    \frac{d}{dv}\int_{a}^{v}f(x)\;dt&=\frac{d}{dv}\bigl[F(v)-F(a)\bigr]=f(v)\\
    \frac{d}{dv}\int_{u}^{b}f(x)\;dt&=\frac{d}{dv}\bigl[F(b)-F(u)\bigr]=-f(u)
\end{align*}
where $u$ and $v$ are a function of $x$, while $a$ and $b$ are a constant. This is the case when we consider $\partial I/\partial v$ or $\partial I/\partial v$; the other variable is constant. Then 
\begin{equation*}
    \frac{d}{dx}\int_{u }^{v }f(t)\;dt=f(v)\frac{dv}{dx}-f(u)\frac{du}{dx}
\end{equation*}

Under not too restrictive conditions, 
\begin{equation*}
    \frac{d}{dx}\int_{a }^{b }f(x,t)\;dt =\int_{a }^{b }\frac{\partial f(x,t)}{\partial x}\;dt
\end{equation*}
where, as before, $a$ and $b$ are constant. In other words, we can differentiate under the integral sign. It is convenient to collect these formulas into one formula known as Leibniz' rule:
\begin{equation*}
    \frac{d}{dx}\int_{u(x)}^{v(x)}f(x,t)\;dt=\int_{u}^{v}\frac{\partial f}{\partial x}\;dt + f(x,v)\frac{dv}{dx}-f(x,u)\frac{du}{dx}\quad\blacksquare
\end{equation*}

\subsubsection*{Leibniz’ rule for differentiating a product.}
\begin{equation*}
    \biggl(\frac{d}{dx}\biggr)^{n}fg=\sum_{k=0}^n {n \choose k}\biggl(\frac{d}{dx}\biggr)^{n-k} f \biggl(\frac{d}{dx}\biggr)^{k}g
\end{equation*}
where
\begin{equation*}
    {n \choose k}={\frac{n!}{k! (n-k)!}}
\end{equation*}

\subsection*{Integral}
\subsubsection*{Basic Formula (integration constant omitted).}
\begin{align*}
    \int x^n \;dx &= \frac{1}{n+1}x^{n+1}\\
    \int \frac{1}{x}\;dx& = \ln |x|\\
    \int u \;dv& = uv - \int v\; du\\
    \int a^x \;dx&=\frac{a^x}{\ln a}
\end{align*}

\subsubsection*{Trigonometry.}
\begin{align*}
    \int \sin x \; dx&=-\cos x\\
    \int \cos x \; dx&=\sin x\\
    \int \sec^2 x \; dx&=\tan x\\
    \int \csc^2 x \; dx&=-\cot x\\
    \int \sec x\tan x \; dx&=\sec x\\
    \int \csc x\tan x \; dx&=-\csc x
\end{align*}

\subsubsection*{Root.}
\begin{align*}
    \int \frac{1}{\sqrt{a^2-x^2}}\;dx&=\arcsin \frac{x}{a}\\
    \int \frac{1}{\sqrt{x^2\pm a^2}}\;dx&=\ln x +\sqrt{x^2\pm a^2}\\
    \int \frac{1}{\sqrt{a^2+x^2}}\;dx&=\frac{1}{a}\arctan \frac{x}{a}
\end{align*}

\subsubsection*{Integration by part.}
\begin{enumerate}
    \item Splits the integrand. Choose $u$ using LIATEN and let the rest be dv. (LIATEN: Log, Inverse trigonometry, Algebra, Trigonometry, ExponeN)
    \item Do the box: 
    $\begin{tabular}{ |c|c|c| } 
        \hline
        u&v&$\downarrow \textrm{diff.}$ \\ 
        \hline
        du&dv&$\uparrow\textrm{int.}$ \\ 
        \hline
       \end{tabular}$
    \item $ \int u \;dv = uv - \int v\; du$
\end{enumerate}

\subsubsection*{Tabular Method.} Refer to the table

$\begin{tabular}{ |c|c|c| } 
    \hline
    &D&I \\ 
    \hline
    +&$a\searrow$&b \\ 
    \hline
    -&$a'\searrow$&b \\ 
    \hline
    +&$a''\searrow$&b \\ \hline
    $\vdots$&$\vdots$&$\vdots$\\
    \hline
\end{tabular}$

\begin{enumerate}
    \item 0 in D collumn or use LIATEN
    \item integrate a row 
    \item a row repeats
\end{enumerate}

\subsubsection*{Trigonometry Integral.} Phytagorian Identity.
\begin{align*}
    \sin^2x+\cos^2x&=1\\
    \sin^2x&=\frac{1-\cos 2x}{2}\\
    \cos^2x&=\frac{1+\cos2x}{2}
\end{align*}
note that argument inside quadratic trigonometry is half of trigonometry, which means $\cos^22x=(1+\cos 4x)/{2}$. There are few cases of tricky trigonometry integral. First, if power of sin is odd and positive.
\begin{itemize}
    \item lop one power off
    \item convert remaining (even power) using Phytagorian Identity in term of cosine 
    \item integrate using subs method
\end{itemize}
If the power of sine is odd and positive.
\begin{itemize}
    \item same as before
\end{itemize}
If the power of sine and cosine is even and nonegative, then:
\begin{itemize}
    \item convert using Phytagorian Identity and solve
\end{itemize}

\subsubsection*{Trigonometry substitution.} Trigonometry function and its radical pair
\begin{align*}
    \tan \theta&=\sqrt{u^2+a^2}\\
    \sin \theta&=\sqrt{a^2-u^2}\\
    \sec \theta&=\sqrt{u^2-a^2}\\
\end{align*}
where $u$ is the variable we are differentiating with respect to. Mnemonics: + looks like tangent; - for sin and sec; and it is \emph{a} \emph{s}in. Trigonometry substitution step is then.
\begin{enumerate}
    \item Draw a right triangle where trigonometry pair equal $\frac{u}{a}$
    \item using the trigonometry pair equation*, solve for x and dx
    \item find trigonometry where $\frac{\sqrt{}}{a}$
    \item subs again if equation* still contain $\theta$ and solve
\end{enumerate}

\subsubsection*{Partial Fraction. }
\begin{enumerate}
    \item Factor out denominator
    \item Breakup the function and put unknown (Capital Letter) into numerator. Put numerator normally if factor is linear, put $Px+Q$ Irreducible quadratic factor IQF. In general, \begin{equation*}
        \frac{Ax^{n-1}+Bx^{n-2}+\cdots}{x^{n}+x^{n-1}+\cdots}
    \end{equation*}
    \item Multiply both side by left side's denominator
    \item Take the roots of the linear factors and plug them into x, and solve for the unknowns
    \item Put unknowns into step 2
    \item Splits Integral, then solve
    \item For equating coefficients like terms, after step 3, expand equation*. Then, collect like terms and equate coefficient of like terms from both side
\end{enumerate}
\clearpage

\subsection*{Appendix} 
\subsubsection*{Integration Technique Example.}

1. Trigonometry substitution. Find $\int \dfrac{dx}{\sqrt{9x^2+4}}$. Refer to the Mnemonics, the trigonometry pair is tangent.
\begin{align*}
    I&=\int \dfrac{dx}{\sqrt{(3x)^2+2^2}}\\
    \tan\theta&=\frac{3x}{2}
\end{align*}
solving for x and dx
\begin{align*}
    x&=\frac{2}{3}\tan\theta\\
    dx&=\frac{2}{3}\sec^2\theta \;d\theta
\end{align*}
trigonometry where $\frac{\sqrt{}}{a}$ holds is secant, solving for radical
\begin{align*}
    \sec \theta&=\frac{\sqrt{9x^2+4}}{2}\\
    \sqrt{9x^2+4}&=2\sec\theta
\end{align*}
the integral is then
\begin{align*}
    I&=\frac{1}{3}\int \sec\theta \;d\theta\\
    &=\frac{1}{3}\ln |\sec\theta+\tan\theta|+C
\end{align*}
substituting the $\theta$ function
\begin{align*}
    I&=\frac{1}{3}\ln \bigg|\frac{\sqrt{9x^2+4}}{2}+\frac{3x}{2}\bigg|+C\\
    &=\frac{1}{3} \ln \bigg | \sqrt{9x^2+4} +3 \bigg|+C
\end{align*}

\subsubsection*{Basic.}
\begin{align*}
    &\int \frac{1}{x}dx = \ln |x| \\
    &\int u dv = uv - \int v du    \\
    &\int u dv = uv - \int v du\\
    &\int \frac{1}{ax+b}dx = \frac{1}{a} \ln |ax + b| \\
\end{align*}

\subsubsection*{Integrals of Rational Functions.}
\begin{align*}
    &\int \frac{1}{(x+a)^2}dx = -\frac{1}{x+a}\\
    &\int (x+a)^n dx = \frac{(x+a)^{n+1}}{n+1}, n\ne -1\\
    &\int x(x+a)^n dx = \frac{(x+a)^{n+1} ( (n+1)x-a)}{(n+1)(n+2)}\\
    & \int \frac{1}{1+x^2}dx = \tan^{-1}x\\
    &\int \frac{1}{a^2+x^2}dx = \frac{1}{a}\tan^{-1}\frac{x}{a}\\
    &\int \frac{x}{a^2+x^2}dx = \frac{1}{2}\ln|a^2+x^2|\\
    &\int \frac{x^2}{a^2+x^2}dx = x-a\tan^{-1}\frac{x}{a}\\
    &\int \frac{x^3}{a^2+x^2}dx = \frac{1}{2}x^2-\frac{1}{2}a^2\ln|a^2+x^2|\\
    &\int \frac{1}{ax^2+bx+c}dx = \frac{2}{\sqrt{4ac-b^2}}\tan^{-1}\frac{2ax+b}{\sqrt{4ac-b^2}}\\
    &\int \frac{1}{(x+a)(x+b)}dx = \frac{1}{b-a}\ln\frac{a+x}{b+x}, \quad a\neq b\\
    &\int \frac{x}{(x+a)^2}dx = \frac{a}{a+x}+\ln |a+x|\\
    &\int \frac{x}{ax^2+bx+c}dx = \frac{1}{2a}\ln|ax^2+bx+c|-\\
    &\frac{b}{a\sqrt{4ac-b^2}}\tan^{-1}\frac{2ax+b}{\sqrt{4ac-b^2}}\\
\end{align*}
  
\subsubsection*{Integrals with Roots.}
\begin{align*}
    &\int \sqrt{x-a}\ dx = \frac{2}{3}(x-a)^{3/2}\\
    &\int \frac{1}{\sqrt{x\pm a}}\ dx = 2\sqrt{x\pm a} \\
    &\int \frac{1}{\sqrt{a-x}}\ dx = -2\sqrt{a-x} \\
    &\int x\sqrt{x-a}\ dx =  \begin{cases}\frac{2 a}{3} \left({x-a}\right)^{3/2} +\frac{2 }{5}\left( {x-a}\right)^{5/2},\text{ or} \\ \frac{2}{3} x(x-a)^{3/2} - \frac{4}{15} (x-a)^{5/2}, \text{ or}\\ \frac{2}{15}(2a+3x)(x-a)^{3/2}
    \end{cases}\\
    &\int \sqrt{ax+b}\; dx = \left(\frac{2b}{3a}+\frac{2x}{3}\right)\sqrt{ax+b} \\
    &\int (ax+b)^{3/2}\; dx =\frac{2}{5a}(ax+b)^{5/2}\\
    &\int \frac{x}{\sqrt{x\pm a} } \; dx = \frac{2}{3}(x\mp 2a)\sqrt{x\pm a}\\
    &\int \sqrt{\frac{x}{a-x}}\; dx =  -\sqrt{x(a-x)}-a\tan^{-1}\frac{\sqrt{x(a-x)}}{x-a}\\
    &\int \sqrt{\frac{x}{a+x}}\; dx =  \sqrt{x(a+x)} -a\ln  [ \sqrt{x} + \sqrt{x+a}] \\
    &\int x \sqrt{ax + b}\; dx =\frac{2}{15 a^2}(-2b^2+abx + 3 a^2 x^2)\sqrt{ax+b}\\
    &\int \sqrt{x^3(ax+b)} \ dx =\left[ \frac{b}{12a}-\frac{b^2}{8a^2x}+\frac{x}{3}\right] \sqrt{x^3(ax+b)} \\
    & + \frac{b^3}{8a^{5/2}}\ln \left| a\sqrt{x} + \sqrt{a(ax+b)} \right|\\
    &\int  \sqrt{a^2 - x^2}\ dx = \frac{1}{2} x \sqrt{a^2-x^2} +\frac{1}{2}a^2\tan^{-1}\frac{x}{\sqrt{a^2-x^2}}\\
    &\int  x \sqrt{x^2 \pm a^2}\ dx= \frac{1}{3}\left( x^2 \pm a^2 \right)^{3/2} \\
    &\int \frac{1}{\sqrt{x^2 \pm a^2}}\; dx = \ln \left| x + \sqrt{x^2 \pm a^2} \right| \\
    &\int \frac{1}{\sqrt{a^2 - x^2}}\; dx = \sin^{-1}\frac{x}{a} \\
    &\int \frac{x}{\sqrt{x^2\pm a^2}}\; dx = \sqrt{x^2 \pm a^2} \\
    &\int \frac{x}{\sqrt{a^2-x^2}}\; dx = -\sqrt{a^2-x^2} \\
    &\int \frac{x^2}{\sqrt{x^2 \pm a^2}}\; dx = \frac{1}{2}x\sqrt{x^2 \pm a^2}\mp \frac{1}{2}a^2 \ln \left| x + \sqrt{x^2\pm a^2} \right| \\
    &\int\frac{1}{\sqrt{ax^2+bx+c}}\ ;dx=\frac{1}{\sqrt{a}}\ln \left| 2ax+b + 2 \sqrt{a(ax^2+bx+c)} \right| \\
    &\int\frac{dx}{(a^2+x^2)^{3/2}}=\frac{x}{a^2\sqrt{a^2+x^2}}\\
\end{align*}

\subsubsection*{Integrals with Logarithms.}

\begin{align*}
    &\int \ln ax \;dx = x \ln (ax) - x \\
    &\int \frac{\ln ax}{x} \;dx = \frac{1}{2} ( \ln ax )^2 \\
    &\int \ln (ax + b) dx =  ( x + \frac{b}{a} ) \ln (ax+b) - x , \quad a \neq 0\\
    &\int \ln  ( x^2 + a^2 )\;{dx} = x \ln (x^2 + a^2  ) +2a\tan^{-1} \frac{x}{a} - 2x\\
    &\int \ln  ( x^2 - a^2 )\;dx= x \ln (x^2 - a^2  ) +a\ln \frac{x+a}{x-a} - 2x\\
    &\int \ln  ( x^2 - a^2 )\;dx = x \ln (x^2 - a^2  ) +a\ln \frac{x+a}{x-a} - 2x \\
    &\int \ln  ( ax^2 + bx + c) dx  = \frac{1}{a}\sqrt{4ac-b^2}\tan^{-1}\frac{2ax+b}{\sqrt{4ac-b^2}} -2x\\
    &+ ( \frac{b}{2a}+x )\ln \ (ax^2+bx+c ) \\
    &\int x \ln (ax + b) dx = \frac{bx}{2a}-\frac{1}{4}x^2+\frac{1}{2}(x^2-\frac{b^2}{a^2})\ln (ax+b) \\
    &\frac{1}{2}( x^2 - \frac{a^2}{b^2}  ) \ln  (a^2 -b^2 x^2 )\\
    &\int x \ln  ( a^2 - b^2 x^2  ) dx = -\frac{1}{2}x^2+ 
\end{align*}
    
\subsubsection*{Integrals with Exponential.}
    
    \begin{equation*}
    \int e^{ax} dx = \frac{1}{a}e^{ax} 
    \end{equation*}
    
    \begin{align*}
    \int \sqrt{x} e^{ax} dx &= \frac{1}{a}\sqrt{x}e^{ax} 
    +\frac{i\sqrt{\pi}}{2a^{3/2}}
    \text{erf}\left(i\sqrt{ax}\right)
    \end{align*}
    
    \begin{equation*}
    \int x e^x dx = (x-1) e^x 
    \end{equation*}
    
    \begin{equation*}
    \int x e^{ax} dx = \left(\frac{x}{a}-\frac{1}{a^2}\right) e^{ax} 
    \end{equation*}
    
    \begin{equation*}
    \int x^2 e^{x} dx = \left(x^2 - 2x + 2\right) e^{x}
    \end{equation*}
    
    \begin{equation*}
    \int x^2 e^{ax} dx = \left(\frac{x^2}{a}-\frac{2x}{a^2}+\frac{2}{a^3}\right) e^{ax} 
    \end{equation*}
    
    \begin{equation*}
    \int x^3 e^{x} dx = \left(x^3-3x^2 + 6x - 6\right) e^{x} 
    \end{equation*}
     
    \begin{equation*}
    \int x^n e^{ax}\hspace{1pt}\text{d}x = \dfrac{x^n e^{ax}}{a} - 
    \dfrac{n}{a}\int x^{n-1}e^{ax}\hspace{1pt}\text{d}x
    \end{equation*} 
     
    \begin{equation*}
    \begin{split}
    \int x^n e^{ax}\hspace{2pt}\text{d}x = \frac{(-1)^n}{a^{n+1}}\Gamma[1+n,-ax], \\
     \text{ where } \Gamma(a,x)=\int_x^{\infty} t^{a-1}e^{-t}\hspace{2pt}\text{d}t
     \end{split}
     \end{equation*}
    
    \begin{equation*}
    \int e^{ax^2}\hspace{1pt}\text{d}x = -\frac{i\sqrt{\pi}}{2\sqrt{a}}\text{erf}\left(ix\sqrt{a}\right) 
    \end{equation*}
    
    \begin{equation*}
    \int e^{-ax^2}\hspace{1pt}\text{d}x = \frac{\sqrt{\pi}}{2\sqrt{a}}\text{erf}\left(x\sqrt{a}\right) 
    \end{equation*}
    
    \begin{equation*}
    \int x e^{-ax^2}\ \text{dx} = -\dfrac{1}{2a}e^{-ax^2} 
    \end{equation*}
    
    \begin{equation*}
    \int x^2 e^{-ax^2}\ \text{dx} = \dfrac{1}{4}\sqrt{\dfrac{\pi}{a^3}}\text{erf}(x\sqrt{a}) -\dfrac{x}{2a}e^{-ax^2}
    \end{equation*}

\subsubsection*{Integrals with Trigonometry Functions.}
    
\begin{equation*}
    \int \sin ax dx = -\frac{1}{a} \cos ax 
    \end{equation*}
    
    \begin{equation*}
    \int \sin^2 ax dx = \frac{x}{2} - \frac{\sin 2ax} {4a} 
    \end{equation*}
    
    \begin{align*}
    \int &\sin^n ax dx =
    \nonumber \\ &
     -\frac{1}{a}{\cos ax} \hspace{2mm}{_2F_1}\left[
    \frac{1}{2}, \frac{1-n}{2}, \frac{3}{2}, \cos^2 ax
    \right] 
    \end{align*}
    
    \begin{equation*}
    \int \sin^3 ax dx = -\frac{3 \cos ax}{4a} + \frac{\cos 3ax} {12a} 
    \end{equation*}
    
    \begin{equation*}
    \int \cos ax dx= \frac{1}{a} \sin ax 
    \end{equation*}
    
    \begin{equation*}
    \int \cos^2 ax dx = \frac{x}{2}+\frac{ \sin 2ax}{4a} 
    \end{equation*}
    
    \begin{align*}
    \int \cos^p ax dx & = -\frac{1}{a(1+p)}{\cos^{1+p} ax} \times 
    \nonumber \\ &
    {_2F_1}\left[
    \frac{1+p}{2}, \frac{1}{2}, \frac{3+p}{2}, \cos^2 ax
    \right] 
    \end{align*}
    
    \begin{equation*}
    \int \cos^3 ax dx = \frac{3 \sin ax}{4a}+\frac{ \sin 3ax}{12a} 
    \end{equation*}
    
    \begin{align*}
    \int \cos ax \sin bx dx &= \frac{\cos[(a-b) x]}{2(a-b)} -
    %\nonumber \\ &
     \frac{\cos[(a+b)x]}{2(a+b)} , a\ne b
    \end{align*}
    
    \begin{align*}
    \int \sin^2 ax \cos bx dx &= 
    -\frac{\sin[(2a-b)x]}{4(2a-b)} \nonumber \\ & 
    + \frac{\sin bx}{2b} 
    - \frac{\sin[(2a+b)x]}{4(2a+b)}
    \end{align*}
    
    \begin{equation*}
    \int \sin^2 x \cos x dx = \frac{1}{3} \sin^3 x
    \end{equation*}
    
    \begin{align*}
    \int \cos^2 ax \sin bx dx &= \frac{\cos[(2a-b)x]}{4(2a-b)} 
    - \frac{\cos bx}{2b}
    \nonumber \\ &
     - \frac{\cos[(2a+b)x]}{4(2a+b)}
    \end{align*}
    
    \begin{equation*}
    \int \cos^2 ax \sin ax dx = -\frac{1}{3a}\cos^3{ax} 
    \end{equation*}
    
    \begin{align*}
    \int \sin^2 ax \cos^2 bx dx &= \frac{x}{4}
    -\frac{\sin 2ax}{8a}-
    \frac{\sin[2(a-b)x]}{16(a-b)}
    \nonumber \\ &
    +\frac{\sin 2bx}{8b}-
    \frac{\sin[2(a+b)x]}{16(a+b)}
    \end{align*}
    
    \begin{equation*}
    \int \sin^2 ax \cos^2 ax dx = \frac{x}{8}-\frac{\sin 4ax}{32a}
    \end{equation*}
    
    \begin{equation*}
    \int \tan ax dx = -\frac{1}{a} \ln \cos ax 
    \end{equation*}
    
    \begin{equation*}
    \int \tan^2 ax dx = -x + \frac{1}{a} \tan ax 
    \end{equation*}
    
    \begin{align*}
    \int &\tan^n ax dx = 
    \frac{\tan^{n+1} ax }{a(1+n)} \times \nonumber \\ &
     {_2}F_1\left( \frac{n+1}{2}, 
    1, \frac{n+3}{2}, -\tan^2 ax \right) 
    \end{align*}
    
    \begin{equation*}
    \int \tan^3 ax dx = \frac{1}{a} \ln \cos ax + \frac{1}{2a}\sec^2 ax 
    \end{equation*}
    
    \begin{align*}
    \int \sec x dx &= \ln | \sec x + \tan x | = 2 \tanh^{-1} \left (\tan \frac{x}{2} \right) 
    \end{align*}
    
    \begin{equation*}
    \int \sec^2 ax dx = \frac{1}{a} \tan ax 
    \end{equation*}
    
    \begin{equation*}
    \int \sec^3 x \hspace{2pt}\text{dx} = \frac{1}{2} \sec x \tan x + \frac{1}{2}\ln | \sec x + \tan x |
    \end{equation*}
    
    \begin{equation*}
    \int \sec x \tan x dx = \sec x 
    \end{equation*}
    
    \begin{equation*}
    \int \sec^2 x \tan x dx = \frac{1}{2} \sec^2 x 
    \end{equation*}
    
    \begin{equation*}
    \int \sec^n x \tan x dx = \frac{1}{n} \sec^n x , n\ne 0
    \end{equation*}
    
    \begin{equation*}
    \int \csc x dx = \ln \left | \tan \frac{x}{2} \right|  = \ln | \csc x - \cot x| + C
    \end{equation*}
    
    \begin{equation*}
    \int \csc^2 ax dx = -\frac{1}{a} \cot ax 
    \end{equation*}
    
    \begin{equation*}
    \int \csc^3 x dx = -\frac{1}{2}\cot x \csc x + \frac{1}{2} \ln | \csc x - \cot x | 
    \end{equation*}
    
    \begin{equation*}
    \int \csc^nx \cot x dx = -\frac{1}{n}\csc^n x, n\ne 0
    \end{equation*}
    
    \begin{equation*}
    \int \sec x \csc x dx = \ln | \tan x | 
    \end{equation*}

\subsubsection*{Products of Trigonometry Functions and Monomials.}
    
\begin{equation*}
    \int x \cos x dx = \cos x + x \sin x 
    \end{equation*}
    
    \begin{equation*}
    \int x \cos ax dx = \frac{1}{a^2} \cos ax + \frac{x}{a} \sin ax 
    \end{equation*}
    
    \begin{equation*}
    \int x^2 \cos x dx = 2 x \cos x + \left ( x^2 - 2 \right ) \sin x 
    \end{equation*}
    
    \begin{equation*}
    \int x^2 \cos ax dx = \frac{2 x \cos ax }{a^2} + \frac{ a^2 x^2 - 2  }{a^3} \sin ax 
    \end{equation*}
    
    \begin{align*}
    \int  x^n cos x dx &= 
    -\frac{1}{2}(i)^{n+1}\left [ \Gamma(n+1, -ix) 
    \right . \nonumber \\ & \left .
    + (-1)^n \Gamma(n+1, ix)\right] 
    \end{align*}
    
    \begin{align*}
    \int x^n cos ax dx &=
     \frac{1}{2}(ia)^{1-n}\left [ (-1)^n  \Gamma(n+1, -iax) 
     \right. \nonumber \\ & \left.
     -\Gamma(n+1, ixa)\right] 
    \end{align*}
    
    \begin{equation*}
    \int x \sin x dx = -x \cos x + \sin x 
    \end{equation*}
    
    \begin{equation*}
    \int x \sin ax dx = -\frac{x \cos ax}{a} + \frac{\sin ax}{a^2} 
    \end{equation*}
    
    \begin{equation*}
    \int x^2 \sin x dx = \left(2-x^2\right) \cos x + 2 x \sin x
    \end{equation*}
    
    \begin{equation*}
    \int x^2 \sin ax dx =\frac{2-a^2x^2}{a^3}\cos ax +\frac{ 2 x \sin ax}{a^2} 
    \end{equation*}
    
    \begin{align*}
    \int x^n \sin x dx &= -\frac{1}{2}(i)^n\left[ \Gamma(n+1, -ix) 
    %\right. \nonumber \\ & \left.
     - (-1)^n\Gamma(n+1, -ix)\right] 
    \end{align*}

\subsubsection*{Products of Trigonometry Functions and Exponential.}
    
\begin{equation*}
    \int e^x \sin x dx = \frac{1}{2}e^x (\sin x - \cos x) 
    \end{equation*}
    
    \begin{equation*}
    \int e^{bx} \sin ax dx = \frac{1}{a^2+b^2}e^{bx} (b\sin ax - a\cos ax) 
    \end{equation*}
    
    \begin{equation*}
    \int e^x \cos x dx = \frac{1}{2}e^x (\sin x + \cos x)  
    \end{equation*}
    
    \begin{equation*}
    \int e^{bx} \cos ax dx = \frac{1}{a^2 + b^2} e^{bx} ( a \sin ax + b \cos ax ) 
    \end{equation*}
    
    \begin{equation*}
    \int x e^x \sin x dx = \frac{1}{2}e^x (\cos x - x \cos x + x \sin x) 
    \end{equation*}
    
    \begin{equation*}
    \int x e^x \cos x dx = \frac{1}{2}e^x (x \cos x 
    - \sin x + x \sin x) 
    \end{equation*}


\subsubsection*{Integrals of Hyperbolic Functions.}
    
\begin{equation*}
    \int \cosh ax dx =\frac{1}{a} \sinh ax 
    \end{equation*}
    
    \begin{align*}
    \int e^{ax} & \cosh bx dx = \nonumber \\ &
    \begin{cases}
    \displaystyle{\frac{e^{ax}}{a^2-b^2} }[ a \cosh bx - b \sinh bx ]  & a\ne b \\
    \displaystyle{\frac{e^{2ax}}{4a} + \frac{x}{2}}  & a = b
    \end{cases}
    \end{align*}
    
    \begin{equation*}
    \int \sinh ax dx = \frac{1}{a} \cosh ax 
    \end{equation*}
    
    \begin{align*}
    \int e^{ax}& \sinh bx dx = \nonumber \\ &
    \begin{cases}
    \displaystyle{\frac{e^{ax}}{a^2-b^2} }[ -b \cosh bx + a \sinh bx ]  & a\ne b \\
    \displaystyle{\frac{e^{2ax}}{4a} - \frac{x}{2}}  & a = b
    \end{cases}
    \end{align*}
    
    \begin{align*}
    \int & e^{ax} \tanh bx dx = \nonumber \\ &
    \begin{cases}
    \displaystyle{ \frac{ e^{(a+2b)x}}{(a+2b)} 
    {_2F_1}\left[ 1+\frac{a}{2b},1,2+\frac{a}{2b}, -e^{2bx}\right] }& \\
    \displaystyle{
    \hspace{1cm}-\frac{1}{a}e^{ax}{_2F_1}\left[ \frac{a}{2b},1,1E, -e^{2bx}\right]
    }
     & a\ne b \\
    \displaystyle{\frac{e^{ax}-2\tan^{-1}[e^{ax}]}{a} } & a = b
    \end{cases}
    \end{align*}
    
    \begin{equation*}
    \int  \tanh ax\hspace{1.5pt} dx =\frac{1}{a} \ln \cosh ax 
    \end{equation*}
    
    \begin{align*}
    \int \cos ax \cosh bx dx &= 
    \frac{1}{a^2 + b^2} \left[
    a \sin ax \cosh bx  \right . \nonumber \\ & \left. + b \cos ax \sinh bx
    \right] 
    \end{align*}
    
    \begin{align*}
    \int \cos ax \sinh bx dx& = 
    \frac{1}{a^2 + b^2} \left[
    b \cos ax \cosh bx +
    \right . \nonumber \\ & \left .
     a \sin ax \sinh bx
    \right] 
    \end{align*}
    
    \begin{align*}
    \int \sin ax \cosh bx dx &= 
    \frac{1}{a^2 + b^2} \left[
    -a \cos ax \cosh bx +
    \right . \nonumber \\ & \left .
     b \sin ax \sinh bx
    \right] 
    \end{align*}
    
    \begin{align*}
    \int \sin ax \sinh bx dx &= 
    \frac{1}{a^2 + b^2} \left[
    b \cosh bx \sin ax -
    \right . \nonumber \\ & \left .
     a \cos ax \sinh bx
    \right] 
    \end{align*}
    
    \begin{equation*}
    \int \sinh ax \cosh ax dx= 
    \frac{1}{4a}\left[ 
    -2ax + \sinh 2ax \right]
    \end{equation*}
    
    \begin{align*}
    \int \sinh ax \cosh bx dx&= 
    \frac{1}{b^2-a^2}\left[ 
    b \cosh bx \sinh ax 
    \right . \nonumber \\ & \left .
    - a \cosh ax \sinh bx \right]
    \end{align*}
\end{document}