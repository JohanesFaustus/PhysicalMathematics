\documentclass[../main.tex]{subfiles}
\begin{document}
\subsection*{Application: Shortest Between two points}
Arbitrary path is given by 
\begin{equation*}
    L=\int_{1}^{2}\sqrt{dx^2+dy^2}=\int_{x_1}^{x_2}\sqrt{1+y'^2}\;dx
\end{equation*}
We factor $dx$ from the integrand in order to make the function we are optimizing not dependent on the $y$ variable and make the evaluation using Euler-Lagrange equation easier
\begin{equation*}
    f(y,y',x)=\sqrt{1+y'^2}
\end{equation*}
Then the Euler-Lagrange equation takes the form of 
\begin{equation*}
    \frac{\partial f}{\partial y}=\frac{d}{dt}\frac{\partial f}{\partial y'}
\end{equation*}
$\partial f/\partial y=0$ implies simply that $\partial f/\partial y'$ is a constant. Accordingly, 
\begin{align*}
    \frac{\partial f}{\partial y'}&=\frac{y'}{\sqrt{1+y'^2}}=C\\
    y'^2&=C^2(1+y^2)\\
    y'^2(1-C^2)&=C^2\\
    y&=\int\frac{C}{\sqrt{1-C^2}}\;dx=\mathcal{C}x
\end{align*}
which is the equation for straight line.

\subsubsection*{Application: Brachistochrone}
Given two points 1 and 2, with 1 higher above the ground, in what shape should we build a frictionless roller coaster track so that a car released from point 1 will reach point 2 in the shortest possible time? 

The speed at which the coaster descend can be determined by the conservation energy principle 
\begin{align*}
    mgy&=\frac{1}{2}mv^2
    v&=\sqrt{2gy}
\end{align*}
Thus the time to travel between points 
\begin{equation*}
    t=\int_{t_1}^{t_2}\frac{ds}{v}=\int_{t_1}^{t_2}\sqrt{\frac{dx^2+dy^2}{2gh}}
\end{equation*}
Since $v$ gives a function of $y$, we take it as independent variable for the same reason as previously 
\begin{align*}
    t=\frac{1}{\sqrt{2g}}\int_{t_1}^{t_2}\sqrt{\frac{1+x'^2}{y}}\;dy
\end{align*}
Ignoring the constant, the function we want to optimize is 
\begin{equation*}
    f(x,x',y)=\sqrt{\frac{1+x'^2}{y}}
\end{equation*}
Then the Euler-Lagrange equation takes the form of 
\begin{equation*}
    \frac{\partial f}{\partial x}=\frac{d}{dy}\frac{\partial f}{\partial x'}
\end{equation*}
$\partial f/\partial x=0$ implies simply that $\partial f/\partial x'$ is a constant. Accordingly, 
\begin{equation*}
    \frac{\partial f}{\partial x'}=\frac{x'}{\sqrt{y(1+x'^2)}}=C
\end{equation*}
Here we take the constant as $\sqrt{1/2a}$
\begin{align*}
    x'^2&=\frac{y(1+x'^2)}{2a}\\
    x'^2\left(1-\frac{y}{2a}\right)&=\frac{y}{2a}\\
    x'&=\sqrt{\frac{y}{2a}\frac{1}{2-y/2a}}\\
    x'&=\sqrt{\frac{y}{2a-y}}\\
    x=\int \sqrt{\frac{y}{2a-y}}\;dy
\end{align*}
To solve this integral, we substitute $y=a(1-\cos\alpha)$ and $dy=a\sin\alpha\;d\theta$
\begin{align*}
    x&=\int\left[\frac{a(1-\cos\theta)}{a(1+\cos\theta)}\right]^{1/2}a\sin\theta\;s\theta\\
    &=a\int \left[\frac{1-\cos\theta}{1+\cos\theta}\right]^{1/2}\left[(1+\cos\theta)(1-\cos\theta)\right]^{1/2}\;d\theta\\
    &=a\int(1-\cos\theta)\;d\theta\\
    x&=a(\theta-\sin\theta)+c
\end{align*}
Therefore the path of the coaster is given by the following parametric equation
\begin{equation*}
    \begin{cases}
        x=a(\theta-\sin\theta)+c\\
        y=a(1-\cos\theta)
    \end{cases}
\end{equation*}



\end{document}