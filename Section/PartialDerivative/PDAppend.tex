\documentclass[../../Main.tex]{subfiles}
\begin{document}
\subsection{Appendix: Lagrange Multipliers}
\subsubsection{Single constraint.} Consider this example.
\begin{quotation}
    Determine the largest volume of parallelepiped--that is, a three-dimensional figure formed by six parallelograms--whose edges parallel with the $x,y,z$ axis inside ellipsoid
    \begin{equation*}
        \frac{x^2}{a^2}+\frac{y^2}{b^2}+\frac{z^2}{c^2}=1
    \end{equation*}
\end{quotation}
The ellipsoid function above acts as constraints, that left is to determine the function that we want to optimize. This requires some clever thinking. We begin by defining point $(x,y,z)$ be the corner of our parallelepiped. Now, this point is located in the first octant of our parallelepiped. The volume of this octant is 
\begin{equation*}
    v=xyz
\end{equation*}
Since the parallelepiped's sides are parallel the axis, its total volume is 
\begin{equation*}
    V=8v
\end{equation*}
Hence, the volume of our parallelepiped is 
\begin{equation*}
    V=8xyz
\end{equation*}
This is the function that we want to maximize. We then construct the function
\begin{equation*}
    F(x,y,z)=8xyz+\lambda\left(\frac{x^2}{a^2}+\frac{y^2}{b^2}+\frac{z^2}{c^2}\right)
\end{equation*}
The partial derivatives of $F$ read as 
\begin{align*}
    \frac{\partial F}{\partial x}=8yz+\frac{2\lambda}{a^2}x,\quad
    \frac{\partial F}{\partial y}=8xz+\frac{2\lambda}{b^2}y,\quad
    \frac{\partial F}{\partial z}=8xy+\frac{2\lambda}{c^2}z
\end{align*}
To find the maximum of $F$, we then must solve the partial derivative equations and constraint equation 
\begin{align*}
    8yz+\dfrac{2\lambda}{a^2}x&=0\\
    8xz+\dfrac{2\lambda}{b^2}y&=0\\
    8xy+\frac{2\lambda}{c^2}z&=0\\
    \dfrac{x^2}{a^2}+\dfrac{y^2}{b^2}+\dfrac{z^2}{c^2}&=1
\end{align*}
Multiplying the first equation by $x$, the second by $y$, the third by $z$ and adding them all together, we get 
\begin{equation*}
    24xyz+2\lambda\left(\dfrac{x^2}{a^2}+\dfrac{y^2}{b^2}+\dfrac{z^2}{c^2}\right) = 24xyz+2\lambda=0
\end{equation*}
Hence
\begin{equation*}
    \lambda=-12xyz
\end{equation*}
Substituting this into the partial derivative equation to obtain
\begin{align*}
    8yz-\dfrac{24yz}{a^2}x^2&=0\implies x=\frac{\sqrt{3}}{3}a\\
    8xz-\dfrac{24xz}{b^2}y^2&=0\implies y=\frac{\sqrt{3}}{3}b\\
    8xy-\frac{24xy}{c^2}z^2&=0\implies z=\frac{\sqrt{3}}{3}c
\end{align*}
Therefore, the maximum volume of said parallelepiped is 
\begin{equation*}
    V=\frac{24\sqrt{3}}{27}abc
\end{equation*}

\subsubsection{Two constraints.} Here's an example.
\begin{quotation}
    Given two equation $z^2=x^2+y^2$ and $x+2z+3=0$, find the shortest and longest distance from the origin and the intersection of those two equations.
\end{quotation}
Here we want to minimize $f=x^2+y^2+z^2$ as usual. We construct auxiliary function
\begin{equation*}
    F=x^2+y^2+z^2+\lambda_1(z^2-x^2-y^2)+\lambda_2(x+2z)
\end{equation*}
The partial differentials of $F$ read
\begin{align*}
    \frac{\partial F}{\partial x}&=2x-2\lambda_1x+\lambda_2,\\
    \frac{\partial F}{\partial y}&=2y-2\lambda_1y,\\
    \frac{\partial F}{\partial z}&=2z+2\lambda_1z+2\lambda_2
\end{align*}
Putting it all together, we have these equations
\begin{align}
    2x-2\lambda_1x+\lambda_2&=0\label{eq:PDA1}\\
    2y-2\lambda_1y&=0\label{eq:PDA2}\\
    2z+2\lambda_1z+2\lambda_2&=0\label{eq:PDA3}\\
    z^2-x^2-y^2&=0\label{eq:PDA4}\\
    x+2z+3&=0\label{eq:PDA5}
\end{align}
By equation \ref{eq:PDA2}, we have two possible cases
\begin{equation*}
    2y-2\lambda_1y=y(1-\lambda_1)=0\implies y=0\;\lor\;\lambda_1=1
\end{equation*}
First we consider $y=0$. Equation \ref{eq:PDA4} reads
\begin{equation*}
    z^2=x^2\implies z=\pm x
\end{equation*}
Then in the subcase $y=0,\;z=x$; equation \ref{eq:PDA5} evaluates into
\begin{equation*}
    3x+3=0\implies x=3
\end{equation*}
In other hand, for subcase $y=0,\;z=-x$; the same equation evaluates into
\begin{equation*}
    x=3
\end{equation*}
Now we consider the case when $\lambda_1=1$. Equation \ref{eq:PDA1} reduces into 
\begin{equation*}
    \lambda_2=0
\end{equation*}
which means equation \ref{eq:PDA5} turns into
\begin{equation*}
    4z=0 \implies z=0
\end{equation*}
and equation \ref{eq:PDA5}
\begin{equation*}
    x=-3
\end{equation*}
Using this result, equation \ref{eq:PDA4} reads 
\begin{equation*}
    y^2=-9
\end{equation*}
which is impossible unless we are willing to take a complex value. Suppose we are willing, we have the $y=3i$. Hence, we have three possibilities that the optimized points might take
\begin{equation*}
    \{\mathbf{P_1},\mathbf{P_2},\mathbf{P_3}\}=\{(-1,0,-1),(3,0,-3),(-3,3i,0)\}
\end{equation*}
The distance from origin then evaluated by 
\begin{align*}
    d_1&=\sqrt{\mathbf{P_1}\cdot\mathbf{P_2}}=\sqrt{2}\\
    d_2&=\sqrt{\mathbf{P_2}\cdot\mathbf{P_2}}=\sqrt{18}\\
    d_3&=\sqrt{\mathbf{P_3}\cdot\mathbf{\overline{P_3}}}=\sqrt{18}
\end{align*}
Hence the shortest distance is $d=\sqrt{2}$ and the longest is $d=\sqrt{18}$.
\end{document}