\documentclass[../../../main.tex]{subfiles}
\begin{document}
\subsection{Appendix: Limit Evaluation}
\subsubsection{1. Consider}
\begin{equation*}
    \lim_{x \to 0} \frac{1 - \cos(3x)}{x^{2}}
\end{equation*}
Evaluation
\begin{align*}
    \lim_{x \to 0} \frac{1 - \cos(3x)}{x^{2}}
     & = \lim_{x \to 0} \frac{2\sin^{2}\!\left(\tfrac{3x}{2}\right)}{x^{2}}                                                     \\
     & = 2 \lim_{x \to 0} \left(\frac{\sin\!\left(\tfrac{3x}{2}\right)}{x}\right)^{2}                                           \\
     & = 2 \lim_{x \to 0} \left[\frac{\sin\!\left(\tfrac{3x}{2}\right)}{\tfrac{3x}{2}} \cdot \tfrac{3}{2}\right]^{2}            \\
     & = 2 \left(\tfrac{3}{2}\right)^{2} \lim_{x \to 0} \left(\frac{\sin\!\left(\tfrac{3x}{2}\right)}{\tfrac{3x}{2}}\right)^{2} \\
     & = 2 \left(\tfrac{3}{2}\right)^{2} (1)^{2}                                                                                \\
     & = \frac{9}{2}.
\end{align*}

\subsubsection{2. Consider}
\begin{equation*}
    \lim_{x \to 0} \frac{1 - \cos(4x)}{x^{2}}
\end{equation*}
Evaluation
\begin{align*}
    \lim_{x \to 0} \frac{1 - \cos(4x)}{x^{2}}
     & = \lim_{x \to 0} \frac{2\sin^{2}(2x)}{x^{2}}                    \\
     & = 2 \lim_{x \to 0} \left(\frac{\sin(2x)}{x}\right)^{2}          \\
     & = 2 \left[\lim_{x \to 0} \frac{\sin(2x)}{2x} \cdot 2\right]^{2} \\
     & = 2 \left(2 \cdot 1\right)^{2}                                  \\
     & = 8.
\end{align*}

\subsubsection{3. Consider}
\begin{equation*}
    \lim_{x \to 1} \frac{16 (x^2-1)}{x-1}
\end{equation*}
Evaluation
\begin{align*}
    \lim_{x \to 1} \frac{16(x^{2} - 1)}{x - 1}
     & = \lim_{x \to 1} \frac{16(x - 1)(x + 1)}{x - 1} \\
     & = \lim_{x \to 1} 16(x + 1)                      \\
     & = 16(2)
     & = 32
\end{align*}
\subsubsection{4. Consider}
\begin{equation*}
    \lim_{x \to \infty} x \sin\!\left(\frac{1}{x}\right)
\end{equation*}
Evaluation
\begin{align*}
    \lim_{x \to \infty} x \sin\!\left(\frac{1}{x}\right)
     & = \lim_{x \to 0} \frac{\sin t}{t} \\
     & = 1
\end{align*}
where we have substituted $t=1/x$.

\subsubsection{5. Consider}
\begin{equation*}
    \lim_{x\to5} \frac{x^2-25 }{x-5}
\end{equation*}
Evaluation
\begin{align*}
    \lim_{x \to 5} \frac{x^{2} - 25}{x - 5}
     & = \lim_{x \to 5} \frac{(x - 5)(x + 5)}{x - 5} \\
     & = \lim_{x \to 5} (x + 5)                      \\
     & = 10
\end{align*}

\subsubsection{6. Consider}
\begin{equation*}
    \lim_{x\to4}\frac{\sqrt{x}-2 }{x-4}
\end{equation*}
Evaluation
\begin{align*}
    \lim_{x\to4}\frac{\sqrt{x}-2 }{x-4} & = \lim_{x\to4}\frac{\sqrt{x} - 2}{x - 4} \cdot \frac{\sqrt{x} + 2}{\sqrt{x} + 2} \\
                                        & =\lim_{x\to4} \frac{x - 4}{(x - 4)(\sqrt{x} + 2)}                                \\
                                        & =\lim_{x\to4}\frac{1}{\sqrt{x} + 2}.                                             \\
                                        & = \frac{1 }{4}
\end{align*}

\subsubsection{7. Consider}
\begin{equation*}
    \lim_{x\to0}\frac{\frac{1 }{x+4 }-\frac{1 }{4}  }{x}
\end{equation*}
Evaluation
\begin{align*}
    \lim_{x \to 0} \frac{\tfrac{1}{x - 4} - \tfrac{1}{4}}{x}
     & =\lim_{x \to 0}\frac{\frac{1 }{x-4 }-\frac{1 }{4}  }{x} \frac{4(x+4 )}{4(x+4)} \\
     & = \lim_{x \to 0}\frac{4-(x+4 )}{4x(x+4)}                                       \\
     & = \lim_{x \to 0} \frac{-x }{4x(x+4)}                                           \\
     & = \lim_{x \to 0} \frac{-1 }{4(x+4)}                                            \\
     & = \frac{-1 }{4(0+4)}=-\frac{1 }{16}
\end{align*}

\subsubsection{8. Consider}
\begin{equation*}
    \lim_{x\to\infty}\sqrt{x^2+x}-x
\end{equation*}
Evaluation
\begin{align*}
    \lim_{x \to \infty} (\sqrt{x^{2} + x} - x)
     & = \lim_{x \to \infty} \frac{(\sqrt{x^{2} + x} - x)(\sqrt{x^{2} + x} + x)}{\sqrt{x^{2} + x} + x} \\
     & = \lim_{x \to \infty} \frac{x}{\sqrt{x^{2} + x} + x}                                            \\
     & = \lim_{x \to \infty} \frac{1}{\sqrt{1 + \tfrac{1}{x}} + 1}                                     \\
     & = \frac{1}{2}
\end{align*}

\subsection{Appendix: Derivative Evaluation}
\subsubsection{1. Limit definition.}
Let \( f(x) = x^3 + 2x^2 - 5x + 7 \).
The derivative is defined as
\[
    f'(x) = \lim_{h \to 0} \frac{f(x + h) - f(x)}{h}.
\]
Substitute the function:
\[
    f'(x) = \lim_{h \to 0} \frac{(x + h)^3 + 2(x + h)^2 - 5(x + h) + 7 - (x^3 + 2x^2 - 5x + 7)}{h}.
\]
Expand each term:
\[
    (x + h)^3 = x^3 + 3x^2h + 3xh^2 + h^3,
\]
\[
    2(x + h)^2 = 2(x^2 + 2xh + h^2) = 2x^2 + 4xh + 2h^2,
\]
\[
    -5(x + h) = -5x - 5h.
\]
Substitute back:
\begin{multline*}
    f'(x) = \lim_{h \to 0} \bigg[]\frac{(x^3 + 3x^2h + 3xh^2 + h^3 + 2x^2 + 4xh + 2h^2 - 5x - 5h + 7) }{h} - \\ \frac{- (x^3 + 2x^2 - 5x + 7) }{h}\bigg]
\end{multline*}
Cancel identical terms \( x^3, 2x^2, -5x, 7 \):
\[
    f'(x) = \lim_{h \to 0} \frac{3x^2h + 3xh^2 + h^3 + 4xh + 2h^2 - 5h}{h}.
\]
Factor \( h \):
\[
    f'(x) = \lim_{h \to 0} \frac{h(3x^2 + 3xh + h^2 + 4x + 2h - 5)}{h}.
\]
Simplify:
\[
    f'(x) = \lim_{h \to 0} (3x^2 + 3xh + h^2 + 4x + 2h - 5).
\]
Take the limit \( h \to 0 \):
\[
    f'(x) = 3x^2 + 4x - 5.
\]

With the power method
\begin{align*}
    D(x^3)  & = 3x^{2},              \\
    D(2x^2) & = 2 \cdot 2x^{1} = 4x, \\
    D(-5x)  & = -5,                  \\
    D(7)    & = 0.
\end{align*}
Combine all terms:
\[
    f'(x) = 3x^2 + 4x - 5.
\]

\subsubsection{2. Power rule.}
Given the function
\[
    f(x) = x^{\frac{5}{2}} + 3x^{\frac{3}{2}} - 4x^{\frac{1}{2}} + 2,
\]
apply the power rule
\begin{align*}
    D\left(x^{\frac{5}{2}}\right)   & = \frac{5}{2}x^{\frac{3}{2}},                                      \\
    D\left(3x^{\frac{3}{2}}\right)  & = 3 \cdot \frac{3}{2}x^{\frac{1}{2}} = \frac{9}{2}x^{\frac{1}{2}}, \\
    D\left(-4x^{\frac{1}{2}}\right) & = -4 \cdot \frac{1}{2}x^{-\frac{1}{2}} = -2x^{-\frac{1}{2}},       \\
    D(2)                            & = 0.
\end{align*}
Combine all terms:
\[
    f'(x) = \frac{5}{2}x^{\frac{3}{2}} + \frac{9}{2}x^{\frac{1}{2}} - 2x^{-\frac{1}{2}}.
\]

\subsubsection{3. Product rule.}
Let
\[
    f(x) = x^2 \sin(x).
\]
The product rule states
\[
    D(uv) = u'v + uv',
\]
where \( u = x^2 \) and \( v = \sin(x) \).
Compute each derivative:
\[
    u' = 2x, \qquad v' = \cos(x).
\]
Substitute into the product rule:
\begin{equation*}
    f'(x) = (2x)\sin(x) + (x^2)\cos(x)=2x\sin(x) + x^2\cos(x).
\end{equation*}


\subsection{Appendix: Example of Optimization Problem}
\subsubsection{One variable problem.} Consider this simple problem.
\begin{quotation}
    What is the maximum volume of a box without top that can be made from a square plate with a side of 30 unit, given that you can cut of its corner?
\end{quotation}
\begin{figure*}[b]
    \centering
    \normfig{../../../Rss/AnalyticsApproach/Fundamentals/OptimizationBox.png}
    \caption*{Figure: Plate and its configuration}
\end{figure*}

We know the volume of the box is calculated by $V=lwh$, then the volume as function of height is
\begin{equation*}
    V(h)=(30-2h)(30-2h)h=4h^3-120h^2+900
\end{equation*}
This is the function that we want to maximize. Setting the derivative to zero, we have
\begin{equation*}
    \begin{array}{l l}
        \dfrac{dV}{dh} & =12h^2-240h+900 \\
        0              & =h^2-20h+75
    \end{array}\implies
    h=15\;\lor\; h=5
\end{equation*}
Putting those critical number, into $V(h)$, we obtain $V(5)=2000$ and $V(15)=0$. Hence, box with height 5 will maximize the volume.

\subsubsection{Two variable problem.} More complicated, but still very easy.
\begin{quotation}
    Consider a floorless pup tent made from the least possible material with width $2w$, length $l$, creating angle $\theta$. Find $\theta$.
\end{quotation}

First we consider the volume and area of the shape in question.
\begin{align*}
    V & =\frac{2w}{2}w\tan \theta\cdot l=w^2l\tan\theta                                               \\
    A & =\frac{2w}{2}w\tan \theta\cdot 2+\frac{wl}{\cos \theta}\cdot 2= 2w^2\tan\theta+2wl\sec \theta
\end{align*}
Solving for $l$ from the equation of $V$, we get $l=V/w^2\tan \theta$. Then we substitute the result into $A$ to obtain
\begin{equation*}
    A=2w^2\tan\theta+\frac{2wV\sec \theta}{w^2\tan\theta}= 2w^2\tan\theta+\frac{2V}{w}\csc\theta
\end{equation*}
This is the function that we want to minimize. Since $A$ is a function of two variable $A(w,\theta)$, we need to differentiate it with respect to those two variable
\begin{align*}
    \frac{\partial A}{\partial w}      & =4w\tan \theta-\frac{2V}{w^2}\csc \theta=0           \\
    \frac{\partial A}{\partial \theta} & =2w^2\sec^2\theta-\frac{2V}{w}\csc\theta\cot\theta=0
\end{align*}
Solving those equations for $w^3$, we get
\begin{align*}
    w^3=\frac{V\csc\theta}{2\tan\theta}\;\land\; w^3=\frac{V\csc\theta\cot\theta}{\sec^2\theta}
\end{align*}
Equating them to obtain
\begin{equation*}
    \frac{1}{\sec^2\theta}=\frac{1}{2}\implies \theta=\frac{\pi}{4}=45^\circ
\end{equation*}

\subsection{Appendix: Example of Optimization Problem with Constrain.}
We try to solve this example using few methods: elimination, implicit derivative, and Lagrange multiplier. Now, consider this problem.
\begin{quotation}
    What is the shortest distance from origin to curve $y=1-x^2$?
\end{quotation}

\subsubsection{Elimination.} What we want to minimize is the distance $d=({x^2+y^2})^{1/2}$, however it is more convenient to minimize $f=x^2+y^2$ instead. The function $y=1-x^2$ acts as constraint. Substituting the constraint into our function, we have
\begin{equation*}
    f(x)=x^2+(1-x^2)^2=x^4-x^2+1
\end{equation*}
The critical points of this function determined by
\begin{equation*}
    \frac{df}{dx}=4x^3-2x=x(4x^2-2)=0\implies x=0\lor x=\pm\sqrt{1/2}
\end{equation*}
Then to determine the maxima or minima of the function, we use second test derivative
\begin{equation*}
    \frac{d^2f}{dx^2}=12x^2-2=\begin{cases}
        -2, & x=0, \quad\text{Maxima}              \\
        4,  & x=\pm \sqrt{1/2},\quad \text{Minima} \\
    \end{cases}
\end{equation*}
Therefore, the minimum distance is
\begin{equation*}
    d=\left[\left(\sqrt{\frac{1}{2}}\right)^2+\left(1-\frac{1}{2}\right)^2\right]^{1/2}=\frac{\sqrt{3}}{2}
\end{equation*}

\subsubsection{Implicit differentiation.} We use this method if it is not possible to substitute the constraint $y(x)$ into our function $f$. Differentiating $f(x,y)x^2+y^2$ with respect to $x$
\begin{equation*}
    \frac{df}{dx}=2x+2y\frac{dy}{dx}
\end{equation*}
From our constraint equation, we have the relation of $dy$ in terms of $dx$ as $dy=-2x\;dx$. Substituting this equation into $df/dx$ while also setting it equal to zero
\begin{equation*}
    2x-4xy=x(1-2y)=0\implies x=0\lor y=1/2
\end{equation*}
And we obtain one of the critical points. To obtain the rest, we substitute the result $y=1$ into our constraint equation $y=1-x^2$. We have then
\begin{equation*}
    x=\left(-\sqrt{\frac{1}{2}},0,\sqrt{\frac{1}{2}}\right)
\end{equation*}
The second derivative test for this method is rather different from the usual. What differs is simply how to evaluate the second derivative. First we determine the second derivative of $f(x,y)$ with respect to $x$
\begin{equation*}
    \frac{d^2f}{dx^2}=\frac{d}{dx}\left(2x+2y\frac{dy}{dx}\right)=2+\left(\frac{dy}{dx}\right)^2 +2y\frac{d^2y}{dx^2}
\end{equation*}
At $x=0$, we have $y=1,\;dy/dx=0,\text{ and } d^2y/dy^2=-2$; while at $x=\pm\sqrt{1/2}$, we have $y=1/2,\;dy/dx=\mp\sqrt{2},\text{ and } d^2y/dy^2=-2$. Hence,
\begin{equation*}
    \frac{d^2f}{dx^2}\bigg|_{x=0}=-2\text{ (Maxima)}\quad\land\quad \frac{d^2f}{dx^2}\bigg|_{x=\pm\sqrt{1/2}}=-4\text{ (Minima)}
\end{equation*}
as before. Substituting the result into the equation for distance $d=(x^2+y^2)^{1/2}$ and we have the same result

\subsubsection{Lagrange multiplier.} We have the function that we want to maximize $f(x,y)=x^2+y^2$ and constrain $\phi(x,y)=x^2+y=1$. Then we construct the function
\begin{equation*}
    F(x,y)=x^2+y^2+\lambda(x^2+y)
\end{equation*}
The partial derivative of $F(x,y)$ with respect to each variable is
\begin{equation*}
    \frac{\partial F}{\partial x}=2x+2\lambda x\quad\land\quad \frac{\partial F}{\partial y}=2y+\lambda
\end{equation*}
In addition with the constraint, we then need to solve those three equations
\begin{align*}
    2x+2\lambda x & =0 \\
    2y+\lambda    & =0 \\
    x^2+y         & =1
\end{align*}
Solving the first equation
\begin{equation*}
    x(1+\lambda)=0\implies x=0\lor\lambda=-1
\end{equation*}
Using $\lambda $ on the second equation
\begin{equation*}
    2y+\frac{1}{2} =0\implies y=\frac{1}{2}
\end{equation*}
Hence the constraint equation reads
\begin{equation*}
    x^2+\frac{1}{2}=1\implies x=\pm\sqrt{1/2}
\end{equation*}
As before, we obtain critical points $x=\left(-\sqrt{1/2},0,\sqrt{1/2}\right)$. We then can move to the next step:  minima test and calculating the distance, both will need not to be repeated.

\subsection{Appendix: Integration Technique Example.}
\subsubsection{Integration by parts.}
Consider the integral
\[
    \int x e^{x}\,dx.
\]
Set \(u=x\) and \(dv=e^{x}\,dx\). Then \(du=dx\) and \(v=e^{x}\). Substituting into the integration–by–parts identity produces
\[
    \int x e^{x}\,dx = x e^{x} - \int e^{x}\,dx.
\]
The remaining integral is elementary, yielding
\[
    \int x e^{x}\,dx = x e^{x} - e^{x} + C
    = e^{x}(x-1)+C.
\]

\subsubsection{Integration by parts 2.}
A classical application of integration by parts yields the fundamental recurrence of the Gamma function. The Gamma function is defined for \(s>0\) by
\[
    \Gamma(s)=\int_{0}^{\infty} x^{\,s-1} e^{-x}\,dx = (s + 1)!
\]

Apply integration by parts to this integral. Set
\[
    u = x^{\,s-1}, \qquad dv = e^{-x}\,dx,
\]
so that
\[
    du = (s-1)x^{\,s-2}\,dx, \qquad v = -e^{-x}
\]
Substitute into the identity \(\int u\,dv = uv - \int v\,du\). This yields
\[
    \Gamma(s)=\left[ -x^{\,s-1} e^{-x} \right]_{0}^{\infty}
    +\int_{0}^{\infty} e^{-x}(s-1)x^{\,s-2}\,dx
\]

The boundary term vanishes. Indeed, as \(x\to\infty\), the exponential suppresses the polynomial, giving \(x^{\,s-1}e^{-x}\to 0\); as \(x\to 0^{+}\), one has \(x^{\,s-1}\to 0\) for \(s>0\). Hence
\[
    \Gamma(s) = (s-1)\int_{0}^{\infty} x^{\,s-2} e^{-x}\,dx
\]

The remaining integral is the definition of \(\Gamma(s-1)\). Thus one obtains the recurrence relation
\[
    \Gamma(s)= (s-1)\Gamma(s-1)
\]



\subsubsection{Trigonometry substitution.} Find $\int \dfrac{dx}{\sqrt{9x^2+4}}$. Refer to the Mnemonics, the trigonometry pair is tangent.
\begin{align*}
    I          & =\int \dfrac{dx}{\sqrt{(3x)^2+2^2}} \\
    \tan\theta & =\frac{3x}{2}
\end{align*}
solving for x and dx
\begin{align*}
    x  & =\frac{2}{3}\tan\theta             \\
    dx & =\frac{2}{3}\sec^2\theta \;d\theta
\end{align*}
trigonometry where $\frac{\sqrt{}}{a}$ holds is secant, solving for radical
\begin{align*}
    \sec \theta   & =\frac{\sqrt{9x^2+4}}{2} \\
    \sqrt{9x^2+4} & =2\sec\theta
\end{align*}
the integral is then
\begin{align*}
    I & =\frac{1}{3}\int \sec\theta \;d\theta     \\
      & =\frac{1}{3}\ln |\sec\theta+\tan\theta|+C
\end{align*}
substituting the $\theta$ function
\begin{align*}
    I & =\frac{1}{3}\ln \bigg|\frac{\sqrt{9x^2+4}}{2}+\frac{3x}{2}\bigg|+C \\
      & =\frac{1}{3} \ln \bigg | \sqrt{9x^2+4} +3 \bigg|+C
\end{align*}

\subsubsection{Unit circle.}
The curve is given by \(y=\pm\sqrt{1-x^{2}}\).
The area \(A\) enclosed by the upper and lower branches over the interval \([-1,1]\) is
\[
    A
    =2\int_{-1}^{1}\sqrt{1-x^{2}}\,dx.
\]

Use the substitution \(x=\sin\theta\). Then \(dx=\cos\theta\,d\theta\) and \(\sqrt{1-x^{2}}=\cos\theta\).
The limits transform as \(x=-1\mapsto\theta=-\tfrac{\pi}{2}\) and \(x=1\mapsto\theta=\tfrac{\pi}{2}\).
Hence
\[
    A=2\int_{-\pi/2}^{\pi/2}\cos\theta\cdot\cos\theta\,d\theta
    =2\int_{-\pi/2}^{\pi/2}\cos^{2}\theta\,d\theta.
\]

Apply the double-angle identity \(\cos^{2}\theta=\tfrac{1+\cos(2\theta)}{2}\):
\[
    A=2\int_{-\pi/2}^{\pi/2}\frac{1+\cos(2\theta)}{2}\,d\theta
    =\int_{-\pi/2}^{\pi/2}\bigl(1+\cos(2\theta)\bigr)\,d\theta.
\]

Evaluate the integral:
\[
    A=\left[\theta+\tfrac{1}{2}\sin(2\theta)\right]_{-\pi/2}^{\pi/2}
    =\bigl(\tfrac{\pi}{2}+0\bigr)-\bigl(-\tfrac{\pi}{2}+0\bigr)
    =\pi.
\]

Therefore the area enclosed by \(y=\pm\sqrt{1-x^{2}}\) for \(x\in[-1,1]\) is \(\pi\), which is the area of the unit circle.

\subsubsection{Gabriel's Horn.}
Consider the classical construction known as Gabriel's horn. It is obtained by revolving the curve
\[
    y=\frac{1}{x}, \qquad x\ge 1,
\]
about the \(x\)-axis. Its surface area and volume follow from standard formulas for surfaces of revolution.

The volume \(V\) of the solid generated by revolving \(y=1/x\) on the interval \([1,b]\) around the \(x\)-axis is
\[
    V(b)=\pi\int_{1}^{b}\left(\frac{1}{x}\right)^{2}\,dx
    =\pi\int_{1}^{b}\frac{1}{x^{2}}\,dx
    =\pi\left[-\frac{1}{x}\right]_{1}^{b}
    =\pi\left(1-\frac{1}{b}\right).
\]
Letting \(b\to\infty\) yields a finite limit,
\[
    \lim_{b\to\infty}V(b)=\pi.
\]

The surface area \(S\) of the same solid on \([1,b]\) is given by
\[
    S(b)=2\pi\int_{1}^{b}y\sqrt{1+\left(\frac{dy}{dx}\right)^{2}}\,dx
    =2\pi\int_{1}^{b}\frac{1}{x}\sqrt{1+\left(-\frac{1}{x^{2}}\right)^{2}}\,dx.
\]
Since \((dy/dx)=-1/x^{2}\), the integrand becomes
\[
    \frac{1}{x}\sqrt{1+\frac{1}{x^{4}}}
    =\frac{1}{x}\sqrt{\frac{x^{4}+1}{x^{4}}}
    =\frac{\sqrt{x^{4}+1}}{x^{3}}.
\]
Hence
\[
    S(b)=2\pi\int_{1}^{b}\frac{\sqrt{x^{4}+1}}{x^{3}}\,dx.
\]

For large \(x\), \(\sqrt{x^{4}+1}\sim x^{2}\), thus the integrand behaves asymptotically like
\[
    \frac{x^{2}}{x^{3}}=\frac{1}{x},
\]
which implies that the surface-area integral diverges because \(\int_{1}^{\infty}x^{-1}dx\) diverges. Indeed, one can show explicitly that
\[
    \lim_{b\to\infty}S(b)=\infty.
\]

Hence Gabriel's horn possesses a finite volume yet an infinite surface area.


\subsection{Appendix: Moar Integral}
\subsubsection{Basic.} Most common integrals.
\begin{flalign*}
     & \begin{aligned}
            & \int \frac{1}{x}\;dx = \ln |x|                     \\
            & \int u \;dv = uv - \int v \;du                     \\
            & \int u\;dv = uv - \int v \;du                      \\
            & \int \frac{1}{ax+b}\;dx = \frac{1}{a} \ln |ax + b| \\
       \end{aligned} &
\end{flalign*}

\subsubsection{Rational Functions.} Integrals of rational function
\begin{flalign*}
     & \begin{aligned}
            & \int \frac{1}{(x+a)^2}\;dx = -\frac{1}{x+a}                                                  \\
            & \int (x+a)^n \;dx = \frac{(x+a)^{n+1}}{n+1}, n\ne -1                                         \\
            & \int x(x+a)^n \;dx = \frac{(x+a)^{n+1} ( (n+1)x-a)}{(n+1)(n+2)}                              \\
            & \int \frac{1}{1+x^2}\;dx = \tan^{-1}x                                                        \\
            & \int \frac{1}{a^2+x^2}\;dx = \frac{1}{a}\tan^{-1}\frac{x}{a}                                 \\
            & \int \frac{x}{a^2+x^2}\;dx = \frac{1}{2}\ln|a^2+x^2|                                         \\
            & \int \frac{x^2}{a^2+x^2}\;dx = x-a\tan^{-1}\frac{x}{a}                                       \\
            & \int \frac{x^3}{a^2+x^2}\;dx = \frac{1}{2}x^2-\frac{1}{2}a^2\ln|a^2+x^2|                     \\
            & \int \frac{1}{ax^2+bx+c}\;dx = \frac{2}{\sqrt{4ac-b^2}}\tan^{-1}\frac{2ax+b}{\sqrt{4ac-b^2}} \\
            & \int \frac{1}{(x+a)(x+b)}\;dx = \frac{1}{b-a}\ln\frac{a+x}{b+x}, \quad a\neq b               \\
            & \int \frac{x}{(x+a)^2}\;dx = \frac{a}{a+x}+\ln |a+x|                                         \\
            & \int \frac{x}{ax^2+bx+c}\;dx = \frac{1}{2a}\ln|ax^2+bx+c|-                                   \\
            & \frac{b}{a\sqrt{4ac-b^2}}\tan^{-1}\frac{2ax+b}{\sqrt{4ac-b^2}}                               \\
       \end{aligned} &
\end{flalign*}

\subsubsection{Roots.} Integrals of roots.
\begin{flalign*}
     & \begin{aligned}
            & \int \sqrt{x-a}\; dx = \frac{2}{3}(x-a)^{3/2}      \\
            & \int \frac{1}{\sqrt{x\pm a}}\; dx = 2\sqrt{x\pm a} \\
       \end{aligned} &
\end{flalign*}
\begin{flalign*}
     & \begin{aligned}
            & \int \frac{1}{\sqrt{a-x}}\; dx = -2\sqrt{a-x}                                                                                                                                                                                     \\
            & \int x\sqrt{x-a}\; dx =  \begin{cases}\frac{2 a}{3} \left({x-a}\right)^{3/2} +\frac{2 }{5}\left( {x-a}\right)^{5/2},\text{ or} \\ \frac{2}{3} x(x-a)^{3/2} - \frac{4}{15} (x-a)^{5/2}, \text{ or}\\ \frac{2}{15}(2a+3x)(x-a)^{3/2}
                                    \end{cases} \\
            & \int \sqrt{ax+b}\; dx = \left(\frac{2b}{3a}+\frac{2x}{3}\right)\sqrt{ax+b}                                                                                                                                                        \\
            & \int (ax+b)^{3/2}\; dx =\frac{2}{5a}(ax+b)^{5/2}                                                                                                                                                                                  \\
            & \int \frac{x}{\sqrt{x\pm a} } \; dx = \frac{2}{3}(x\mp 2a)\sqrt{x\pm a}                                                                                                                                                           \\
            & \int \sqrt{\frac{x}{a-x}}\; dx =  -\sqrt{x(a-x)}-a\tan^{-1}\frac{\sqrt{x(a-x)}}{x-a}                                                                                                                                              \\
            & \int \sqrt{\frac{x}{a+x}}\; dx =  \sqrt{x(a+x)} -a\ln  [ \sqrt{x} + \sqrt{x+a}]                                                                                                                                                   \\
            & \int x \sqrt{ax + b}\; dx =\frac{2}{15 a^2}(-2b^2+abx + 3 a^2 x^2)\sqrt{ax+b}                                                                                                                                                     \\
            & \int \sqrt{x^3(ax+b)} \; dx =\left[ \frac{b}{12a}-\frac{b^2}{8a^2x}+\frac{x}{3}\right] \sqrt{x^3(ax+b)}                                                                                                                           \\
            & + \frac{b^3}{8a^{5/2}}\ln \left| a\sqrt{x} + \sqrt{a(ax+b)} \right|                                                                                                                                                               \\
            & \int  \sqrt{a^2 - x^2}\; dx = \frac{1}{2} x \sqrt{a^2-x^2} +\frac{1}{2}a^2\tan^{-1}\frac{x}{\sqrt{a^2-x^2}}                                                                                                                       \\
            & \int  x \sqrt{x^2 \pm a^2}\; dx= \frac{1}{3}\left( x^2 \pm a^2 \right)^{3/2}                                                                                                                                                      \\
            & \int \frac{1}{\sqrt{x^2 \pm a^2}}\; dx = \ln \left| x + \sqrt{x^2 \pm a^2} \right|                                                                                                                                                \\
            & \int \frac{1}{\sqrt{a^2 - x^2}}\; dx = \sin^{-1}\frac{x}{a}                                                                                                                                                                       \\
            & \int \frac{x}{\sqrt{x^2\pm a^2}}\; dx = \sqrt{x^2 \pm a^2}                                                                                                                                                                        \\
            & \int \frac{x}{\sqrt{a^2-x^2}}\; dx = -\sqrt{a^2-x^2}                                                                                                                                                                              \\
            & \int \frac{x^2}{\sqrt{x^2 \pm a^2}}\; dx = \frac{1}{2}x\sqrt{x^2 \pm a^2}\mp \frac{1}{2}a^2 \ln \left| x + \sqrt{x^2\pm a^2} \right|                                                                                              \\
            & \int\frac{1}{\sqrt{ax^2+bx+c}}\;dx=\frac{1}{\sqrt{a}}\ln \left| 2ax+b + 2 \sqrt{a(ax^2+bx+c)} \right|                                                                                                                             \\
            & \int\frac{dx}{(a^2+x^2)^{3/2}}=\frac{x}{a^2\sqrt{a^2+x^2}}                                                                                                                                                                        \\
       \end{aligned} &
\end{flalign*}

\subsubsection{Integrals with Logarithms.}

\begin{flalign*}
     & \begin{aligned}
            & \int \ln ax \;dx = x \ln (ax) - x                                                                   \\
            & \int \frac{\ln ax}{x} \;dx = \frac{1}{2} ( \ln ax )^2                                               \\
            & \int \ln (ax + b) dx =  ( x + \frac{b}{a} ) \ln (ax+b) - x , \quad a \neq 0                         \\
            & \int \ln  ( x^2 + a^2 )\;{dx} = x \ln (x^2 + a^2  ) +2a\tan^{-1} \frac{x}{a} - 2x                   \\
            & \int \ln  ( x^2 - a^2 )\;dx= x \ln (x^2 - a^2  ) +a\ln \frac{x+a}{x-a} - 2x                         \\
            & \int \ln  ( x^2 - a^2 )\;dx = x \ln (x^2 - a^2  ) +a\ln \frac{x+a}{x-a} - 2x                        \\
            & \int \ln  ( ax^2 + bx + c) dx  = \frac{1}{a}\sqrt{4ac-b^2}\tan^{-1}\frac{2ax+b}{\sqrt{4ac-b^2}} -2x \\
            & + ( \frac{b}{2a}+x )\ln \ (ax^2+bx+c )                                                              \\
       \end{aligned} &
\end{flalign*}
\begin{flalign*}
     & \begin{aligned}
            & \int x \ln (ax + b) dx = \frac{bx}{2a}-\frac{1}{4}x^2+\frac{1}{2}(x^2-\frac{b^2}{a^2})\ln (ax+b) \\
            & \frac{1}{2}( x^2 - \frac{a^2}{b^2}  ) \ln  (a^2 -b^2 x^2 )                                       \\
       \end{aligned} &
\end{flalign*}

\subsubsection{Integrals with Exponential.}
\begin{flalign*}
     & \begin{aligned}
            & \int e^{ax} dx = \frac{1}{a}e^{ax}                                                                           \\
            & \int \sqrt{x} e^{ax} dx = \frac{1}{a}\sqrt{x}e^{ax} +\frac{i\sqrt{\pi}}{2a^{3/2}}\erf\left(i\sqrt{ax}\right) \\
            & \int x e^x dx = (x-1) e^x                                                                                    \\
            & \int x e^{ax} dx = \left(\frac{x}{a}-\frac{1}{a^2}\right) e^{ax}                                             \\
            & \int x^2 e^{x} dx = \left(x^2 - 2x + 2\right) e^{x}                                                          \\
            & \int x^2 e^{ax} dx = \left(\frac{x^2}{a}-\frac{2x}{a^2}+\frac{2}{a^3}\right) e^{ax}                          \\
            & \int x^3 e^{x} dx = \left(x^3-3x^2 + 6x - 6\right) e^{x}                                                     \\
            & \int x^n e^{ax}\; dx= \dfrac{x^n e^{ax}}{a} -\dfrac{n}{a}\int x^{n-1}e^{ax}\; dx                             \\
            & \int x^n e^{ax}\;dx = \frac{(-1)^n}{a^{n+1}}\Gamma[1+n,-ax]                                                  \\
            & \int e^{ax^2}\; dx= -\frac{i\sqrt{\pi}}{2\sqrt{a}}\erf\left(ix\sqrt{a}\right)                                \\
            & \int e^{-ax^2}\; dx= \frac{\sqrt{\pi}}{2\sqrt{a}}\erf\left(x\sqrt{a}\right)                                  \\
       \end{aligned} &
\end{flalign*}
\begin{flalign*}
     & \begin{aligned}
            & \int x e^{-ax^2}\ dx= -\dfrac{1}{2a}e^{-ax^2}                                                      \\
            & \int x^2 e^{-ax^2}\ dx= \dfrac{1}{4}\sqrt{\dfrac{\pi}{a^3}}\erf(x\sqrt{a}) -\dfrac{x}{2a}e^{-ax^2} \\
       \end{aligned} &
\end{flalign*}

\subsubsection{Integrals with Trigonometry Functions.}
\begin{flalign*}
     & \begin{aligned}
            & \int \sin ax \;dx = -\frac{1}{a} \cos ax                                                                                                   \\
            & \int \sin^2 ax \;dx = \frac{x}{2} - \frac{\sin 2ax} {4a}                                                                                   \\
            & \int \sin^n ax \;dx =-\frac{1}{a}\cos ax \times {_2F_1}\left[\frac{1}{2}, \frac{1-n}{2}, \frac{3}{2}, \cos^2 ax \right]                    \\
            & \int \sin^3 ax \;dx = -\frac{3 \cos ax}{4a} + \frac{\cos 3ax} {12a}                                                                        \\
            & \int \cos ax \;dx= \frac{1}{a} \sin ax                                                                                                     \\
            & \int \cos^2 ax \;dx = \frac{x}{2}+\frac{ \sin 2ax}{4a}                                                                                     \\
            & \int \cos^p ax \;dx  = -\frac{1}{a(1+p)}{\cos^{1+p} ax} \times  {_2F_1}\left[ \frac{1+p}{2}, \frac{1}{2}, \frac{3+p}{2}, \cos^2 ax \right] \\
            & \int \cos^3 ax \;dx = \frac{3 \sin ax}{4a}+\frac{ \sin 3ax}{12a}                                                                           \\
       \end{aligned} &
\end{flalign*}
\begin{flalign*}
     & \begin{aligned}
            & \int \cos ax \sin bx \;dx = \frac{\cos[(a-b) x]}{2(a-b)} -  \frac{\cos[(a+b)x]}{2(a+b)} ,\quad a\neq b                           \\
            & \int \sin^2 ax \cos bx \;dx = -\frac{\sin[(2a-b)x]}{4(2a-b)}  + \frac{\sin bx}{2b} - \frac{\sin[(2a+b)x]}{4(2a+b)}               \\
            & \int \sin^2 x \cos x \;dx = \frac{1}{3} \sin^3 x                                                                                 \\
            & \int \cos^2 ax \sin bx \;dx = \frac{\cos[(2a-b)x]}{4(2a-b)}  - \frac{\cos bx}{2b} - \frac{\cos[(2a+b)x]}{4(2a+b)}                \\
            & \int \cos^2 ax \sin ax \;dx = -\frac{1}{3a}\cos^3{ax}                                                                            \\
            & \int \sin^2 ax \cos^2 bx \;dx = \frac{x}{4} -\frac{\sin 2ax}{8a}- \frac{\sin[2(a-b)x]}{16(a-b)} +\frac{\sin 2bx}{8b}             \\
            & - \frac{\sin[2(a+b)x]}{16(a+b)}                                                                                                  \\
            & \int \sin^2 ax \cos^2 ax \;dx = \frac{x}{8}-\frac{\sin 4ax}{32a}                                                                 \\
            & \int \tan ax \;dx = -\frac{1}{a} \ln \cos ax                                                                                     \\
            & \int \tan^2 ax \;dx = -x + \frac{1}{a} \tan ax                                                                                   \\
            & \int \tan^n ax \;dx =     \frac{\tan^{n+1} ax }{a(1+n)} \times {_2}F_1\left( \frac{n+1}{2}, 1, \frac{n+3}{2}, -\tan^2 ax \right) \\
            & \int \tan^3 ax \;dx = \frac{1}{a} \ln \cos ax + \frac{1}{2a}\sec^2 ax                                                            \\
            & \int \sec x \;dx= \ln | \sec x + \tan x | = 2 \tanh^{-1} \left(\tan \frac{x}{2} \right)                                          \\
            & \int \sec^2 ax \;dx = \frac{1}{a} \tan ax                                                                                        \\
            & \int \sec^3 x \;dx= \frac{1}{2} \sec x \tan x + \frac{1}{2}\ln | \sec x + \tan x |                                               \\
            & \int \sec x \tan x \;dx = \sec x                                                                                                 \\
            & \int \sec^2 x \tan x \;dx = \frac{1}{2} \sec^2 x                                                                                 \\
            & \int \sec^n x \tan x \;dx = \frac{1}{n} \sec^n x , \quad n\neq 0                                                                 \\
            & \int \csc x \;dx = \ln \left| \tan \frac{x}{2} \right|  = \ln | \csc x - \cot x| + C                                             \\
            & \int \csc^2 ax \;dx = -\frac{1}{a} \cot ax                                                                                       \\
            & \int \csc^3 x \;dx = -\frac{1}{2}\cot x \csc x + \frac{1}{2} \ln | \csc x - \cot x |                                             \\
            & \int \csc^nx \cot x \;dx = -\frac{1}{n}\csc^n x, n\ne 0                                                                          \\
            & \int \sec x \csc x \;dx = \ln | \tan x |
       \end{aligned} &
\end{flalign*}
\subsubsection{Products of Trigonometry Functions and Monomials.}
\begin{flalign*}
     & \begin{aligned}
            & \int x \cos x \;dx = \cos x + x \sin x                                                                 \\
            & \int x \cos ax \;dx = \frac{1}{a^2} \cos ax + \frac{x}{a} \sin ax                                      \\
            & \int x^2 \cos x \;dx = 2 x \cos x + \left ( x^2 - 2 \right ) \sin x                                    \\
            & \int x^2 \cos ax \;dx = \frac{2 x \cos ax }{a^2} + \frac{ a^2 x^2 - 2  }{a^3} \sin ax                  \\
            & \int  x^n \cos x \;dx = -\frac{1}{2}(i)^{n+1}\left[ \Gamma(n+1, -ix)  + (-1)^n \Gamma(n+1, ix)\right]  \\
            & \int x^n cos ax \;dx = \frac{1}{2}(ia)^{1-n}\left[ (-1)^n  \Gamma(n+1, -iax)  -\Gamma(n+1, ixa)\right] \\
            & \int x \sin x \;dx = -x \cos x + \sin x                                                                \\
            & \int x \sin ax \;dx = -\frac{x \cos ax}{a} + \frac{\sin ax}{a^2}                                       \\
            & \int x^2 \sin x \;dx = \left(2-x^2\right) \cos x + 2 x \sin x                                          \\
            & \int x^2 \sin ax \;dx =\frac{2-a^2x^2}{a^3}\cos ax +\frac{ 2 x \sin ax}{a^2}                           \\
            & \int x^n \sin x \;dx = -\frac{1}{2}(i)^n\left[ \Gamma(n+1, -ix)  - (-1)^n\Gamma(n+1, -ix)\right]
       \end{aligned} &
\end{flalign*}

\subsubsection{Products of Trigonometry Functions and Exponential.}
\begin{flalign*}
     & \begin{aligned}
            & \int e^x \sin x \;dx = \frac{1}{2}e^x (\sin x - \cos x)                         \\
            & \int e^{bx} \sin ax \;dx = \frac{1}{a^2+b^2}e^{bx} (b\sin ax - a\cos ax)        \\
            & \int e^x \cos x \;dx = \frac{1}{2}e^x (\sin x + \cos x)                         \\
            & \int e^{bx} \cos ax \;dx = \frac{1}{a^2 + b^2} e^{bx} ( a \sin ax + b \cos ax ) \\
            & \int x e^x \sin x \;dx = \frac{1}{2}e^x (\cos x - x \cos x + x \sin x)          \\
            & \int x e^x \cos x \;dx = \frac{1}{2}e^x (x \cos x - \sin x + x \sin x)          \\
       \end{aligned} &
\end{flalign*}

\subsubsection{Integrals of Hyperbolic Functions.}
\begin{flalign*}
     & \begin{aligned}
            & \int \cosh ax dx =\frac{1}{a} \sinh ax                                                                                                                             \\
            & \int e^{ax}  \cosh bx dx = \begin{cases}
                                          \dfrac{e^{ax}}{a^2-b^2} \left[ a \cosh bx - b \sinh bx \right],\quad    a\neq b \\\\
                                          \dfrac{e^{2ax}}{4a} + \dfrac{x}{2},\quad a = b
                                      \end{cases}                               \\
            & \int \sinh ax dx = \frac{1}{a} \cosh ax                                                                                                                            \\
            & \int e^{ax} \sinh bx dx =  \begin{cases}
                                          {\dfrac{e^{ax}}{a^2-b^2} }\left[ -b \cosh bx + a \sinh bx \right]  ,\quad a\neq b \\\\
                                          {\dfrac{e^{2ax}}{4a} - \dfrac{x}{2}}   a = b
                                      \end{cases}                                                     \\
            & \int  e^{ax} \tanh bx dx =   \begin{cases}
                                            \displaystyle{ \frac{ e^{(a+2b)x}}{(a+2b)} {_2F_1}\left[ 1+\frac{a}{2b},1,2+\frac{a}{2b}, -e^{2bx}\right] } & \\\\
                                            \displaystyle{ -\frac{1}{a}e^{ax}{_2F_1}\left[ \frac{a}{2b},1,1E, -e^{2bx}\right]},\quad a\neq b              \\\\
                                            \displaystyle{\frac{e^{ax}-2\tan^{-1}[e^{ax}]}{a} },\quad a = b
                                        \end{cases} \\
            & \int  \tanh ax\;dx =\frac{1}{a} \ln \cosh ax                                                                                                                       \\
            & \int \cos ax \cosh bx dx = \frac{1}{a^2 + b^2} \left[  a \sin ax \cosh bx + b \cos ax \sinh bx \right]                                                             \\
            & \int \cos ax \sinh bx dx = \frac{1}{a^2 + b^2} \left[    b \cos ax \cosh bx + a \sin ax \sinh bx  \right]                                                          \\
            & \int \sin ax \cosh bx dx = \frac{1}{a^2 + b^2} \left[ -a \cos ax \cosh bx +b \sin ax \sinh bx \right]                                                              \\
            & \int \sin ax \sinh bx dx =  \frac{1}{a^2 + b^2} \left[ b \cosh bx \sin ax - a \cos ax \sinh bx \right]                                                             \\
            & \int \sinh ax \cosh ax dx= \frac{1}{4a}\left[  -2ax + \sinh 2ax \right]                                                                                            \\
            & \int \sinh ax \cosh bx dx= \frac{1}{b^2-a^2}\left[  b \cosh bx \sinh ax  - a \cosh ax \sinh bx \right]                                                             \\
       \end{aligned} &
\end{flalign*}
\end{document}