\documentclass[../../../main.tex]{subfiles}
\begin{document}
Mainly consist of precalculus, and basic Calculus.
\subsection{Algebra}
\subsubsection{Laws of Exponents.}
\begin{align*}
    x^{\tfrac{m}{n}}&=\sqrt[n]{m}\\
    (x^m)^n&=x^{mn}\\
    x^mx^n&=x^{m+n}\\
    x^ay^a&=(xy)^a\\
\end{align*}

\subsubsection{Special Factorization.}
\begin{align*}
    x^2-y^2&=(x+y)(x-y)\\
    x^3-y^3&=(x-y)(x^2+xy+y^2)\\
    x^3+y^3&=(x+y)(x^2-xy+y^2)\\
\end{align*}

\subsubsection{Quadratic formula.} The following formula can be used to find the roots of quadratic equation $ax^2+bx+c$
\begin{align*}
    x_1,x_2=\dfrac{-b\pm \sqrt{b^2-4ac}}{2a}\;\;
    \begin{cases}
        D>0 \;\;\text{re(2)}\\
        D=0\;\;\text{re(1)}\\
        D<0\;\;\text{im(2)}
    \end{cases}
\end{align*}
The said quadratic equation can also be writen as 
\begin{equation*}
    ax^2+bc+c=a(x-x_1)(x-x_2)
\end{equation*}

\subsubsection{Binomial theorem.}
\begin{align*}
    (a+b)^n=\sum_{k=0}^{\infty}\binom{n}{k}a^{n-k}b^{k}
\end{align*}
with
\begin{equation*}
    \binom{n}{k} = \frac{n!}{k!(n-k)!}=\frac{\Gamma (n+1)}{\Gamma(k+1)\Gamma(n-k+1)}
\end{equation*}

\subsection{Trigonometry}
\subsubsection{Trigonometry Definitions.}
\begin{align*}
    \sin \theta&=\dfrac{1}{\csc \theta}\\
    \cos \theta&=\dfrac{1}{\sec \theta}\\
    \tan \theta&=\dfrac{1}{\cot \theta}\\
\end{align*}

\subsubsection{Pythagorean Identities.}
\begin{align*}
    \sin^2 \theta+ \cos^2 \theta &=1\\
    \sec^2 \theta- \tan^2 \theta &=1\\
    \csc^2 \theta- \cot^2 \theta &=1\\
\end{align*}

\subsubsection{Law of Sines.}
\begin{align*}
    \frac{\sin A}{a}=\frac{\sin B}{b}=\frac{\sin C}{c}
\end{align*}

\subsubsection{Law of Cosines.}
\begin{align*}
    a^2=b^2+c^2-2bc\cos A
\end{align*}

\subsubsection{Trigonometry Double Angle Identities.}
\begin{align*}
    \sin 2\theta&=2\sin \theta \cos \theta\\
    \cos 2 \theta&=1-2\sin^2\theta\\
    &=2\cos^2\theta -1\\
    &=\cos^2\theta-\sin^2\theta\\
    \tan 2\theta&=\frac{2\tan\theta}{1-\tan^2\theta}
\end{align*}

\subsubsection{Trigonometry Addition and Difference Identities.}
\begin{align*}
    \sin(x+y)&=\sin x \cos y+ \cos x \sin y\\
    \sin(x-y)&=\sin x \cos y- \cos x \sin y\\
    \cos(x+y)&=\cos x \cos y- \sin x \sin y\\
    \cos(x-y)&=\cos x \cos y- \sin x \sin y\\
    \tan (x+y)&=\frac{\tan x+\tan y}{1-\tan x\tan y}\\
    \tan (x-y)&=\frac{\tan x-\tan y}{1+\tan x\tan y}\\
\end{align*}

\subsubsection{Trigonometry Product Rule.}
\begin{align*}
    \cos x \cos y&= \frac{1}{2}[\cos(x-y)+\cos(x+y)]\\
    \sin x \sin y&= \frac{1}{2}[\cos(x-y)-\cos(x+y)]\\
    \sin x \cos y&= \frac{1}{2}[\sin(x+y)+\sin(x-y)]\\
    \cos x \sin y&= \frac{1}{2}[\sin(x+y)-\sin(x-y)]
\end{align*}

\subsubsection{Neat Mnemonics.}
\begin{align*}
    \begin{vmatrix}
        \mathrm{S^+}\\
        \mathrm{S^-}\\
        \mathrm{C^+}\\
        \mathrm{C^-}
    \end{vmatrix}=
    \begin{vmatrix}
        \mathrm{SC+CS}\\
        \mathrm{SC-CS}\\
        \mathrm{CC-SS}\\
        \mathrm{CC+SS}
    \end{vmatrix}
    &&\mathrm{and}&&
    \begin{vmatrix}
        \mathrm{CC}\\
        \mathrm{SS}\\
        \mathrm{SC}\\
        \mathrm{CS}
    \end{vmatrix}=
    \begin{vmatrix}
        \mathrm{C^-+C^+}\\
        \mathrm{C^--C^+}\\
        \mathrm{S^++S^-}\\
        \mathrm{S^+-S^-}
    \end{vmatrix}
\end{align*}

\subsection{Logarithm}
\subsubsection{Definition (informal).} $\log_a b$ means $a$ to the power of what equal $b$.

\subsubsection{Few important log rule.}
\begin{align*}
    \log_c (ab)&=\log_c (a)+\log_c (b)\\
    \log_c (\frac{a}{b})&=\log_c (a)-\log_c (b)\\
    \log_a b&=\frac{\log_c (b)}{\log_c (a)}\\
    a^{\log_a b}&=b
\end{align*}

\subsection{Limit}
\subsubsection{Few Important Limits.}
\begin{align*}
    \lim_{x\to a} c&=c\\
    \lim_{x\to 0^+} \frac{1}{x}&=\infty\rightarrow\frac{1}{0^+}=\infty\\
    \lim_{x\to 0^-} \frac{1}{x}&=-\infty\rightarrow\frac{1}{0^-}=-\infty
    \lim_{x\to0} \frac{\sin x }{x}=\lim_{x\to0} \frac{\sin x }{x}=1\\
    \lim_{x\to0} \frac{\cos x-1 }{x}=0\\
    \lim_{x\to\infty} \left(1+\frac{1 }{x }\right)^x=e
\end{align*}
\subsubsection{Limit as Definition of Derivative.}
\begin{align*}
    \df y=\lim_{h\to 0}\frac{f(x+h)-f(x)}{h}
\end{align*}

\subsection{Derivative}

\subsubsection{Order of calculation.} How to determine the order of derivation: last computation is the first thing to do.

\subsubsection{General Formula.}
\begin{align*}
     D\; x^n&=nx^{n-1}\\
     D\; (uv)&= D\;u\cdot v+u\cdot D\;v\\
     D\;\bigg(\frac{u}{v}\bigg)&=\frac{ D\;u\cdot v-u\cdot D\;v}{v^2}
\end{align*}

\subsubsection{Trigonometry Formula.}
\begin{align*}
     D\; \sin x&= \cos x\\
     D\; \cos x&= -\sin x\\
     D\; \tan x&= \sec^2 x\\
     D\; \cot x&= -\csc^2 x\\
     D\; \sec x&= \sec x\tan x\\
     D\; \csc x&= -\cot x\csc x\\
\end{align*}

\subsubsection{Neat Mnemonics.}
\begin{align*}
\begin{matrix}
    \sec&\sec&\tan&\downarrow \textrm{cofunction}\\
    \csc&-\csc&\cot&\\
    &\leftrightarrow\textrm{multiply}
\end{matrix}
\end{align*}

\subsubsection{Exponential and Logarithmic Functions.}
\begin{align*}
     D\; \ln x &= \frac{1}{x}\\
     D\; a^x&=a^x\ln x\\
     D\; \log_a b&=\frac{1}{b\ln a}
\end{align*}

\subsubsection{Minima and Maxima test.} First derivative test:
\begin{itemize}
    \item Determine critical points ($ Dy=0$), then divide into region;
    \item Pick value from each region and plug into \emph{derivative}; and
    \item Do the sign-graph thing.
\end{itemize}
Second derivative test:
\begin{itemize}
    \item Determine critical points;
    \item Plug critical into second derivative; and
    \item Positive $ D^2y$ means concave up ($\smile$) or minima, negative means concave down ($\frown$) or maxima, and 0 means inconclusive. Simply put, positive means minima, while negative means maxima.
\end{itemize}

\subsubsection{Optimization with constrain.} Elimination method:
\begin{enumerate}
    \item Write the function $f$. The function itself must be in terms of one independent variable, say $x$, which can often be achieved by substituting our constraint, say $y(x)$, into the function $f(x)$.
    \item Find the critical points.
    \item Use whatever test you need.
\end{enumerate}
Implicit differentiation method:
\begin{enumerate}
    \item Write the function $f(x,y)$. This method assumes that it is not possible to solve substituting the constant $y$ into our equation.
    \item Write the differentiation with respect to independent $x$ variable. Note that the derivative of the dependent variable often still remains; we need to solve for them too.
    \item Use the following result to find the critical points.
    \item Use the second derivative test. Note that you'll also need the second derivative of the constraint $y$ evaluated at critical points to determine the second derivative of the function $f(x,y)$
\end{enumerate}

\subsubsection{Differentiation under integral sign.} Differentiation under integral sign stated by Leibniz' rule
\begin{equation*}
    \frac{d}{dx}\int_{u(x)}^{v(x)}f(x,t)\;dt=\int_{u}^{v}\frac{\partial f}{\partial x}\;dt + f(x,v)\frac{dv}{dx}-f(x,u)\frac{du}{dx}
\end{equation*}

\emph{Proof.} Suppose we want $dI/dx$ where
\begin{equation*}
    I=\int_{u }^{v }f(t)\;dt
\end{equation*}
By the fundamental theorem of calculus
\begin{equation*}
    I=F(v)-F(u)=\mathcal{F}(v,u)
\end{equation*}
or $I$ is a function of $v$ and $u$. Finding $dI/dx$ is then a partial differentiation problem. We can write
\begin{equation*}
    \frac{dI}{dx}=\frac{\partial I}{\partial v}\frac{dv}{dx}+\frac{\partial I}{\partial u}\frac{du}{dx}
\end{equation*}
By the fundamental theorem of calculus, we have 
\begin{align*}
    \frac{d}{dv}\int_{a}^{v}f(x)\;dt&=\frac{d}{dv}\bigl[F(v)-F(a)\bigr]=f(v)\\
    \frac{d}{dv}\int_{u}^{b}f(x)\;dt&=\frac{d}{dv}\bigl[F(b)-F(u)\bigr]=-f(u)
\end{align*}
where $u$ and $v$ are a function of $x$, while $a$ and $b$ are a constant. This is the case when we consider $\partial I/\partial v$ or $\partial I/\partial v$; the other variable is constant. Then 
\begin{equation*}
    \frac{d}{dx}\int_{u }^{v }f(t)\;dt=f(v)\frac{dv}{dx}-f(u)\frac{du}{dx}
\end{equation*}

Under not too restrictive conditions, 
\begin{equation*}
    \frac{d}{dx}\int_{a }^{b }f(x,t)\;dt =\int_{a }^{b }\frac{\partial f(x,t)}{\partial x}\;dt
\end{equation*}
where, as before, $a$ and $b$ are constant. In other words, we can differentiate under the integral sign. It is convenient to collect these formulas into one formula known as Leibniz' rule:
\begin{equation*}
    \frac{d}{dx}\int_{u(x)}^{v(x)}f(x,t)\;dt=\int_{u}^{v}\frac{\partial f}{\partial x}\;dt + f(x,v)\frac{dv}{dx}-f(x,u)\frac{du}{dx}\quad\blacksquare
\end{equation*}

\subsubsection{Leibniz’ rule for differentiating a product.}
\begin{equation*}
    \left(\frac{d}{dx}\right)^n(fg)=\sum_{k=0}^{n}{n\choose k}\left(\frac{d}{dx}\right)^{n-k}(f)\left(\frac{d}{dx}\right)^k(g)
\end{equation*}
where
\begin{equation*}
    {n \choose k}={\frac{n!}{k! (n-k)!}}
\end{equation*}

\subsection{Integral}
\subsubsection{Basic Formula (integration constant omitted).}
\begin{align*}
    \int x^n \;dx &= \frac{1}{n+1}x^{n+1}\\
    \int \frac{1}{x}\;dx& = \ln |x|\\
    \int u \;dv& = uv - \int v\; du\\
    \int a^x \;dx&=\frac{a^x}{\ln a}
\end{align*}

\subsubsection{Trigonometry.}
\begin{align*}
    \int \sin x \; dx&=-\cos x\\
    \int \cos x \; dx&=\sin x\\
    \int \sec^2 x \; dx&=\tan x\\
    \int \csc^2 x \; dx&=-\cot x\\
    \int \sec x\tan x \; dx&=\sec x\\
    \int \csc x\tan x \; dx&=-\csc x
\end{align*}

\subsubsection{Root.}
\begin{align*}
    \int \frac{1}{\sqrt{a^2-x^2}}\;dx&=\arcsin \frac{x}{a}\\
    \int \frac{1}{\sqrt{x^2\pm a^2}}\;dx&=\ln x +\sqrt{x^2\pm a^2}\\
    \int \frac{1}{\sqrt{a^2+x^2}}\;dx&=\frac{1}{a}\arctan \frac{x}{a}
\end{align*}

\subsubsection{Integration by part.}
\begin{enumerate}
    \item Splits the integrand. Choose $u$ using LIATEN and let the rest be $dv$. (LIATEN: Log, Inverse trigonometry, Algebra, Trigonometry, ExponeN)
    \item Do the box thing
    \begin{table*}[h]
        \begin{center}
    \caption*{Table: The box thing}
    \begin{tabular}{c || c || c}
        $u$&$v$&$\downarrow$ Differentiate\\ 
        \hline\hline
        $du$&$dv$&$\uparrow$ Integrate
    \end{tabular}
        \end{center}
    \end{table*}
    \item $ \int u \;dv = uv - \int v\; du$
\end{enumerate}

\subsubsection{Tabular Method.} Refer to the table.
\begin{table*}[h]
\begin{center}
\caption*{Table: The table}
\begin{tabular}{c || c || c}
    &Differentiate&Integrate \\ 
    \hline\hline
    +&$a\searrow$&b \\ 
    \hline
    -&$a'\searrow$&b \\ 
    \hline
    +&$a''\searrow$&b \\ 
    \hline
    $\vdots$&$\vdots$&$\vdots$
\end{tabular}
\end{center}
\end{table*}
Steps:
\begin{enumerate}
    \item 0 in $D$ column or use LIATEN,
    \item Integrate a row, and
    \item A row repeats.
\end{enumerate}

\subsubsection{Trigonometry Integral.} Pythagorean Identity.
\begin{align*}
    \sin^2x+\cos^2x&=1\\
    \sin^2x&=\frac{1-\cos 2x}{2}\\
    \cos^2x&=\frac{1+\cos2x}{2}
\end{align*}
note that argument inside quadratic trigonometry is half of trigonometry, which means $\cos^22x=(1+\cos 4x)/{2}$. There are few cases of tricky trigonometry integral. First, if power of sin is odd and positive. The steps to evaluate it are as follows.
\begin{enumerate}
    \item Remove one power off
    \item convert remaining (even power) using Pythagorean Identity in terms of cosine 
    \item integrate using subs method
\end{enumerate}
If the power of sine is odd and positive.
\begin{enumerate}
    \item Same as before
\end{enumerate}
If the power of sine and cosine is even and nonegative, then:
\begin{enumerate}
    \item convert using Pythagorean Identity and solve
\end{enumerate}

\subsubsection{Trigonometry substitution.} Trigonometry function and its radical pair
\begin{align*}
    \tan \theta&=\sqrt{u^2+a^2}\\
    \sin \theta&=\sqrt{a^2-u^2}\\
    \sec \theta&=\sqrt{u^2-a^2}\\
\end{align*}
where $u$ is the variable we are differentiating with respect to. Mnemonics: $+$ looks like tangent; $-$ for sin and sec; and it is \emph{a} \emph{s}in. Trigonometry substitution step is then as follows.
\begin{enumerate}
    \item Draw a right triangle where trigonometry pair equal $u/a$
    \item using the trigonometry pair equation*, solve for $x$ and $dx$
    \item find trigonometry where $\sqrt{}/a$
    \item subs again if equation* still contain $\theta$ and solve
\end{enumerate}

\subsubsection{Partial Fraction. }
\begin{enumerate}
    \item Factor out denominator
    \item Breakup the function and put unknown (Capital Letter) into numerator. Put numerator normally if factor is linear, put $Px+Q$ Irreducible quadratic factor IQF. In general, \begin{equation*}
        \frac{Ax^{n-1}+Bx^{n-2}+\cdots}{x^{n}+x^{n-1}+\cdots}
    \end{equation*}
    \item Multiply both side by left side's denominator
    \item Take the roots of the linear factors and plug them into x, and solve for the unknowns
    \item Put unknowns into step 2
    \item Splits Integral, then solve
    \item For equating coefficients like terms, after step 3, expand equation*. Then, collect like terms and equate coefficient of like terms from both side
\end{enumerate}
\end{document}