\documentclass[../../../main.tex]{subfiles}
\begin{document}
\subsection{Definition}
Many older books define a tensor as a collection of objects which carry indices and which 'transform' in a particular way specified by those indices.
This definition takes a tensor to be a function which eats a certain number of vectors (known as the rank $r$ of the tensor) and produces a number.
The distinguishing characteristic of a tensor is a special property called multilinearity, which enables us to express the value of the function on an arbitrary set of $r$ vectors in terms of the values of the function on $r$ basis vectors.
In older treatments, these are usually introduced as components of the tensor.

In physics textbooks, tensors (usually of the second rank) are often represented as matrices
It is crucial to keep in mind, though, that this association between a tensor and a matrix depends entirely on a choice of basis, and that matrix $[T]$ is useful mainly as a computational tool, not a conceptual handle.
Tensor $T$ is best thought of abstractly as a multilinear function, and matrix $[T ]$ as its representation in a particular coordinate system.

\end{document}