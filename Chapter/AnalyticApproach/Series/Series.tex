\documentclass[../../../main.tex]{subfiles}
\begin{document}
\subsection{Converge Test}
Geometric Series (for r less than one):
\begin{align*}
    S_n&=\frac{a(1-r^n)}{1-r}\\
    S&=\lim_{n \to \infty} S_n=\frac{a}{1-r}
\end{align*}

Preliminary Test:
\begin{align*}
    \text{if }\lim_{n \to \infty} a_n \neq 0\;,\;\text{then series Diverges}
\end{align*}

Comparation Test:
\begin{align*}
    A&\leq C\text{ , A Converges}\\
    A&\geq D\text{ , A Diverges}
\end{align*}

Integral Test:
\begin{align*}
    \int_{}^{\infty} A\;dn
    \begin{cases}
        \text{A finite $\rightarrow$ Converges}\\
        \text{A infinite $\rightarrow$ Diverges}
    \end{cases}
\end{align*}

Ratio Test:
\begin{align*}
    \rho_n&=\bigg \lvert \frac{a_{(n+1)}}{a_n} \bigg \rvert\\
    \rho&=\lim_{n\to \infty} \rho_n
    \begin{cases}
    \rho > 1 \text{ Diverge}\\
    \rho = 0 \text{ Inconclusive}\\
    \rho < 1 \text{ Converges}
    \end{cases}
\end{align*}

Special Comparation:
\begin{align*}
    \lim_{n\to \infty} \frac{A}{C}& \text{ Limit finite $\rightarrow$ A Converges}\\
    \lim_{n\to \infty} \frac{A}{D}& \text{ Limit $>$ 0 $\rightarrow$ A Diverges}
\end{align*}

Raabe Test:
\begin{align*}
    \rho\equiv\lim_{n\to \infty} \bigg[n(\frac{a_n}{a_{n+1}}-1)\bigg]\begin{cases}
        \rho > 1,\textrm{ Converge}\\\rho = 1,\textrm{ Inconclusive}\\\rho < 1,\textrm{ Diverge}
    \end{cases}
\end{align*}

Root Test:
\begin{align*}
    \rho&\equiv\lim_{n\to \infty} \sqrt[ n]{|a_n|}\\
    &\equiv \lim_{n\to \infty} |a_n|^{\frac{1}{n}}
    \begin{cases}
        \rho > 1,\textrm{ Converge}\\\rho = 1,\textrm{ Inconclusive}\\\rho < 1,\textrm{ Diverge}
    \end{cases}
\end{align*}

Alternating series Test
\begin{align*}
    \textrm{if }\quad|a_{n+1}|\leq|a_n|\quad \textrm{and} \quad\lim_{n\to \infty} a_n=0
\end{align*}
then series converges.

\subsection{Function Expansion}
\subsubsection{Taylor series.}
For a function $f(x)$ infinitely differentiable around $x=a$, the Taylor expansion is
\begin{equation*}
    f(x) = \sum_{n=0}^{\infty} \frac{f^{(n)}(a)}{n!}(x-a)^n
\end{equation*}
or 
\begin{equation*}
    f(x)= f(a)+(x-a)f'(a)+\frac{1}{2!}(x-a)^2f''(a)+\frac{1}{3!}(x-a)^3f'''(a)+\cdots    
\end{equation*}

\subsubsection{MacLaurin series.}
A MacLaurin series is a Taylor series centered at $a=0$
\begin{equation*}
    f(x) = \sum_{n=0}^{\infty} \frac{f^{(n)}(0)}{n!}x^n
\end{equation*}
or 
\begin{equation*}
    f(x) = f(0)+(x)f'(0)+\frac{1}{2!}(x)^2f''(0)+\frac{1}{3!}(x)^3f'''(0)+\cdots    
\end{equation*}

\subsubsection{MacLaurin series for basic function.}
\begin{align*}
    \sin x&= \sum_{n=0}^{\infty}\frac{(-1)^nx^{2n+1}}{(2n+1)!} \quad \text{for all }x\\
    \cos x&= \sum_{n=0}^{\infty}\frac{(-1)^nx^{2n}}{(2n)!}\quad \text{for all }x\\
    e^x&= \sum_{n=0}^{\infty}\frac{x^{n}}{n!}               \quad\text{for }-1<x\leq1\\
    \ln (1+x)&= \sum_{n=1}^{\infty} \frac{(-1)^{n+1}x^{n}}{(n)} \quad \text{for all }x\\
    (1+x)^p&=\sum_{n=0}^{\infty}\binom{p}{n}x^n \quad \text{for }|x|<1
\end{align*}
or more explicitly
\begin{align*}
    \sin x&=x-\frac{x^3}{3!}+\frac{x^5}{5!}-\cdots\\
    \cos x&= 1-\frac{x^2}{2!}+\frac{x^4}{4!}+\cdots\\
    e^x&=1+x+\frac{x^2}{2!}+\frac{x^3}{3!} \\
    \ln (1+x)&=x-\frac{x^2}{2}+\frac{x^3}{3}-\frac{x^4}{4}\\
    (1+x)^p&=+px+\frac{p(p-1)}{2!}x^2 +\frac{p(p-1)(p-2)}{3!}x^3+\cdots
\end{align*}
\end{document}