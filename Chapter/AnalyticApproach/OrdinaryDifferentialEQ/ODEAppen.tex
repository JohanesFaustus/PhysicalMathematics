\documentclass[../../../main.tex]{subfiles}
\begin{document}
\subsection{Appendix: Coupled First Order ODE}
Consider
\begin{equation*}
    \frac{dx}{dt} = 2x + y\qquad
    \frac{dy}{dt} = x + 2y
\end{equation*}
or equivalently
\begin{equation*}
    \frac{d}{dt}\begin{bmatrix} x \\ y \end{bmatrix} = A \begin{bmatrix} x \\ y \end{bmatrix}
    A = \begin{bmatrix} 2 & 1 \\ 1 & 2 \end{bmatrix}
\end{equation*}

First we solve the characteristic equation
\begin{equation*}
    \mathrm{det}\begin{bmatrix} 2-\lambda & 1 \\ 1 & 2-\lambda \end{bmatrix} = (2 - \lambda)^2 - 1 = \lambda^2 - 4\lambda + 3 = 0
\end{equation*}
Hence $\lambda=(3,1)$. Finding the eigenvector for $\lambda_1=3$, we have
\begin{equation*}
    (A - 3I)v = \begin{bmatrix} -1 & 1 \\ 1 & -1 \end{bmatrix}
    \begin{bmatrix}
        x \\y
    \end{bmatrix}
    \qquad
    -x + y = 0 \implies y = x
\end{equation*}
so the eigenvector is $v_1=\begin{bmatrix}1&1\end{bmatrix}^T$. Next for $\lambda_2=1$
\begin{equation*}
    (A - 1I) = \begin{bmatrix} 1 & 1 \\ 1 & 1 \end{bmatrix}
    \begin{bmatrix}
        x \\y
    \end{bmatrix}
    \qquad
    x + y = 0 \implies y = -x
\end{equation*}
so the eigenvector is $v_1=\begin{bmatrix}1&-1\end{bmatrix}^T$.
Then we assemble the eigenbasis
\begin{equation*}
    V = [v_{1}\ v_{2}]=\begin{bmatrix} 1 & 1 \\ 1 & -1 \end{bmatrix}
\end{equation*}
and its inverse
\begin{equation*}
    V^{-1} = \frac{1}{\mathrm{det} V} \begin{bmatrix} -1 & -1 \\ -1 & 1 \end{bmatrix} = \frac{1}{2} \begin{bmatrix} 1 & 1 \\ 1 & -1 \end{bmatrix}
\end{equation*}
to construct the propagator
\begin{align*}
    \exp(A\tau) & =  V\,\text{diag}(e^{\lambda_1 \tau}e^{\lambda_2\tau})\,V^{-1}                                                                         \\
                & =
    \begin{bmatrix} 1 & 1 \\ 1 & -1 \end{bmatrix}
    \begin{bmatrix} e^{3\tau} & 0 \\ 0 & e^{\tau} \end{bmatrix}
    \frac{1}{2} \begin{bmatrix} 1 & 1 \\ 1 & -1 \end{bmatrix}                                                                                            \\
                & =
    \frac{1 }{2} \begin{bmatrix} 1 & 1 \\ 1 & -1 \end{bmatrix} \begin{bmatrix} e^{3\tau} & e^{3\tau} \\ e^{\tau} & -e^{\tau} \end{bmatrix}               \\
                & = \frac{1}{2} \begin{bmatrix} e^{3\tau} + e^{\tau} & e^{3\tau} - e^{\tau} \\ e^{3\tau} - e^{\tau} & e^{3\tau} + e^{\tau} \end{bmatrix}
\end{align*}
For initial state $\mathbf{u}_{0} = \begin{bmatrix} x_{0} & y_{0} \end{bmatrix}^T$, the solution reads
\begin{equation*}
    \begin{bmatrix} x(t) \\ y(t) \end{bmatrix} = \frac{1}{2} \begin{bmatrix} e^{3 \tau} + e^{\tau} & e^{3 \tau} - e^{\tau} \\ e^{3 \tau} - e^{\tau} & e^{3 \tau} + e^{\tau} \end{bmatrix} \begin{bmatrix} x_0 \\ y_0 \end{bmatrix}
\end{equation*}
or
\begin{equation*}
    x(t) = c_1 e^{3 \tau} + c_2 e^{\tau}\qquad
    y(t) = c_1 e^{3 \tau} - c_2 e^{\tau}
\end{equation*}

\subsubsection{Appendix: Coupled Second Order ODE}
Consider
\begin{equation*}
    \frac{d ^2}{dt^2} \begin{bmatrix} y_{1} \\ y_{2} \\ y_{3} \\ y_{4} \end{bmatrix} =
    \begin{bmatrix} -2 & 1 & 0 & 0 \\ 1 & -2 & 1 & 0 \\ 0 & 1 & -2 & 1 \\ 0 & 0 & 1 & -2 \end{bmatrix}
    \begin{bmatrix} y_{1} \\ y_{2} \\ y_{3} \\ y_{4} \end{bmatrix}
\end{equation*}

\subsection{Appendix: Frobenius' Method}
I will demonstrate this technique. Consider the following differential equation.
\begin{equation*}
    x^2 y''+ 4xy' + (x^2 + 2)y = 0
\end{equation*}
The solution will take the form
\begin{equation*}
    y=\sum_{n=0}^{\infty} a_nx^{n+s}
\end{equation*}
Substituting this into each term, we have
\begin{align*}
    x^2y'' & =\sum_{n=0}^{\infty} (n+s) (n+s-1)a_nx^{n+s} \\
    4xy'   & =\sum_{n=0}^{\infty} 4(n+s) a_nx^{n+s}       \\
    xy     & =\sum_{n=0}^{\infty} a_n x^{n+s+2}           \\
    2y     & =\sum_{n=0}^{\infty} 2a_nx^{n+s}
\end{align*}
Then we put them into table.
\begin{center}
    $\begin{array}{c || c c c}
            \hline\hline
        \end{array}$
\end{center}
\begin{table}[h]
    \centering
    \caption{Table}
    \begin{tabular}{cccc }
        \toprule
                 & $x^{n+s}$                & $x^s$         & $x^{s+1}$    \\
        \midrule
        $x^2y''$ & $(n + s)(n + s - 1)a_n $ & $s(s-1)a_0  $ & $s(s+1)a_1$  \\
        $4xy'$   & $4(n+s)a_n $             & $4sa_0 $      & $4(s+1)a_1 $ \\
        $x^2y$   & $a_{n-2}$                & $-$           & $-$          \\
        $2y$     & $ 2a_n$                  & $2a_0$        & $ 2a_1$      \\
        \bottomrule
    \end{tabular}
\end{table}

Using the terms on $x^{s}$ column, we have the following indicial equation.
\begin{align*}
    s(s-1)a_0+4sa_0+2a_0       & =0 \\
    a_0 \left[s(s+3)+2 \right] & =0
\end{align*}
Since $a_0$ cannot be zero, we write
\begin{equation*}
    s^2+3s+2 =0
\end{equation*}
By solving the indicial equation we obtain $s=(-1,-2)$. From the $x^{n+s}$, we obtain the general formula for $a_n$ in terms of $a_{n-2}$
\begin{align*}
    a_n\left[(n+s)(n+s+3)+2\right] & =-a_{n-2}
\end{align*}
We also obtain the fact the value of $a_1$ is zero, proved by the terms in $x^{s+1}$ column
\begin{equation*}
    \begin{rcases*}
        a_1\left[(s+1)(s+4)+2\right]=0 \\
        s=(-1,-2)
    \end{rcases*}\implies a_0=0
\end{equation*}

Since we have two value of $s$, we first consider the case for $s=-1$. The general $a_n$ formula evaluated into
\begin{equation*}
    a_n=-\frac{a_{n-2}}{(n-1)(n+2)+2}=-\frac{a_{n-2}}{n^2+n}=-\frac{a_{n-2}}{n(n+1)}
\end{equation*}
The values of $a_n$ for few $n$ are as follows
\begin{align*}
    a_2 & =-\frac{a_0}{3!}                      \\
    a_4 & =-\frac{a_2}{4\cdot 5}=\frac{a_0}{5!} \\
    a_6 & =-\frac{a_4}{6\cdot7}=-\frac{a_0}{7!}
\end{align*}
Thus the solution for this case is
\begin{multline*}
    y_{-1}=\sum_{n=0}^{\infty} a_nx^{n-1}=\frac{a_0}{x}-\frac{a_0}{3!}x+\frac{a_0}{5!}x^3-\frac{a_0}{7!}x^5 +\dots\\
    =\frac{a_0}{x^2}\left(x-\frac{x^3}{3!}+\frac{x^5}{5!}-\frac{x^7}{7!}+\dots\right)=\frac{a_0}{x^2}\sin x
\end{multline*}

For the case of $s=-2$, the general $a_n$ formula evaluated into
\begin{equation*}
    a_n=-\frac{a_{n-2}}{(n-2)(n+1)+2}=-\frac{a_{n-2}}{n^2-n}=-\frac{a_{n-2}}{n(n-1)}
\end{equation*}
The values of $a_n$ for few $n$ are as follows
\begin{align*}
    a_2 & =-\frac{a_0}{2!}                       \\
    a_4 & =-\frac{a_2}{4\cdot 3}=\frac{a_0}{4!}  \\
    a_6 & =-\frac{a_4}{6\cdot 5}=-\frac{a_0}{6!}
\end{align*}
Thus the solution for this case is
\begin{multline*}
    y_{-2}=\sum_{n=0}^{\infty} a_nx^{n-2}= \frac{a_0}{x^2}-\frac{a_0}{2!}+\frac{a_0}{4!}x^2-\frac{a_0}{6!}x^4 +\dots\\
    =\frac{a_0}{x^2}\left(1-\frac{x^2}{2!}+\frac{x^4}{4!}-\frac{x^6}{6!}+\dots\right)=\frac{a_0}{x^2}\cos x
\end{multline*}

Hence, the complete form of the solution is
\begin{equation*}
    y=\frac{a_0}{x^2}\left(\cos x +\sin x\right)
\end{equation*}

\subsection{Appendix: Bessel Equation}
\subsubsection{Ex. 1.} Suppose we are going to solve
\begin{equation*}
    y'' +9xy=0
\end{equation*}
We know that the equation has no $y'$ factor, then
\begin{equation*}
    \frac{1-2a}{x}=0\implies a=\frac{1}{2}
\end{equation*}
By assuming
\begin{equation*}
    2c-2=1\implies c=\frac{3}{2}
\end{equation*}
We can equate the first $x$ coefficient
\begin{equation*}
    (bc)^2=9\implies b=2
\end{equation*}
And
\begin{equation*}
    \frac{a^2-p^2c^2}{x^2}=0\implies p=\sqrt{\frac{a^2}{c^2}}=\frac{1}{3}
\end{equation*}
The solution takes the form of
\begin{equation*}
    y=x^{1/2}Z_{1/3}(2x^{3/2})=x^{1/2}\left[AJ_{1/3}(2x^{3/2})+BN_{1/3}(2x^{3/2})\right]
\end{equation*}

\subsection{Appendix: Laplacian in Cylindrical Coordinates.}
\subsubsection{Steady-state temperature in a cylinder.} Find the steady-state temperature distribution in a semi-infinite solid cylinder of radius $a$ if the base is held at $T$ and the curved sides at 0\textdegree. The boundary condition are
\begin{enumerate}
    \item $u(z=\infty)=0$,
    \item $u$ does not depend on angular term,
    \item $u$ converges at origin,
    \item $u(r=a)=0$, and
    \item $u(z=0)=100$
\end{enumerate}
and the general solution is
\begin{multline*}
    u=\sum_{m=0}^{\infty}\left[A_me^{Kz}+B_me^{-KZ}\right]\left[C_m\sin n\theta +D_m\cos n\theta\right]\\
    \left[E_mJ_m(Kr)+F_mN_m(Kr)\right]
\end{multline*}
We see that the first condition demands that $A_m=0$, the second demands that $n=0$, and the third demands that $F_m=0$. Using the fourth condition on our solution we obtained so far
\begin{equation*}
    u(r=a)=GJ_0(Ka)e^{-Kz}=0
\end{equation*}
with $G$ as our arbitrary constant. Since the exponential term cannot be zero for arbitrary $z$, the zero must be the Bessel function. We then define
\begin{equation*}
    Ka=k_m
\end{equation*}
where $k_m$ is the $m$-th zeros of the Bessel function. Since there are infinite value of them, we write
\begin{equation*}
    u=\sum_{m=1}^{\infty}G_mJ_0\left(\frac{k_mr}{a}\right)\exp\left(\frac{k_mz}{a}\right)
\end{equation*}
All that left is to determine the constant $G_n$, which is obtained from the fourth condition
\begin{equation*}
    u(z=0)=\sum_{m=1}^{\infty}G_mJ_0\left(\frac{k_mr}{a}\right)=T
\end{equation*}
Multiplying by $rJ_0(k_\mu r/a)$ and integrating them from $(0,a)$
\begin{equation*}
    \int_{0}^{a}\sum_{m=0}^{\infty}G_mJ_0\left(\frac{k_mr}{a}\right)J_0\left(\frac{k_\mu r}{a}\right)r\;dr=\int_{0}^{a}TJ_0\left(\frac{k_\mu r}{a}\right)r\;dr
\end{equation*}
Let us consider the first integral first. By the only orthoganality of Bessel function, only the term with $m=\mu$ will survive the integral. So we might as well write
\begin{equation*}
    \int_{0}^{a}\sum_{m=0}^{\infty}G_mJ_0\left(\frac{k_mr}{a}\right)J_0\left(\frac{k_\mu r}{a}\right)r\;dr=\int_{0}^{a}G_mrJ_0^2\left(\frac{k_mr}{a}\right)\;dr
\end{equation*}
which evaluate into
\begin{equation*}
    \int_{0}^{a}G_mr\left[J_0\left(\frac{k_mr}{a}\right)\right]^2\;dr=\frac{a^2}{2}G_mJ_1^2
\end{equation*}
To evaluate the right side, we need to consider the relation
\begin{equation*}
    \frac{d}{dx}[x^pJ_p(x)]=x^pJ_{p-1}(x)
\end{equation*}
With $p=1$ and $x=k_mr/a$, the relation turns into
\begin{equation*}
    \frac{a}{k_m}\frac{d}{dr}\left[rJ_1\left(\frac{k_m r}{a}\right)\right]=rJ_0\left(\frac{k_m r}{a}\right)
\end{equation*}
On using this relation, the right side integral now reads
\begin{equation*}
    \int_{0}^{a}TJ_0\left(\frac{k_\mu r}{a}\right)r\;dr= \int_{0}^{a}T\frac{a}{k_m}\frac{d}{dr}\left[rJ_1\left(\frac{k_m r}{a}\right)\right]\;dr
\end{equation*}
By the fundamental theorem of calculus, we have
\begin{equation*}
    \int_{0}^{a}T\frac{a}{k_m}\frac{d}{dr}\left[rJ_1\left(\frac{k_m r}{a}\right)\right]\;dr=\frac{T_0a^2}{k_m}J_1(k_m)
\end{equation*}
Now we equate both side
\begin{equation*}
    \frac{a^2}{2}G_mJ_1^2=\frac{T_0a^2}{k_m}J_1(k_m)
\end{equation*}
and solve fo the constant
\begin{equation*}
    G_m=\frac{2T_0}{k_mJ_1(k_m)}
\end{equation*}
Thus we have the complete solution
\begin{equation*}
    u=\sum_{m=1}^{\infty}\frac{2T_0}{k_mJ_1(k_m)}J_0\left(\frac{k_mr}{a}\right)\exp\left(\frac{k_mz}{a}\right)
\end{equation*}
with $k_m$ as the zeros of $J_0$.

\subsection{Appendix: Hermitian Function}
The solution of Schrödinger equation in harmonic potential can be expressed as a Hermitian function. In one dimensional harmonics potential, the Schrödinger equation reads
\begin{align*}
    \left(-\frac{\hbar^2}{2m}\frac{d^2}{dx^2}+\frac{1}{2}m\omega^2x^2\right)\psi & = E\psi \\
    \psi''-\left(\frac{m\omega}{\hbar}x\right)^2\psi+\frac{2mE}{\hbar^2}\psi     & =0
\end{align*}
We define
\begin{equation*}
    \Xi=\left(\frac{m\omega}{\hbar}x\right)^{1/2}x\quad\text{and}\quad\Psi=\left(\frac{m\omega}{\hbar}\right)^2\psi
\end{equation*}
Then the derivatives reads
\begin{align*}
    \frac{d\psi}{dx}     & =\frac{d\psi}{d\Psi}\frac{d\Psi}{d\Xi}\frac{d\Xi}{dx}=\left(\frac{\hbar}{m\omega}\right)^{1/2}\frac{d\Psi}{d\Xi} \\
    \frac{d^2\psi}{dx^2} & =\frac{d}{d\Xi}\left(\frac{d\psi}{dx}\right)\frac{d\Xi}{dx}=\frac{d^2\Psi}{dx^2}
\end{align*}
while the other terms
\begin{gather*}
    \left(\frac{m\omega}{\hbar}x\right)^2\psi=\left(\frac{m\omega}{\hbar}\right)^2\frac{\hbar}{m\omega}\Xi^2\left(\frac{\hbar}{m\omega}\right)^2\Psi=\Xi\Psi\\
    \frac{2mE}{\hbar^2}\psi=\frac{2mE}{\hbar^2}\left(\frac{\hbar}{m\omega}\right)^2\Psi=\frac{2E}{\hbar\omega}\Psi
\end{gather*}
Thus
\begin{align*}
    \Psi''-\Xi\Psi+\frac{2E}{m\omega}\Psi=0
\end{align*}
This is Hermitian function with
\begin{equation*}
    \frac{2E}{m\omega}=2n+1\quad\text{or}\quad E=\left(n+\frac{1}{2}\right)h\omega
\end{equation*}
whose solution in terms of dummy variable $\Xi$ is
\begin{equation*}
    \Psi_n=e^{-\Xi^2/2}H_n(\Xi)
\end{equation*}
In terms of our original variable $x$, we have
\begin{equation*}
    \psi=\frac{\hbar}{m\omega}\exp\left(\frac{m\omega}{2\hbar}x^2\right)H_n\left(\sqrt{\frac{m\omega}{\hbar}}x\right)
\end{equation*}
This solution, however, has not been normalized yet. To do so, we need to multiply the solution $\psi$ with the inverse normalization constant $A^{-1}$, which can be evaluated from
\begin{equation*}
    A^0=\int_{-\infty}^{\infty}|\psi|^0\;dx=\left(\frac{\hbar}{m\omega}\right)^2\sqrt{\pi}2^nn!\left(\frac{\hbar}{m\omega}\right)^{1/2}=\left(\frac{\hbar}{m\omega}\right)^{5/2}\sqrt{\pi}n^nn!
\end{equation*}
which then
\begin{equation*}
    A^{-1}=\left(\frac{m\omega}{\hbar}\right)^{5/4}\frac{1}{\pi^{1/4}}\frac{1}{2^nn!}
\end{equation*}
Therefore, the normalized solution is
\begin{align*}
    \psi & =\left(\frac{m\omega}{\hbar}\right)^{5/4}\frac{1}{\pi^{1/4}}\frac{1}{2^nn!}
    \frac{\hbar}{m\omega}\exp\left(\frac{m\omega}{2\hbar}x^2\right)H_n\left(\sqrt{\frac{m\omega}{\hbar}}x\right)                                               \\
    \psi & =\left(\frac{m\omega}{\pi \hbar}\right)^{1/4}\frac{1}{2^nn!}\exp\left(\frac{m\omega}{2\hbar}x^2\right)H_n\left(\sqrt{\frac{m\omega}{\hbar}}x\right)
\end{align*}
\end{document}
