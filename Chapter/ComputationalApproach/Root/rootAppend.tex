\documentclass[../../../main.tex]{subfiles}
\begin{document}
\subsection{Appendix: Algorithm Application}
Let us try to find the root of the following polynomials
\begin{equation*}
    x^2- 2 x + -3  =
    \left(x-3\right)\left(x+1\right)
\end{equation*}

\subsubsection{Bisection method.}
We follow the procedure.
\begin{enumerate}
    \item \textbf{Input.} We define the function as $f(x)=x^2- 2 x + -3$ within interval [2.5;4]
    \item \textbf{Midpoint.} Compute the midpoint:
          \begin{equation*}
              x_t= \frac{2.5+4}{2}=3.25
          \end{equation*}
    \item \textbf{Evaluation and loop.} We evaluate $f(3.25)=1.0625$. Also, we only use 3 loops because the actual root can only be found with 51 loops.
          \begin{enumerate}
              \item \textbf{First loop}. Since $f(x_1)f(x_t)=-1.85<0$, set $x_2=3.25$. Then the midpoint
                    \begin{equation*}
                        x_t=\frac{2.5+3.25}{2}=2.875
                    \end{equation*}
              \item \textbf{Second loop}. Since $f(x_1)f(x_t)=0.84>0$, set $x_1=2.875$. Then the midpoint
                    \begin{equation*}
                        x_t=\frac{2.875+3.25 }{2}=3.0625
                    \end{equation*}
              \item \textbf{Second loop}. Since $f(x_1)f(x_t)=-0.12<0$, set $x_2=3.0625$. Then the midpoint
                    \begin{equation*}
                        x_t=\frac{2.875+3.0625}{2}=2.96875
                    \end{equation*}
                    For this third loop, we return this $x_t=x_r=2.96875$ as the root. Very close with the actual root of $x_r=3$.
          \end{enumerate}
\end{enumerate}

\subsubsection{Newton-Raphson.} We follow the procedure (not that it exists).
We just keep iterating the Newton-Raphson, 3 times to make it equal with bisection.
\begin{enumerate}
    \item \textbf{First loop.} Simply
          \begin{equation*}
              x_r=2.5-\frac{-1.75}{3.000000000019653}=3.083333333329512
          \end{equation*}
    \item \textbf{Second loop.} Like
          \begin{multline*}
              x_r=3.083333333329512-\frac{0.34027777776185597}{4.166666666691188}\\= 3.0016666666671474
          \end{multline*}
    \item \textbf{Third loop.} This
          \begin{multline*}
              x_r=3.0016666666671474-\frac{0.0066694444463691075}{4.003333333360602}\\=3.000000693866234
          \end{multline*}
          Practically the same as the actual root $x_r=3$
\end{enumerate}
\end{document}