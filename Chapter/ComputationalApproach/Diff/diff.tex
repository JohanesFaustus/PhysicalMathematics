\documentclass[../../../main.tex]{subfiles}
\begin{document}
\subsection{Finite Method}
\subsubsection{Forward difference.}
The forward difference of a function $f(x)$ with step size $h$ is defined as
\begin{equation*}
    \Delta f(x)=f(x+h)-f(x)
\end{equation*}
The corresponding approximation of the first derivative is
\begin{equation*}
    f'(x)\approx \frac{f(x+h )-f(x )}{h}
\end{equation*}
Forward difference uses data at the current and next point.

\subsubsection{Backward difference.}
The backward difference of a function $f(x)$ with step size $h$ is defined as
\begin{equation*}
    \nabla f(x)=f(x)-f(x-h)
\end{equation*}
The corresponding approximation of the first derivative is
\begin{equation*}
    f'(x)\approx \frac{f(x )-f(x-h )}{h}
\end{equation*}
Backward difference uses data at the current and previous point.

\subsubsection{Central difference.}
Denoted by
\begin{equation*}
    \delta f(x)=f(x+h/2)-f(x-h/2)
\end{equation*}
with the first derivative as
\begin{equation*}
    f'(x)\approx \frac{f(x+h/2)-f(x-h/2)}{h}
\end{equation*}
or equivalently
\begin{equation*}
    f'(x)\approx \frac{f(x+h)-f(x-h)}{2h}
\end{equation*}

\subsubsection{Taylor series relation.}
Recall the Taylor expansion of $f(x)$
\begin{equation*}
    f(x) = \sum_{n=0}^{\infty} \frac{f^{(n)}(a)}{n!}(x-a)^n
\end{equation*}
Now expand $f(x+h)$ with Taylor expansion by substituting $x=a$ with $a$ as the expansion point
\begin{equation*}
    f(x+h) = \sum_{n=0}^{\infty} \frac{f^{(n)}(a)}{n!}(x+h-a)^n= \sum_{n=0}^{\infty} \frac{f^{(n)}(x)}{n!}h^n
\end{equation*}
Now do the same thing for $f(x-h)$ instead
\begin{equation*}
    f(x+h) = \sum_{n=0}^{\infty} \frac{f^{(n)}(a)}{n!}(x-h-a)^n= \sum_{n=0}^{\infty} (-1)^n\frac{f^{(n)}(x)}{n!}h^n
\end{equation*}
Explicitly, both equation can be written
\begin{align*}
    f(x+h) & =  f(x) + h f'(x) + \frac{h^2}{2} f''(x) + \frac{h^3}{6} f^{(3)}(x) + \cdots \\
    f(x-h) & =  f(x) - h f'(x) + \frac{h^2}{2} f''(x) - \frac{h^3}{6} f^{(3)}(x) + \cdots
\end{align*}

To obtain forward difference, subtract $f(x+h)$ by $f(x)$ and divide by $h$
\begin{equation*}
    \frac{f(x+h) - f(x)}{h} = f'(x) + \frac{h}{2} f''(x) + \frac{h^2}{6} f^{(3)}(x) +\cdots
\end{equation*}
Thus
\begin{equation*}
    f'(x) \approx \frac{f(x+h) - f(x)}{h}, \quad \text{error } = O(h)
\end{equation*}

In order to obtain backward difference, subtract $f(x-h)$ by $f(x)$ and divide by $h$
\begin{equation*}
    \frac{f(x+h) - f(x)}{h} = f'(x) + \frac{h}{2} f''(x) + \frac{h^2}{6} f^{(3)}(x) +\cdots
\end{equation*}
Thus
\begin{equation*}
    f'(x) \approx \frac{f(x) - f(x-h)}{h}, \quad \text{error } = O(h).
\end{equation*}

To obtain central difference, subtract $f(x+h)$ by $f(x-h)$ and divide by $h$
\begin{equation*}
    f(x+h) - f(x-h) = 2h f'(x) + \frac{h^3}{3} f^{(3)}(x) +\cdots
\end{equation*}
then divide by $2h$
\begin{equation*}
    \frac{f(x+h) - f(x-h)}{2h} = f'(x) + \frac{h^2}{6} f^{(3)}(x) +\cdots
\end{equation*}
Thus
\begin{equation*}
    f'(x) \approx \frac{f(x+h) - f(x-h)}{2h}, \quad \text{error } = O(h^2)
\end{equation*}

\subsection{1D Linear Convection}
The equation is written as 
\begin{equation*}
    \frac{\partial u}{\partial t} + c \frac{\partial u}{\partial x} = 0   
\end{equation*}
From the definition of a derivative
\begin{equation*}
    \frac{u_i^{n+1}-u_i^n}{\Delta t} + c \frac{u_i^n - u_{i-1}^n}{\Delta x} = 0 
\end{equation*}
Our discrete equation, then, is
\begin{equation*}
    \frac{u_i^{n+1}-u_i^n}{\Delta t} + c \frac{u_i^n - u_{i-1}^n}{\Delta x} = 0 
\end{equation*}
Where $n$ and $n+1$ are two consecutive steps in time, while $i-1$ and $i$ are two neighboring points of the discretized $x$ coordinate. 
If there are given initial conditions, then the only unknown in this discretization is $u_i^{n+1}$, which can be solved as 
\begin{equation*}
    u_i^{n+1} = u_i^n - c \frac{\Delta t}{\Delta x}(u_i^n-u_{i-1}^n)
\end{equation*}

\subsubsection*{Implementiation.}
Given the initial conditions, the solution can be written in python numerically as  
\begin{minted}[breaklines]{python}
un = numpy.ones(nx) #initialize a temporary array
for n in range(nt):  #loop for values of n from 0 to nt, so it will run nt times
    un = u.copy() ##copy the existing values of u into un
    #for i in range(1, nx): ## you can try commenting this line and...
    for i in range(nx): ## ... uncommenting this line and see what happens!
        u[i] = un[i] - c * dt / dx * (un[i] - un[i-1])
\end{minted}
\end{document}