\documentclass[../../../main.tex]{subfiles}
\begin{document}
\subsection{Appendix: Algorithm Application}
Let us consider the following $3 \times 3$ system in the form $\Omega x=W$
\begin{equation*}
    \begin{bmatrix}
        2  & 1  & -1 \\
        -3 & -1 & 2  \\
        -2 & 1  & 2
    \end{bmatrix}
    \begin{bmatrix}
        x_1 \\ x_2 \\ x_3
    \end{bmatrix}
    =
    \begin{bmatrix}
        8 \\ -11 \\ -3
    \end{bmatrix}
\end{equation*}

\text{The augmented matrix is:}\begin{equation*}
    \quad
    [\Omega|W] =
    \begin{bmatrix}
        2  & 1  & -1 & 8   \\
        -3 & -1 & 2  & -11 \\
        -2 & 1  & 2  & -3
    \end{bmatrix}
\end{equation*}
\subsubsection{Gauss elimination.} 
Following the algorithm.
\begin{enumerate}
    \item \textbf{Pivot in column 1.}
          Using $\omega_{11} = 2$ as pivot, we eliminate below row.
          For row 2:
          \begin{equation*}
              R_2 \leftarrow R_2 - \left(\frac{-3}{2}\right)R_1 \quad \Rightarrow \quad
              R_2=
              \begin{bmatrix}
                  -3 \\-1\\2\\-1\\
              \end{bmatrix}
              - \frac{3 }{2}
              \begin{bmatrix}
                  2 \\1\\-1\\8
              \end{bmatrix}
              =
              \begin{bmatrix}
                  0 \\ 0.5 \\ 0.5 \\ 1
              \end{bmatrix}
          \end{equation*}
          For row 3:
          \begin{equation*}
              R_3 \leftarrow R_3 - \left(\frac{-2}{2}\right)R_1 \quad \Rightarrow \quad
              R_3=
              \begin{bmatrix}
                  -2 \\1\\2\\-3\\
              \end{bmatrix}
              +
              \begin{bmatrix}
                  2 \\1\\-1\\8
              \end{bmatrix}
              =
              \begin{bmatrix}
                  0 \\ 2 \\ 1 \\ 5
              \end{bmatrix}
          \end{equation*}
          So the matrix becomes:
          \begin{equation*}
              \begin{bmatrix}
                  2 & 1   & -1  & 8 \\
                  0 & 0.5 & 0.5 & 1 \\
                  0 & 2   & 1   & 5
              \end{bmatrix}
          \end{equation*}
    \item \textbf{Pivot in column 2.}
          Pivot is $\omega_{22}^{(1)} = 0.5$.
          Eliminate below, only row 3 though
          \begin{equation*}
              R_3 \leftarrow R_3 - \frac{2}{0.5} R_2 \quad \Rightarrow \quad
              R_3=
              \begin{bmatrix}
                  0 \\2\\1\\5\\
              \end{bmatrix}
              -4
              \begin{bmatrix}
                  - \\0.5\\0.5\\1
              \end{bmatrix}
              =
              \begin{bmatrix}
                  0 \\0\\-1\\1
              \end{bmatrix}
          \end{equation*}
          Thus:
          \begin{equation*}
              \begin{bmatrix}
                  2 & 1   & -1  & 8 \\
                  0 & 0.5 & 0.5 & 1 \\
                  0 & 0   & -1  & 1
              \end{bmatrix}
          \end{equation*}

    \item \textbf{Step 3: Pivot in column 3.}
          Pivot is $\omega_{33}^{(2)} = -1$. No elimination is needed because there are no rows below.
          The system is now in upper triangular form.
    \item \textbf{Back substitution:}
          From the last row:
          \begin{equation*}
              - x_3 = 1 \quad \Rightarrow \quad x_3 = -1.
          \end{equation*}

          From the second row:
          \begin{equation*}
              0.5x_2 + 0.5x_3 = 1 \quad \Rightarrow \quad 0.5x_2 - 0.5 = 1 \quad \Rightarrow \quad x_2 = 3.
          \end{equation*}

          From the first row:
          \begin{equation*}
              2x_1 + x_2 - x_3 = 8 \quad \Rightarrow \quad 2x_1 + 3 - (-1) = 8 \quad \Rightarrow \quad x_1 = 2.
          \end{equation*}
    \item \textbf{Final solution:}
          \begin{equation*}
              x =
              \begin{bmatrix}
                  2 \\ 3 \\ -1
              \end{bmatrix}
          \end{equation*}
\end{enumerate}

\subsubsection{Gaussian-Jordan elimination.} 
Following the procedure.
\begin{enumerate}
    \item \textbf{Decomposition.}
          For $k=1$ (that is, compute first row of $U$ and first column of $L$). The formula for $U$  with $i=1$ yields empty sum
          \begin{equation*}
              u_{1j} = \omega_{1j}\Rightarrow
              \begin{cases}
                  u_{11}=2 \\u_{12}=1\\u_{13}=-1
              \end{cases}
          \end{equation*}
          For the first column of $L$ the sum is empty and $u_{11}=2$, hence
          \begin{equation*}
              l_{21} = \frac{\omega_{21}}{u_{11}} = -\frac{3}{2}, \qquad
              l_{31} = \frac{\omega_{31}}{u_{11}} = -1.
          \end{equation*}
          Now for $k=2$, the second row of $U$ reads
          \begin{equation*}
              u_{2j}=\omega_{2j}-\sum_{k=1 }^{1}l_{2k}u_{kj}=\omega_{2j}-l_{21}u_{1j}\Rightarrow
              \begin{cases}
                  u_{22} = \frac{1}{2}  \\
                  u_{23}  = \frac{1}{2} \\
              \end{cases}
          \end{equation*}
          and the second column of $L$
          \begin{equation*}
              l_{32} = \frac{1}{u_{22}} \left( \omega_{32} -\sum_{k=1 }^{1} l_{3k}u_{k2} \right)=\frac{1}{1/2}\left(1 - (-1)\cdot 1\right) = 4
          \end{equation*}
          And finally for $k=3$, the third row of $U$ reads
          \begin{multline*}
              u_{33} =\omega_{33}-\sum_{k=1 }^{2}l_{3k}u_{k3}= \omega_{33} - (l_{31}u_{13} + l_{32}u_{23})\\
               = 2 - \big( (-1)(-1) + 4\cdot \frac{1}{2} \big) = -1.
          \end{multline*}
    \item \textbf{Result.} We have
          \begin{equation*}
              L=\begin{bmatrix}
                  1            & 0 & 0 \\
                  -\frac{3}{2} & 1 & 0 \\
                  -1           & 4 & 1
              \end{bmatrix},
              \qquad
              U=\begin{bmatrix}
                  2 & 1           & -1          \\
                  0 & \frac{1}{2} & \frac{1}{2} \\
                  0 & 0           & -1
              \end{bmatrix}.
          \end{equation*}
    \item \textbf{Forward substitution.} From the equation $Ly=W$, the first row reads
          \begin{equation*}
              y_1 = w_1 = 8
          \end{equation*}
          Next the second row
          \begin{equation*}
              -\frac{3}{2}y_1 + y_2 = w_2 \qquad\Rightarrow\qquad y_2 = -11 + \frac{3}{2}\cdot 8 = 1
          \end{equation*}
          And the third
          \begin{equation*}
              - y_1 + 4y_2 + y_3 = w_3 \qquad\Rightarrow\qquad y_3 = -3 + 8 - 4 = 1
          \end{equation*}
          This yields
          \begin{equation*}
              y = \begin{bmatrix} 8 \\ 1 \\ 1 \end{bmatrix}
          \end{equation*}
    \item \textbf{Backwards substitution.} From the equation $Ux=y$, the first row reads
          \begin{equation*}
              - x_3 = y_3 \qquad\Rightarrow\qquad x_3 = -1
          \end{equation*}
          Next the second row
          \begin{equation*}
              \frac{1}{2}x_2 + \frac{1}{2}x_3 = y_2 \qquad\Rightarrow\qquad x_2 =2 \left(1 + \frac{1}{2} \right)
          \end{equation*}
          And the third
          \begin{equation*}
              2x_1 + x_2 - x_3 = y_1 \qquad\Rightarrow\qquad x_1 =\frac{1 }{2} (8 - 3 - 1) = 4 
          \end{equation*}
          This yields our solution
          \begin{equation*}
              x = \begin{bmatrix}2 \\ 3 \\ -1\end{bmatrix}.
          \end{equation*}
\end{enumerate}

\subsubsection{LU decomposition.} 
Using the upper triangular matrix obtained by the Gaussian elimination 
\begin{equation*}
    \begin{bmatrix}
        2 & 1   & -1  & 8 \\
        0 & 0.5 & 0.5 & 1 \\
        0 & 0   & -1  & 1
    \end{bmatrix}
\end{equation*}
we follow the order.
\begin{enumerate}
    \item \textbf{Normalization.}
          For the first row,
          \begin{equation*}
              R_1 \leftarrow \frac{1 }{2 }R_1\qquad\Rightarrow\qquad R_1=
              \begin{bmatrix}
                  1 \\1/2\\-1/2\\4
              \end{bmatrix}
          \end{equation*}
          For the second row,
          \begin{equation*}
              R_2 \leftarrow 2R_2 \qquad\Rightarrow\qquad R_2=
              \begin{bmatrix}
                  0 \\1\\1\\2
              \end{bmatrix}
          \end{equation*}
          For the third row,
          \begin{equation*}
              R_3 \leftarrow -R_3 \qquad\Rightarrow\qquad R_3=
              \begin{bmatrix}
                  0 \\0\\1\\-1
              \end{bmatrix}
          \end{equation*}
          So the matrix becomes:
          \begin{equation*}
              \begin{bmatrix}
                  1 & 1/2 & -1/2 & 4 \\
                  0 & 1   & 1    & 2 \\
                  0 & 0   & 1    & 1
              \end{bmatrix}
          \end{equation*}
    \item \textbf{Elimination above the pivot.}
          For pivot in row 3
          \begin{equation*}
              \begin{cases}
                  R_2\leftarrow R_2-R_3 \\
                  R_1\leftarrow R_1-\frac{1 }{2}R_2
              \end{cases}
              \qquad \Rightarrow \qquad
              A=
              \begin{bmatrix}
                  1 & 1/2 & 0 & 3.5 \\
                  0 & 1   & 0 & 3   \\
                  0 & 0   & 1 & 1
              \end{bmatrix}
          \end{equation*}
          For pivot in row 2
          \begin{equation*}
              R_1 \leftarrow R_1-\frac{1 }{2 }R_2
              \qquad \Rightarrow \qquad
              A=
              \begin{bmatrix}
                  1 & 0 & 0 & 2 \\
                  0 & 1 & 0 & 3 \\
                  0 & 0 & 1 & 1
              \end{bmatrix}
          \end{equation*}
    \item \textbf{Final solution:} The same as Gaussian elimination
          \begin{equation*}
              x =
              \begin{bmatrix}
                  2 \\ 3 \\ -1
              \end{bmatrix}
          \end{equation*}
\end{enumerate}
\end{document}