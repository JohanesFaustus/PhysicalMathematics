\documentclass[../../../main.tex]{subfiles}
\begin{document}
Many methods solving linear system here uses the definition of augmented matrix.
Consider the linear  equation in the form
\begin{equation*}
    \Omega x=W
\end{equation*}
with $\Omega$ as is the coefficient matrix, $x$ column vector of unknowns, and $W$ column vector of constants.
The augmented matrix $[\Omega|W]$ is simply the coefficient matrix $\Omega$ with the constant vector $W$ appended as an extra column.

For example a system in the form $\Omega x=W$
\begin{equation*}
    \begin{cases}
        \omega_{11}x_1 + \omega_{12}x_2 + \dots + \omega_{1n}x_n & = w_1, \\
                                                                 & \vdots \\
        \omega_{n1}x_1 + \omega_{n2}x_2 + \dots + \omega_{nn}x_n & = w_n.
    \end{cases}
\end{equation*}
Has the augmented matrix
\begin{equation*}
    [\Omega|W] =
    \begin{bmatrix}
        \omega_{11} & \omega_{12} & \cdots & \omega_{1n} & | & w_1    \\
        \omega_{21} & \omega_{22} & \cdots & \omega_{2n} & | & w_2    \\
        \vdots      & \vdots      & \ddots & \vdots      &   & \vdots \\
        \omega_{n1} & \omega_{n2} & \cdots & \omega_{nn} & | & w_n
    \end{bmatrix}
\end{equation*}

\subsection{Gaussian Elimination}
Gaussian elimination is an algorithm in linear algebra for solving systems of linear equations, computing the rank of a matrix, and finding the inverse of invertible matrices.
The algorithm is as follows.

\begin{enumerate}
    \item \textbf{Step 1: Pivot in column 1.}
          Choose $\omega_{11}$ as pivot (if $\omega_{11}=0$, swap with a lower row).
          For $i=2,\dots,n$, eliminate entries below the pivot.
          \begin{equation*}
              R_i \;\leftarrow\; R_i - \frac{a_{i1}}{a_{11}} R_1
          \end{equation*}
    \item \textbf{Step 2: Pivot in column 2.}
          The new pivot is $a_{22}^{(1)}$ (the updated entry in row 2, column 2).
          For $i=3,\dots,n$, eliminate the entries below.
          \begin{equation*}
              R_i \;\leftarrow\; R_i - \frac{a_{i2}^{(1)}}{a_{22}^{(1)}} R_2
          \end{equation*}
    \item \textbf{Step 3: Continue the process.}  At the $k$-th step, the pivot is $a_{kk}^{(k-1)}$.
          For each $i = k+1, \dots, n$.
          \begin{equation*}
              R_i \;\leftarrow\; R_i - \frac{a_{ik}^{(k-1)}}{a_{kk}^{(k-1)}} R_k
          \end{equation*}
    \item \textbf{Result:}
          After $n-1$ steps, the matrix is in row echelon (upper triangular) form:
          \[
              \begin{bmatrix}
                  \omega_{11}^{*} & \omega_{12}^{*} & \cdots & \omega_{1n}^{*} & w_1^{*} \\
                  0               & \omega_{22}^{*} & \cdots & \omega_{2n}^{*} & w_2^{*} \\
                  0               & 0               & \cdots & \omega_{3n}^{*} & w_3^{*} \\
                  \vdots          & \vdots          & \ddots & \vdots          & \vdots  \\
                  0               & 0               & \cdots & \omega_{nn}^{*} & w_n^{*}
              \end{bmatrix}
          \]

    \item \textbf{Back substitution:}
          The last row yields $\omega_{nn}^{*} x_n = w_n^{*}$, hence
          \[
              x_n = \frac{w_n^{*}}{\omega_{nn}^{*}}
          \]
          Substituting upward gives each preceding variable until $x_1$ is obtained.
          Thus the solution vector $x$ is fully determined.
\end{enumerate}

\subsubsection{Example.}
Let us consider the following $3 \times 3$ system in the form $\Omega x=W$
\begin{equation*}
    \begin{bmatrix}
        2  & 1  & -1 \\
        -3 & -1 & 2  \\
        -2 & 1  & 2
    \end{bmatrix}
    \begin{bmatrix}
        x_1 \\ x_2 \\ x_3
    \end{bmatrix}
    =
    \begin{bmatrix}
        8 \\ -11 \\ -3
    \end{bmatrix}
\end{equation*}

\text{The augmented matrix is:}\begin{equation*}
    \quad
    [\Omega|W] =
    \begin{bmatrix}
        2  & 1  & -1 & 8   \\
        -3 & -1 & 2  & -11 \\
        -2 & 1  & 2  & -3
    \end{bmatrix}
\end{equation*}
\begin{enumerate}
    \item \textbf{Pivot in column 1.}
          Using $\omega_{11} = 2$ as pivot, we eliminate below row.
          For row 2:
          \begin{equation*}
              R_2 \leftarrow R_2 - \left(\frac{-3}{2}\right)R_1 \quad \Rightarrow \quad
              R_2=
              \begin{bmatrix}
                -3\\-1\\2\\-1\\
              \end{bmatrix}
              - \frac{3 }{2}
              \begin{bmatrix}
                2\\1\\-1\\8
              \end{bmatrix}
              =
              \begin{bmatrix}
                  0 \\ 0.5 \\ 0.5 \\ 1
              \end{bmatrix}
          \end{equation*}
          For row 3:
          \begin{equation*}
              R_3 \leftarrow R_3 - \left(\frac{-2}{2}\right)R_1 \quad \Rightarrow \quad
                R_3=
              \begin{bmatrix}
                -2\\1\\2\\-3\\
              \end{bmatrix}
              +
              \begin{bmatrix}
                2\\1\\-1\\8
              \end{bmatrix}
              =
              \begin{bmatrix}
                  0 \\ 2 \\ 1 \\ 5
              \end{bmatrix}
          \end{equation*}
          So the matrix becomes:
          \begin{equation*}
              \begin{bmatrix}
                  2 & 1   & -1  & 8 \\
                  0 & 0.5 & 0.5 & 1 \\
                  0 & 2   & 1   & 5
              \end{bmatrix}
          \end{equation*}
    \item \textbf{Pivot in column 2.}
          Pivot is $\omega_{22}^{(1)} = 0.5$.
          Eliminate below, only row 3 though
          \begin{equation*}
              R_3 \leftarrow R_3 - \frac{2}{0.5} R_2 \quad \Rightarrow \quad
                R_3=
              \begin{bmatrix}
                0\\2\\1\\5\\
              \end{bmatrix}
              -4
              \begin{bmatrix}
                -\\0.5\\0.5\\1
              \end{bmatrix}
              =
              \begin{bmatrix}
                0\\0\\-1\\1
              \end{bmatrix}
          \end{equation*}
          Thus:
          \begin{equation*}
              \begin{bmatrix}
                  2 & 1   & -1  & 8 \\
                  0 & 0.5 & 0.5 & 1 \\
                  0 & 0   & -1  & 1
              \end{bmatrix}
          \end{equation*}

    \item \textbf{Step 3: Pivot in column 3.}
          Pivot is $\omega_{33}^{(2)} = -1$. No elimination is needed because there are no rows below.
          The system is now in upper triangular form.
    \item \textbf{Back substitution:}
          From the last row:
          \begin{equation*}
              - x_3 = 1 \quad \Rightarrow \quad x_3 = -1.
          \end{equation*}

          From the second row:
          \begin{equation*}
              0.5x_2 + 0.5x_3 = 1 \quad \Rightarrow \quad 0.5x_2 - 0.5 = 1 \quad \Rightarrow \quad x_2 = 3.
          \end{equation*}

          From the first row:
          \begin{equation*}
              2x_1 + x_2 - x_3 = 8 \quad \Rightarrow \quad 2x_1 + 3 - (-1) = 8 \quad \Rightarrow \quad x_1 = 2.
          \end{equation*}
    \item \textbf{Final solution:}
          \begin{equation*}
              \quad
              x =
              \begin{bmatrix}
                  2 \\ 3 \\ -1
              \end{bmatrix}
          \end{equation*}
\end{enumerate}


\end{document}