\documentclass[../../../main.tex]{subfiles}
\begin{document}
Interpolation is a method used to construct a function that exactly passes through a given set of data points. It assumes that the data are precise and free from measurement noise.
The primary goal is to reconstruct or estimate the values of the underlying function at intermediate points within the range of the observed data.
Unlike regression, which tolerates deviations between the model and the data due to noise, interpolation enforces an exact fit at each known point.

\subsection{Lagrange Interpolation}
Lagrange interpolation is a polynomial interpolation method that constructs a unique polynomial of degree at most $n$ passing exactly through $n+1$ given data points \((x_i, F_i)\) for \(i = 1, \dots, N\).
The interpolating polynomial is written as
\[
    P_n(x) = \sum_{i=0}^{n} y_i L_i(x)
\]
where \( L_i(x) \) are the Lagrange basis polynomials defined by
\[
    L_i(x) = \prod_{\substack{j=0 \\ j \ne i}}^{n} \frac{x - x_j}{x_i - x_j}
\]

\end{document}