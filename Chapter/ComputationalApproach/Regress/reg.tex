\documentclass[../../../main.tex]{subfiles}
\begin{document}

Regression, conversely, is employed when the data contain measurement noise or intrinsic variability, and the objective is to infer the underlying trend or functional relationship rather than to reproduce every observation.
It is ubiquitous in statistics, experimental physics, and data analysis.
Unlike interpolation which used for exact reconstruction of a presumed error-free function, regression is used for approximate modeling of noisy or empirical relationships

\subsection{Linear Regression (unfinished)}
The simplest form models the dependent variable \( y \) as a linear combination of independent variables \( x_j \):
\begin{equation*}
    y = \beta_0 + \beta_1 x_1 + \beta_2 x_2 + \cdots + \varepsilon,
\end{equation*}
where the coefficients \( \beta_j \) are estimated by minimizing the sum of squared residuals.

\subsection{Multiple Linear Regression (unfinished)}
A generalization of linear regression with several independent variables.
The same formulation holds but with \( p > 1 \) predictors.

\subsection{Polynomial Regression}
For $N$ distinct data points, the highest possible polynomial degree $m$ that can be exactly fitted is $N-1$.
\begin{equation*}
    \begin{cases}
        m<N-1 & \text{overdetermined, minimizes the deviation} \\
        m=N-1 & \text{determined, becomes interpolation}       \\
        m>N-1 & \text{underdetermined, solution is non-unique}
    \end{cases}
\end{equation*}

From a mathematical and statistical standpoint, increasing the number of data points $N$ generally provides a more reliable and physically meaningful model than increasing the polynomial degree $m$

The algorithm is simply as follows.
\begin{enumerate}
    \item Give data points \((x_i, F_i)\) for \(i = 1, \dots, N\) and a polynomial model of degree \(m\),
    \item Constructs the design matrix
          \begin{equation*}
              \mathbf{F} =
              \begin{bmatrix}
                  F_1 \\ F_2 \\ \vdots \\ F_N
              \end{bmatrix},
              \qquad
              \mathbf{X} =
              \begin{bmatrix}
                  1      & x_1    & x_1^2  & \cdots & x_1^m  \\
                  1      & x_2    & x_2^2  & \cdots & x_2^m  \\
                  \vdots & \vdots & \vdots & \ddots & \vdots \\
                  1      & x_N    & x_N^2  & \cdots & x_N^m
              \end{bmatrix},
              \qquad
              \boldsymbol{\beta} =\begin{bmatrix}
                  \beta_0 \\ \beta_1 \\ \beta_2\\ \ldots \\ \beta_m
              \end{bmatrix}
          \end{equation*}
    \item Solve the normal equation
          \begin{equation*}
              \mathbf{X}^{T}\mathbf{X}\boldsymbol{\beta} = \mathbf{X}^{T}\mathbf{F}
          \end{equation*}
          using whatever system of linear equation solver.
    \item Use the solution to constructs
          \begin{equation*}
              \mathbf{F} = \sum_{i=0}^{m} \beta_i x^i
          \end{equation*}
\end{enumerate}

\subsubsection{Derivation.}
Consider a set of $N$ data points $(x_i,F_i)$ where $x_i$ is the independent variable and $F_i$ the measured dependent variable.
The aim is to approximate the data by a polynomial of degree $m$
\begin{equation*}
    F(x) = \beta_0 + \beta_1 x + \beta_2 x^2 + \beta_3 x^3 + \cdots + \beta_m x^m,
\end{equation*}
where the coefficients \(\boldsymbol{\beta} =\begin{bmatrix}
    \beta_0 & \beta_1 & \ldots & \beta_m
\end{bmatrix}^{T}\) are to be determined.
In matrix form, the system can be written as
\begin{equation*}
    \mathbf{X}\boldsymbol{\beta}=\mathbf{F}
\end{equation*}
where
\begin{equation*}
    \mathbf{F} =
    \begin{bmatrix}
        F_1 \\ F_2 \\ \vdots \\ F_N
    \end{bmatrix},
    \qquad
    \mathbf{X} =
    \begin{bmatrix}
        1      & x_1    & x_1^2  & \cdots & x_1^m  \\
        1      & x_2    & x_2^2  & \cdots & x_2^m  \\
        \vdots & \vdots & \vdots & \ddots & \vdots \\
        1      & x_N    & x_N^2  & \cdots & x_N^m
    \end{bmatrix},
    \qquad
    \boldsymbol{\beta} =\begin{bmatrix}
        \beta_0 \\ \beta_1 \\ \beta_2\\ \ldots \\ \beta_m
    \end{bmatrix}
\end{equation*}
This can be solved by rewriting it using normal equation $    \mathbf{X}^{T}\mathbf{X}\hat{\boldsymbol{\beta}} = \mathbf{X}^{T}\mathbf{F}$.
We write the term $    \mathbf{X}^{T}\mathbf{X}$ as
\begin{align*}
    \mathbf{X}^T\mathbf{X} & =
    \begin{bmatrix}
        1      & 1      & 1      & \cdots & 1      \\
        x_1    & x_2    & x_3^2  & \cdots & x_N^2  \\
        \vdots & \vdots & \vdots & \ddots & \vdots \\
        x_1^m  & x_2^m  & x_3^m  & \cdots & x_N^m
    \end{bmatrix}
    \begin{bmatrix}
        1      & x_1    & x_1^2  & \cdots & x_1^m  \\
        1      & x_2    & x_2^2  & \cdots & x_2^m  \\
        \vdots & \vdots & \vdots & \ddots & \vdots \\
        1      & x_N    & x_N^2  & \cdots & x_N^m
    \end{bmatrix} \\
    \mathbf{X}^T\mathbf{X} & =
    \begin{bmatrix}
        N            & \sum x_i       & \sum x_i^2     & \sum x_i^3     & \cdots & \sum x_i^m     \\
        \sum x_i     & \sum x_i^2     & \sum x_i^3     & \sum x_i^4     & \cdots & \sum x_i^{m+1} \\
        \sum x_i^2   & \sum x_i^3     & \sum x_i^4     & \sum x_i^5     & \cdots & \sum x_i^{m+2} \\
        \sum x_i^3   & \sum x_i^4     & \sum x_i^5     & \sum x_i^6     & \cdots & \sum x_i^{m+3} \\
        \vdots       & \vdots         & \vdots         & \vdots         & \ddots & \vdots         \\
        \sum x_i^{m} & \sum x_i^{m+1} & \sum x_i^{m+2} & \sum x_i^{m+3} & \cdots & \sum x_i^{2m}
    \end{bmatrix}
\end{align*}
And, the $\mathbf{X}^T \mathbf{F}$ can be written as
\begin{align*}
    \mathbf{X}^T \mathbf{F} & =
    \begin{bmatrix}
        1      & 1      & 1      & \cdots & 1      \\
        x_1    & x_2    & x_3^2  & \cdots & x_N^2  \\
        \vdots & \vdots & \vdots & \ddots & \vdots \\
        x_1^m  & x_2^m  & x_3^m  & \cdots & x_N^m
    \end{bmatrix}
    \begin{bmatrix}
        F_1 \\ F_2 \\ \vdots \\ F_N
    \end{bmatrix}  \\
    \mathbf{X}^T \mathbf{F} & =
    \begin{bmatrix}
        \sum F_i \\ \sum x_i F_i \\ \sum x_i^2 F_i \\ \sum x_i^3 F_i \\ \vdots \\ \sum x_i^m F_i
    \end{bmatrix}
\end{align*}
As such we have
\begin{equation*}
    \begin{bmatrix}
        N            & \sum x_i       & \sum x_i^2     & \cdots & \sum x_i^m     \\
        \sum x_i     & \sum x_i^2     & \sum x_i^3     & \cdots & \sum x_i^{m+1} \\
        \sum x_i^2   & \sum x_i^3     & \sum x_i^4     & \cdots & \sum x_i^{m+2} \\
        \sum x_i^3   & \sum x_i^4     & \sum x_i^5     & \cdots & \sum x_i^{m+3} \\
        \vdots       & \vdots         & \vdots         & \ddots & \vdots         \\
        \sum x_i^{m} & \sum x_i^{m+1} & \sum x_i^{m+2} & \cdots & \sum x_i^{2m}
    \end{bmatrix}
    \begin{bmatrix}
        \beta_0 \\
        \beta_1 \\
        \beta_2 \\
        \beta_3 \\
        \vdots  \\
        \beta_m
    \end{bmatrix}
    =
    \begin{bmatrix}
        \sum F_i \\ \sum x_i F_i \\ \sum x_i^2 F_i \\ \sum x_i^3 F_i \\ \vdots \\ \sum x_i^m F_i
    \end{bmatrix}
\end{equation*}

\subsubsection{Normal equation.}
The normal equation is obtained by minimizing the sum of squared deviations between the observed data and the fitted model.
The derivation is as follows.
The least-squares objective function is defined as
\begin{equation*}
    S(\boldsymbol{\beta}) = \sum_{i=1}^{N}
    \left[ F_i - \left( \beta_0 + \beta_1 x_i + \beta_2 x_i^2 + \cdots + \beta_m x_i^m \right) \right]^2.
\end{equation*}
In compact matrix notation, this can be written as
\begin{equation*}
    S(\boldsymbol{\beta}) = (\mathbf{F} - \mathbf{X}\boldsymbol{\beta})^{T}(\mathbf{F} - \mathbf{X}\boldsymbol{\beta}).
\end{equation*}

To find the optimal coefficients \(\hat{\boldsymbol{\beta}}\) that minimize \(S\), one takes the gradient of \(S\) with respect to \(\boldsymbol{\beta}\):
\begin{equation*}
    \nabla_{\boldsymbol{\beta}} S =
    \frac{\partial S}{\partial \boldsymbol{\beta}}
    = -2\mathbf{X}^{T}\mathbf{F} + 2\mathbf{X}^{T}\mathbf{X}\boldsymbol{\beta}.
\end{equation*}

Setting the gradient equal to zero yields the condition for the minimum:
\begin{equation*}
    \nabla_{\boldsymbol{\beta}} S = 0
    \quad \Longrightarrow \quad
    \mathbf{X}^{T}\mathbf{X}\hat{\boldsymbol{\beta}} = \mathbf{X}^{T}\mathbf{F}
\end{equation*}

These are known as the normal equations.
Provided that \(\mathbf{X}^{T}\mathbf{X}\) is invertible, the least-squares estimate of the coefficient vector is obtained as
\begin{equation*}
    \hat{\boldsymbol{\beta}} = (\mathbf{X}^{T}\mathbf{X})^{-1}\mathbf{X}^{T}\mathbf{F}.
\end{equation*}

\subsection{Regularized Regression (unfinished)}
To prevent overfitting or manage multicollinearity, penalty terms are introduced:
\begin{align*}
    \text{Ridge:} & \quad \min_{\boldsymbol{\beta}} \left[ \sum_i (y_i - \mathbf{x}_i^T \boldsymbol{\beta})^2 + \lambda \|\boldsymbol{\beta}\|_2^2 \right], \\
    \text{Lasso:} & \quad \min_{\boldsymbol{\beta}} \left[ \sum_i (y_i - \mathbf{x}_i^T \boldsymbol{\beta})^2 + \lambda \|\boldsymbol{\beta}\|_1 \right].
\end{align*}

\subsection{Logistic Regression (unfinished)}
Despite its name, logistic regression is used for binary classification.
It models the probability of one class as
\begin{equation*}
    P(y=1|x) = \frac{1}{1 + e^{-(\beta_0 + \beta_1 x_1 + \cdots)}}.
\end{equation*}

\subsection{Nonlinear Regression (unfinished)}
When the model depends nonlinearly on its parameters, for instance:
\begin{equation*}
    y = A e^{-Bx} + C,
\end{equation*}
the parameters \( A, B, C \) cannot be solved by linear algebra and must be obtained through iterative optimization.

\subsection{Robust Regression (unfinished)}
To reduce the influence of outliers, the least-squares criterion is replaced by alternative loss functions, such as the absolute deviation.

\subsection{Bayesian Regression (unfinished)}
Regression parameters are treated as random variables with prior distributions.
Given data, posterior distributions of parameters are obtained via Bayes' theorem.

\subsection{Gaussian Process Regression (unfinished)}
A nonparametric Bayesian approach in which a Gaussian process defines a prior over possible functions.
It is especially useful in physical modeling when data are sparse.

\end{document}