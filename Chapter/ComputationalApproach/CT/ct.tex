\documentclass[../../../main.tex]{subfiles}
\begin{document}
\section{Beer's Law}
The general form of Beer’s law describes how a beam of radiation is attenuated as it propagates through an absorbing medium.
In three-dimensional space, when the attenuation coefficient is not constant but varies with position, the law generalizes to a path integral form
\begin{equation*}
    I=I_0\exp \left( \int_S \mu(\mathbf{r })\;ds \right)
\end{equation*}

To obtain a form suitable for tomographic reconstruction, one takes the natural logarithm and rearranges
\begin{equation*}
    \ln\!\left(\frac{I_0}{I}\right) = \int_{S} \mu(\mathbf{r}) \, ds.
\end{equation*}
The left-hand side is experimentally measurable, since $I$ and $I_0$ are detector readings.
The right-hand side is the line integral of the attenuation coefficient along the X-ray path
Defining the projection function $p(\theta, t)$, one obtains
\begin{equation}
    p(\theta, t) 
    = \int_{\mathcal{L}(\theta,t)} \mu(x,y) \, ds,
\end{equation}
where $\theta$ is the projection angle and $t$ denotes the detector position. 
\end{document}